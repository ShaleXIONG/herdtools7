%%%%%%%%%%%%%%%%%%%%%%%%%%%%%%%%%%%%%%%%%%%%%%%%%%%%%%%%%%%%%%%%%%%%%%%%%%%%%%%%
\chapter{Lexical Structure\label{chap:LexicalStructure}}
%%%%%%%%%%%%%%%%%%%%%%%%%%%%%%%%%%%%%%%%%%%%%%%%%%%%%%%%%%%%%%%%%%%%%%%%%%%%%%%%
This chapter defines the various elements of an ASL specification text in a high-level way,
and then formalizes the lexical analysis as a function that takes a text and returns
either a list of \emph{tokens} or a lexical error.

\ChapterOutline
\begin{itemize}
  \item \secref{ASL Specification Text} defines what constitutes the text of an ASL specification,
  \item \secref{Lexical Regular Expressions} defines the regular expressions used to define the lexical structure,
  \item \secref{Whitespace} defines whitespace,
  \item \secref{Comments} defines comments,
  \item \secref{Integer Literals} defines integer literals,
  \item \secref{Rational Number Literals} defines rational number literals,
  \item \secref{Boolean Literals} defines Boolean literals,
  \item \secref{Bitmasks} defines bitmasks,
  \item \secref{String Literals} defines string literals,
  \item \secref{LexicalIdentifiers} defines identifiers, and
  \item \secref{LexicalAnalysis} defines the analysis used to transform text into tokens.
\end{itemize}

\section{ASL Specification Text\label{sec:ASL Specification Text}}
An ASL specification is a string, that is a list of ASCII characters, consisting of a \emph{content text}
followed by an \emph{end-of-file}.
The content text is a list of
ASCII characters that have the decimal encoding of 32 through 126 (inclusive),
which includes the space character (decimal encoding 32),
as well as
carriage return (decimal encoding 13) and line feed (decimal encoding 10).
\hypertarget{def-eof}{}
The end of file character is denoted by $\eof$.
The content text does not contain an end-of-file character.

\RequirementDef{TabCharacter}
% NO_EXAMPLE
In particular, it is an error to use a tab character in ASL specification text (decimal encoding 9).

\section{Lexical Regular Expressions\label{sec:Lexical Regular Expressions}}

\hypertarget{def-regex}{}
\taref{LexicalRegularExpressions} defines the regular expressions $\RegExp$ used to define
\emph{lexemes} --- substrings of the ASL specification text that are used to form \emph{tokens}.

\begin{table}
\caption{Lexical Regular Expressions \label{ta:LexicalRegularExpressions}}
\begin{center}
\begin{tabular}{ll}
\hline
\textbf{RegExp} & \textbf{Matches}\\
\hline
\anycharacter{\texttt{a\_string}}     & Any character in \texttt{a\_string}\\
$\square$                             & The space character (decimal 32) \hypertarget{def-ascii}{}\\
\ascii{$a$}                           & The ASCII with decimal $a$\\
\ascii{$a$-$b$}                       & The ASCII range between decimals $a$ and $b$\\
(\texttt{$A$})                        & $A$\\
$A$ $B$                               & $A$ followed by $B$\\
\texttt{$A$ | $B$}                    & A or B \hypertarget{def-regexminus}{}\\
\texttt{$A\ \regexminus{}\, B$}       & $A$ but not $B$\\
$A*$                                  & Zero or more repetitions of A\\
$A+$                                  & One or more repetitions of A\\
\verb|"a_string"|                     & The string \verb|a_string| verbatim\\
\texttt{<}r\texttt{>}                 & The lexical regular expression defined for \texttt{<}r\texttt{>}\\
\hline
\end{tabular}
\end{center}
\end{table}

Let $\REasciichar$ stand for any ASCII character:
\hypertarget{def-reasciichar}{}
\begin{center}
\begin{tabular}{rcl}
$\REasciichar$  &$\triangleq$& \ascii{0-255}
\end{tabular}
\end{center}

Let $\REchar$ stand for an ASCII character that may appear in the content text:
\hypertarget{def-rechar}{}
\begin{center}
\begin{tabular}{rcl}
$\REchar$       &$\triangleq$& \ascii{10} \texttt{|} \ascii{13} \texttt{|} \ascii{32-126}\\
\end{tabular}
\end{center}

\hypertarget{def-lang}{}
The notation $\Lang(e)$ stands for \emph{formal language} of a regular expression $e$.
That is, the set of strings that match that regular expression.

\section{Whitespace\label{sec:Whitespace}}
\RequirementDef{Whitespace}
Comments and whitespace characters are collectively referred to as whitespace lexemes
and are discarded. Technically, line comments and multi-line comments are treated
separately from whitespace characters.

For example,
the specification in \listingref{syntax-whitespace1}
is equivalent to \listingref{syntax-whitespace2}
in the sense that they both parse to the same AST.
Of course, \listingref{syntax-whitespace2} is preferred in terms of style.
\ASLListing{A badly-formatted specification}{syntax-whitespace1}{\syntaxtests/GuideRule.Whitespace1.asl}
\ASLListing{A well-formatted specification}{syntax-whitespace2}{\syntaxtests/GuideRule.Whitespace2.asl}

Formally, whitespace lexemes other than comments are defined via the following regular expression:
\hypertarget{def-rewhitespace}{}
\begin{center}
\begin{tabular}{rcl}
$\REwhitespace$ & $\triangleq$ & (\ascii{10} $|$ \ascii{13} $|$ \ascii{32})+\\
\end{tabular}
\end{center}

\section{Comments\label{sec:Comments}}
ASL supports comments in the style of C++:
\begin{itemize}
\item Single-line comments: the text from \text{//} until the end of the line
is a comment (\ascii{10} is the line feed character \verb|\n|).
\item Multi-line comments: the text between \texttt{/*} and \texttt{*/} is a comment.
\end{itemize}
Comments do not nest and the two styles of comments do not interact with each other,
as exemplified in \listingref{LexicalComments}.

\ASLListing{Examples of comments}{LexicalComments}{\syntaxtests/Lexical.Comments.asl}

Line comments are formally defined via the following regular expressions:
\hypertarget{def-relinecomment}{}
\begin{center}
\begin{tabular}{rcl}
$\RElinecomment$  &$\triangleq$& \texttt{"//"} (\REchar\ \regexminus{} \ascii{10})* \texttt{|} \texttt{"/*"} \REchar* \texttt{"*/"}\\
\end{tabular}
\end{center}

Multi-line comments are defined via $\actionstartcomment$ and in \secref{scanningmultilinecomments}.

\section{Integer Literals\label{sec:Integer Literals}}
Integers are written either in decimal using one or more of the characters \texttt{0-9} and underscore, or in hexadecimal
using \texttt{0x} at the start followed by the characters \texttt{0-9, a-f, A-F} and underscore. An integer literal cannot start with
an underscore.

This is formalized by the following lexical regular expression:
\hypertarget{def-redigit}{}
\hypertarget{def-reintlit}{}
\hypertarget{def-rehexlit}{}
\hypertarget{def-forbiddenhexfirst}{}
\hypertarget{def-forbiddenhexremaining}{}
\hypertarget{def-rehexletter}{}
\begin{center}
\begin{tabular}{rcl}
$\REdigit$  &$\triangleq$& \anycharacter{\texttt{0123456789}}\\
$\REhexletter$ &$\triangleq$& \anycharacter{\texttt{abcdefABCDEF}}\\
$\REintlit$ &$\triangleq$& \texttt{\REdigit\ (\Underscore\ | \REdigit)*}\\
$\REhexlit$ &$\triangleq$& \texttt{"0x"} (\REdigit\ \texttt{|} \REhexletter) \\
            &            & $\wrappedline$ (\Underscore\ \texttt{|} \REdigit\ \texttt{|} \REhexletter)* \\
$\REforbiddenhexfirst$ &$\triangleq$& \texttt{"0x"} ((\REchar\ \regexminus{} \REdigit) \regexminus{} \REhexletter) \\
$\REforbiddenhexremaining$ &$\triangleq$& \texttt{"0x"} (\REdigit\ \texttt{|} \REhexletter)  \\
                           &            & $\wrappedline$ (\Underscore\ \texttt{|} \REdigit\ \texttt{|} \REhexletter)*  \\
                           &            & $\wrappedline$ (\REletter{} \regexminus{} \REhexletter) \\
\end{tabular}
\end{center}

\section{Rational Number Literals\label{sec:Rational Number Literals}}
Rational numbers are written in decimal and consist of one or more decimal digits, a decimal point and one
or more decimal digits. Underscores can be added between digits to aid readability

Underscores in numbers are not significant, and their only purpose is to separate groups of digits to make constants
such as \texttt{0xefff\_fffe}, \texttt{1\_000\_000} or \texttt{3.141\_592\_654} easier to read,

The patterns $\REforbiddenrealfirst$ and $\REforbiddenrealremaining$ are
forbidden and trigger a lexical error.

\hypertarget{def-reallit}{}
\hypertarget{def-forbiddenrealfirst}{}
\hypertarget{def-forbiddenrealremaining}{}
This is formalized by the following lexical regular expression:
\begin{center}
\begin{tabular}{rcl}
$\REreallit$ &$\triangleq$& \texttt{\REdigit\ (\Underscore\ | \REdigit)* "." \REdigit\ (\Underscore\ | \REdigit)*} \\
$\REforbiddenrealfirst$ &$\triangleq$& \texttt{\REdigit\ (\Underscore\ | \REdigit)* "." (\REchar\ \regexminus{} \anycharacter{0123456789.})} \\
$\REforbiddenrealremaining$ &$\triangleq$& \texttt{\REdigit\ (\Underscore\ | \REdigit)* "." \REdigit\ (\Underscore\ | \REdigit)* \REletter{}} \\
\end{tabular}
\end{center}

\section{Boolean Literals\label{sec:Boolean Literals}}
Boolean literals are written using \texttt{TRUE} or \texttt{FALSE}.

\section{Bitvector Literals\label{Bitvector Literals}}
Constant bitvectors are written using 1, 0 and spaces surrounded by single-quotes.

We first define a regular expression for a bit character:
\hypertarget{def-rebit}{}
\begin{center}
\begin{tabular}{rcl}
$\REbit$ &$\triangleq$& \anycharacter{\texttt{01}\square}
\end{tabular}
\end{center}

Now, we define a regular expression for bitvector literals:
\hypertarget{def-rebitvectorlit}{}
\begin{center}
\begin{tabular}{rcl}
$\REbitvectorlit$ &$\triangleq$& \anycharacter{\texttt{'}} \REbit* \anycharacter{\texttt{'}}
\end{tabular}
\end{center}

The spaces in a bitvector are not significant and are only used to improve readability.
For example, \texttt{'1111 1111 1111 1111'} is the same as \texttt{'1111111111111111'}.

\section{Bitmasks\label{sec:Bitmasks}}
Constant bitmasks are written using \texttt{1}, \texttt{0}, \texttt{x}, and \texttt{0}s/\texttt{1}s enclosed in parentheses.
Any \texttt{x}s or characters enclosed in parentheses represent don't care characters.

\hypertarget{def-rebitmasklit}{}
\begin{center}
\begin{tabular}{rcl}
$\REbitmasklit$ &$\triangleq$& \anycharacter{\texttt{'}} \big(
  $\REbit$ \texttt{|}
  \anycharacter{\texttt{x}} \texttt{|}
  \anycharacter{\texttt{(}} $\REbit$+ \anycharacter{\texttt{)}}
  \big)* \anycharacter{\texttt{'}}
\end{tabular}
\end{center}

The spaces in a constant bitmask are not significant and are only used to improve readability.
Similarly, the specific characters enclosed within parentheses are not significant, as each \texttt{0} or \texttt{1} is equivalent to an \texttt{x}.
For example, the bitmask \verb|'0xx1'| is equivalent to \verb|'0(00)1'|, \verb|'0(01)1'|, \verb|'0(10)1'|, and \verb|'0(11)1'|.
Similarly, \verb|'0xx(1)'| is equivalent to \verb|'0xxx'|.

\RequirementDef{BitmasksMSB}
The most significant bit (\ProseMSB) of a bitmask is its leftmost value and
the least significant bit (\ProseLSB) is its rightmost value.
% INLINED_EXAMPLE
For example, in \verb|'10'| the \ProseMSB{} is \verb|1| and the \ProseLSB{} is \verb|0|.

\section{String Literals\label{sec:String Literals}}
String literals consist of printable characters surrounded by double quotes.
They are used to create string values, which are strings of zero or more characters,
where a character is a printable ASCII character,
tab (ASCII code \texttt{9}),
newline (ASCII code \texttt{10}),
the backslash character (ASCII code \texttt{92}),
and double-quote character (ASCII code \texttt{34}).
Unprintable characters (tabs and newlines) are not permitted in string literals,
so they are represented by treating the backslash character \textbackslash, as an escape character.
Note therefore that string literals cannot span multiple source lines.

The escape sequences allowed in string literals appear in \taref{EscapeSequences}.
\begin{table}
\caption{Escape Sequences in String Literals\label{ta:EscapeSequences}}
\begin{center}
\begin{tabular}{ll}
\hline
\textbf{Escape sequence} & \textbf{Meaning}\\
\hline
\verb|\n| & The newline, ASCII code \texttt{10}\\
\verb|\t| & The tab, ASCII code \texttt{9}\\
\verb|\\| & The backslash character, \verb|\|, ASCII code \texttt{92}\\
\verb|\"| & The double-quote character, \texttt{"}, ASCII code \texttt{34}\\
\hline
\end{tabular}
\end{center}
\end{table}

\hypertarget{def-restringlit}{}
\hypertarget{def-restrchar}{}
\begin{center}
\begin{tabular}{rcl}
$\REstrchar$ &$\triangleq$& \ascii{32-126}\\
$\REstringlit$ &$\triangleq$& \anycharacter{\texttt{"}} ( ($\REstrchar$ \regexminus{} \anycharacter{\texttt{"}\ \backslash\ }) $|$ (\anycharacter{\backslash\ } \anycharacter{\texttt{" n t }\backslash\ })  )* \anycharacter{\texttt{"}}
\end{tabular}
\end{center}

\section{Identifiers\label{sec:LexicalIdentifiers}}
Identifiers start with a letter or underscore and continue with zero or more letters, underscores or digits.
Identifiers are case sensitive.
\hypertarget{def-reletter}{}
\hypertarget{def-reidentifier}{}
\begin{center}
\begin{tabular}{rcl}
$\REletter$ &$\triangleq$& \anycharacter{\texttt{abcdefghijklmnopqrstuvwxyz}} $|$ \anycharacter{\texttt{ABCDEFGHIJKLMNOPQRSTUVWXYZ}}\\
$\REidentifier$ &$\triangleq$& \texttt{($\REletter$ $|$ $\underbracket{\texttt{ \_ } }$) ($\REletter$ $|$ $\underbracket{\texttt{ \_ } }$ $|$ $\REdigit$)*}\\
\end{tabular}
\end{center}

An enumeration literal is classed as a literal value
(see \chapref{Literals}), but is syntactically an identifier.

Tuple element selectors are classed as identifiers.
%
For example, \texttt{item0} is classed as an identifier
in the expression \verb|(1, 2).item0|.
\identi{TSXL}

\RequirementDef{ReservedIdentifiers}
Identifiers that begin with double-underscore are reserved for use
by the typechecker and should
not be used in specifications \\
(see \LexicalRuleRef{ReservedIdentifiers}).

For example, \listingref{reserved-identifiers} shows an illegal specification that uses the reserved
identifier \verb|__internal_var| in an attempt to declare a local variable,
which yields a lexical error by the lexical analysis.
\ASLListing{A specification using a reserved identifier}{reserved-identifiers}{\syntaxtests/GuideRule.ReservedIdentifiers.bad.asl}

\RequirementDef{IdentifiersKeywords}
Keywords cannot be used as identifiers.

For example, the specification in \listingref{identifiers-keywords}
uses the keyword \verb|case| in an attempt to declare a local variable,
which yields a lexical error by the lexical analysis.
\ASLListing{A specification using a keyword for an identifier}{identifiers-keywords}{\syntaxtests/GuideRule.IdentifiersKeywords.bad.asl}

\ConventionDef{IdentifiersDifferingByCase}
To improve readability, it is recommended to avoid the use of identifiers that differ
only by the case of some characters.
%
For example, the specification in \listingref{IdentifiersDifferingByCase}
may confuse readers and the variable names \verb|color| and \verb|Color| are better renamed.
\ASLListing{Two identifiers differing only by case}{IdentifiersDifferingByCase}{\syntaxtests/ConventionRule.IdentifiersDifferingByCase.asl}

\ConventionDef{IdentifierSingleUnderscore}
Any identifier with a single underscore followed by an alphanumeric character
is treated as a normal identifier. We recommend that these are only
for use by platform-specific code, which should not clash with the rest of a
portable ASL specification.
%
For example, \listingref{IdentifierSingleUnderscore}
can be useful for declaring a constant specific to the platform \verb|my_platform|.
\ASLListing{An identifier starting with a single underscore}{IdentifierSingleUnderscore}{\syntaxtests/ConventionRule.IdentifierSingleUnderscore.asl}

\section{Lexical Analysis\label{sec:LexicalAnalysis}}
Lexical analysis, which is also referred to as \emph{scanning}, is defined via the function
\hypertarget{def-aslscan}{}
\[
\aslscan : \LexSpec \times \KleeneStar{\REasciichar} \aslto (\KleeneStar{\Token} \cup \{\LexicalErrorConfig\})
\]
\hypertarget{def-lexicalerrorresult}{}
which takes a \emph{lexical specification} (explained soon), an ASL specification string
(where characters are simply numbers representing ASCII characters)
and returns a sequence of tokens (tokens are defined below) or a \emph{lexical error} $\LexicalErrorConfig$.

Tokens have one of two forms:
\begin{description}
  \item[Value-carrying] Tokens that carry value have the form $L(v)$ where $L$ is a token label,
        signifying the meaning of the token, and $v$ is a value carried by the token,
        which is used to construct the respective Abstract Syntax Tree nodes.
  \item[Valueless] Tokens that do not carry values have the form $L$ where $L$ is a token label.
\end{description}

\hypertarget{def-token}{}
The set of tokens used for the lexical analysis of ASL strings is defined below.
\[
\begin{array}{rcll}
\Token &\triangleq& \{\ \Tintlit(n) \;|\; n\in\Z\ \} & \cup\\
        & & \{\ \Treallit(q) \;|\; q\in\Q\ \} & \cup\\
        & & \{\ \Tstringlit(s) \;|\; s\in \Lang(\REstringlit) \} & \cup\\
        & & \{\ \Tstringchar(c) \;|\; c \in \Lang(\REchar) \} & \cup\\
        & & \{\ \Tstringend \} & \cup\\
        & & \{\ \Tbitvectorlit(b) \;|\; b\in\KleeneStar{\Bit}\ \} & \cup\\
        & & \{\ \Tmasklit(m) \;|\; m\in\KleeneStar{\{0,1,x\}}\ \} & \cup\\
        & & \{\ \Tboollit(\True), \Tboollit(\False)\ \} & \cup \\
        & & \{\ \Tidentifier(\id) \;|\; \id\in \Lang(\REidentifier)\ \} & \cup \\
        & & \{\ \Tlexeme(s) \;|\; s\in\Strings\ \} & \cup \\
        & & \{\ \Twhitespace, \Teof, \Terror\ \} &
\end{array}
\]

\hypertarget{def-tintlit}{}
\begin{itemize}
  \item Tokens of the form $\Tintlit(n)$ represent integer literals; \hypertarget{def-treallit}{}
  \item Tokens of the form $\Treallit(q)$ represent rational number literals; \hypertarget{def-tstringlit}{}
  \item Tokens of the form $\Tstringlit(s)$ represent string literals;
  \hypertarget{def-tstringchar}{}
  \item Tokens of the form $\Tstringchar(c)$ represent a single character in a string literal;
  \hypertarget{def-tstringend}{}
  \item The token $\Tstringend$ represents the closing quotes of a string literal;
  \hypertarget{def-tbitvectorlit}{}
  \item Tokens of the form $\Tbitvectorlit(b)$ represent bitvector literals; \hypertarget{def-tmasklit}
  \item Tokens of the form $\Tmasklit(m)$ represent constant bitmasks; \hypertarget{def-tboollit}{}
  \item Tokens of the form $\Tboollit(b)$ represent Boolean literals; \hypertarget{def-tidentifier}{}
  \item Tokens of the form $\Tidentifier(i)$ represent identifiers; \hypertarget{def-tlexeme}{}
  \item Tokens with the label $\Tlexeme$ are ones where the value $s$ is simply the \emph{lexeme} for that token.
  That is, the substring representing that token. Later we will refer to such token by simply quoting
  the lexeme of the token and dropping the label, for brevity. For example, instead of $\Tlexeme(\texttt{for})$,
  we will write $\Tfor$. \hypertarget{def-twhitespace}{}
  \item The valueless token $\Twhitespace$ represents white spaces; \hypertarget{def-terror}{}
  \item The valueless token $\Terror$ represents an illegal lexeme such as the use of a reserved keyword;
  \hypertarget{def-teof}{}
  \item The valueless token $\Teof$ represents $\eof$.
\end{itemize}

\hypertarget{def-lexspec}{}
\begin{definition}[Lexical Specification]
A \emph{lexical specification} consists of a list of pairs $[(r_1,a_1),\ldots,(r_k,a_k)] \in \LexSpec$
where each pair $(r_i,a_i)$ consists of a lexical regular expression $r_i$
and a \emph{lexeme action} $a_i : \Strings\times\Strings \aslto \KleeneStar{\Token}$.
\end{definition}

\hypertarget{def-rematch}{}
The function
\[
\remaxmatch : \overname{\RegExp}{e} \times \overname{\Strings}{s} \aslto (\overname{\Strings}{s_1} \times \overname{\Strings}{s_2}) \cup \{\bot\}
\]
returns the \emph{longest} match of a regular expression $e$ for a prefix of a string $s$.
More precisely:
$\remaxmatch(e, s) = (s_1,s_2)$ means that $s_1\in\Lang(e)$ and $s = s_1 \concat s_2$.
If no match exists, it is indicated by returning $\bot$.

\hypertarget{def-maxmatch}{}
The function $\maxmatches : \overname{\LexSpec}{R} \times \overname{\Strings}{s} \aslto \overname{\LexSpec}{R'}$
returns the sublist of $R$ consisting of pairs whose maximal matches for $s$ are equal. Importantly, the result sublist $R'$ maintains
the order of pairs in $R$. If all expressions in $R$ do not match (that is $\remaxmatch$ returns $\bot$ for all pairs in $R$), then $R'$ is the empty list.

The function $\aslscan$ is constructively defined via the following inference rules:

\begin{mathpar}
\inferrule[no\_match]{
  \maxmatches(R, s) = \emptylist
}{
  \aslscan(R, s) \scanarrow \LexicalErrorConfig
}
\end{mathpar}

\begin{mathpar}
\inferrule[token]{
  \maxmatches(R, s) = [(e_1,a_1),\ldots,(e_n,a_n)]\\
  \remaxmatch(s, e_1) = (s_1, s_2)\\
  a_1(s_1, s_2) \scanarrow \ts \terminateas \LexicalErrorConfig
}{
  \aslscan(R, s) \scanarrow \ts
}
\end{mathpar}

This form of scanning is referred to as ``Maximal Munch'' in Compiler Theory
and is the most common form of scanning.
See ``Compilers: Principles, Techniques, and Tools''~\cite{ASU86} for more details.

While Maximal Munch is a useful policy for scanning of most tokens,
it does not work well for string literals and multi-line comments, which require
identifying the respective tokens via shortest match.
%
For this purpose, most lexical analyzers split the analysis into separate ``states'' ---
one for keywords, symbols, single-line comments, and identifiers, one for string literals,
and one for multi-line comments. The lexical analyzers switches between the states as
needed, and analyzing string literals involves concatenating the individual characters
of the string literal into a single token.

Lexical analysis of ASL follows this approach by defining three specifications:
\begin{itemize}
  \item $\spectoken$: For keywords, symbols, single-line comments, and identifiers;
  \item $\speccomment$: For multi-line comments;
  \item $\specstring$: For string literals.
\end{itemize}

Additionally, lexical analysis of string literals carries the extra state ---
the string characters encountered along the way.

The rest of this section defines each of the lexical specifications and related lexeme actions.

Each lexical specification is depicted by a table where the order of elements
of a specification corresponds to the order of rows in the table.

\subsection{Scanning Regular Tokens}
To scan keywords, symbols, single-line comments, and identifiers,
we define the following lexeme actions:

\hypertarget{def-actiondiscard}{}
\begin{itemize}
\item
The lexeme action
\hypertarget{def-discard}{}
\[
\actiondiscard(s_1, s_2) \triangleq \aslscan(\spectoken, s_2)
\]
discards the string $s_1$ and continues scanning $s_2$ with $\spectoken$.
This is used for whitespace lexemes.

\item
\hypertarget{def-actiontoken}{}
The lexeme action
\[
\begin{array}{l}
\actiontoken(f) \triangleq \lambda (s_1,s_2). \\
\wrappedline\
\begin{cases}
  \LexicalErrorConfig & \text{if }f(s_1) = \Terror \text{ or}\\
   & \aslscan(\spectoken, s_2) = \Terror\\
  [f(s_1)] \concat \aslscan(\spectoken, s_2) & \text{else}
\end{cases}
\end{array}
\]

is parameterized by a function $f$ that converts strings into corresponding tokens.
It applies $f$ to convert $s_1$ into a token and then continues scanning $s_2$ with \\
$\spectoken$.
If at any point a lexical error is encountered, the entire result is a lexical error.

\item
\hypertarget{def-actionstartstring}{}
The lexeme action
\[
\actionstartstring(s_1, s_2) \triangleq \scanstring(\emptylist, s_2)
\]
switches to scanning literal strings via $\scanstring$.

\item The lexeme action
\hypertarget{def-actionstartcomment}{}
\[
\actionstartcomment(s_1, s_2) \triangleq \aslscan(\speccomment, s_2)
\]
switches to scanning multi-line comments by changing the lexical specification
to $\speccomment$.

\hypertarget{def-decimaltolit}{}
\item The function $\decimaltolit(s)$ returns $\Tintlit(n)$ where $n$ is the integer represented by $s$
by decimal representation.
\hypertarget{def-hextolit}{}
\item The function $\hextolit(s)$ returns $\Tintlit(n)$ where $n$ is the integer represented by $s$
by hexadecimal representation.
\hypertarget{def-realtolit}{}
\item The function $\realtolit(s)$ returns $\Treallit(q)$ where $q$ is the rational number for
$s$, which is given via a floating point representation.
\hypertarget{def-strtolit}{}
\item The function $\strtolit(s)$ returns $\Tstringlit(s')$ where $s'$ is the string value represented by $s$.
\hypertarget{def-bitstolit}{}
\item The function $\bitstolit(s)$ returns $\Tbitvectorlit(b)$ where $b$ is the sequence of bits
      given by $s$.
\hypertarget{def-masktolit}{}
\item The function $\masktolit(s)$ returns $\Tmasklit(m)$ where $m$ is the bitmask given by $s$.
\hypertarget{def-falsetolit}{}
\item The function $\falsetolit(s)$ returns $\Tboollit(\False)$ ($s$ is guaranteed to be \\
      \texttt{FALSE}).
\hypertarget{def-truetolit}{}
\item The function $\truetolit(s)$ returns $\Tboollit(\True)$ ($s$ is guaranteed to be \\
      \texttt{TRUE}).
\hypertarget{def-tokenid}{}
\item The function $\tokenid(s)$ returns $\Tlexeme(s)$.
\hypertarget{def-lexicalerror}{}
\item The function $\lexicalerror$ returns $\Terror$.
\hypertarget{def-eoftoken}{}
\item The lexeme action
\[
\eoftoken(s_1, s_2) \triangleq \begin{cases}
  \emptylist & s_2 = \emptylist\\
  \LexicalErrorConfig & \text{else}
\end{cases}
\]
checks whether $\eof$ is not followed by more characters and returns a lexical error otherwise.
\end{itemize}

\LexicalRuleDef{ReservedIdentifiers}
\hypertarget{def-toidentifier}{}
The function $\toidentifier(s)$ checks whether $s$ is a legal identifier
and if so returns $\Tidentifier(s)$. Otherwise, the result is a lexical error.

\ProseParagraph
Given a string $s$, $\toidentifier(s)$ checks whether $s$ starts with a double
underscore. If so, the result is a lexical error. Otherwise the result is
$\Tidentifier(s)$.

\FormallyParagraph
\[
  \toidentifier(s) = \begin{cases}
    \LexicalError & \text{if }s = \underbracket{\texttt{ \_ } }\ \underbracket{\texttt{ \_ } } s'\\
    \Tidentifier(s) & \text{else}
  \end{cases}
\]

\subsection{Regular Tokens Tables}

\hypertarget{def-spectoken}{}
The lexical specification  $\spectoken$ is given by the following four tables.
Splitting the lexical specification into four tables is done for presentation purposes ---
the ordering between the entries is induced by the order between the tables
and the order of entries in each table.
%
When several regular expressions are listed in a row, it means that they are all associated with the same
token function.

\begin{center}
\begin{tabular}{ll}
\textbf{Lexical Regular Expressions} & \textbf{Lexeme Action}\\
\hline
$\REwhitespace$                       & $\discard$ \\
$\RElinecomment$                      & $\discard$ \\
$\texttt{"/*"}$                       & $\actionstartcomment$ \\
\anycharacter{\texttt{"}}             & $\actionstartstring$ \\
$\REintlit$                           & $\actiontoken(\decimaltolit)$ \\
$\REhexlit$                           & $\actiontoken(\hextolit)$ \\
$\REreallit$                          & $\actiontoken(\realtolit)$ \\
$\REstringlit$                        & $\actiontoken(\strtolit)$ \\
$\REbitvectorlit$                     & $\actiontoken(\bitstolit)$ \\
$\REbitmasklit$                       & $\actiontoken(\masktolit)$ \\
\texttt{"!"}, \texttt{","}, \texttt{"<"}, \shiftrightlexeme, \texttt{"\&\&"}, \texttt{"==>"}, \shiftleftlexeme{}                         & $\actiontoken(\tokenid)$  \\
\texttt{"]"}, \texttt{"]]"}, \texttt{")"}, \texttt{".."}, \texttt{"="}, \texttt{"\{"}, \texttt{"!="}, \texttt{"-"}, \texttt{"<=>"}                        & $\actiontoken(\tokenid)$  \\
\texttt{"["}, \texttt{"[["}, \texttt{"("}, \texttt{"."}, \texttt{"<="}, \texttt{"\textasciicircum"}, \texttt{"*"}, \texttt{"/"}                          & $\actiontoken(\tokenid)$  \\
\texttt{"=="}, \texttt{"||"}, \texttt{"+"}, \texttt{":"}, \texttt{"=>"}, \texttt{"<=>"}                         & $\actiontoken(\tokenid)$  \\
\texttt{"\}"}, \texttt{"++"}, \texttt{">"}, \texttt{"+:"}, \texttt{"*:"}, \texttt{";"}, \texttt{">="}                         & $\actiontoken(\tokenid)$  \\
\hline
\end{tabular}
\end{center}

\begin{center}
\begin{tabular}{ll}
\textbf{Lexical Regular Expressions} & \textbf{Lexeme Action}\\
\hline
\texttt{"accessor"}, \texttt{"AND"}, \texttt{"array"}, \texttt{"as"}, \texttt{"assert"},      & $\actiontoken(\tokenid)$ \\
\texttt{"begin"}, \texttt{"bit"}, \texttt{"bits"}, \texttt{"boolean"}       & $\actiontoken(\tokenid)$ \\
\texttt{"case"}, \texttt{"catch"}, \texttt{"collection"}, \texttt{"config"}, \texttt{"constant"}      & $\actiontoken(\tokenid)$ \\
\texttt{"DIV"}, \texttt{"DIVRM"}, \texttt{"do"}, \texttt{"downto"}        & $\actiontoken(\tokenid)$ \\
\texttt{"else"}, \texttt{"elsif"}, \texttt{"end"}, \texttt{"enumeration"}   & $\actiontoken(\tokenid)$ \\
\texttt{"XOR"}           & $\actiontoken(\tokenid)$ \\
\texttt{"exception"}     & $\actiontoken(\tokenid)$ \\
\texttt{"FALSE"} & $\actiontoken(\falsetolit)$  \\
\texttt{"for"}, \texttt{"func"}          & $\actiontoken(\tokenid)$ \\
\texttt{"getter"}        & $\actiontoken(\tokenid)$ \\
\texttt{"if"}, \texttt{"impdef"}, \texttt{"implementation"}, \texttt{"IN"}, \texttt{"integer"}       & $\actiontoken(\tokenid)$ \\
\texttt{"let"}, \texttt{"looplimit"}           & $\actiontoken(\tokenid)$ \\
\texttt{"MOD"}           & $\actiontoken(\tokenid)$ \\
\texttt{"noreturn"}, \texttt{"NOT"}           & $\actiontoken(\tokenid)$ \\
\texttt{"of"},      \texttt{"OR"},      \texttt{"otherwise"}                  & $\actiontoken(\tokenid)$ \\
\texttt{"pass"},    \texttt{"pragma"},  \texttt{"print"}, \texttt{"pure"}                      & $\actiontoken(\tokenid)$ \\
\texttt{"readonly"}, \texttt{"real"},    \texttt{"record"}  & $\actiontoken(\tokenid)$ \\
\texttt{"recurselimit"}, \texttt{"repeat"}, \texttt{"return"}  & $\actiontoken(\tokenid)$ \\
\texttt{"setter"},  \texttt{"string"},  \texttt{"subtypes"}                   & $\actiontoken(\tokenid)$ \\
\texttt{"then"},    \texttt{"throw"},   \texttt{"to"}, \texttt{"try"}         & $\actiontoken(\tokenid)$ \\
\texttt{"TRUE"}          & $\actiontoken(\truetolit)$ \\
\texttt{"type"}          & $\actiontoken(\tokenid)$ \\
\texttt{"ARBITRARY"}, \texttt{"unreachable"}, \texttt{"until"}         & $\actiontoken(\tokenid)$ \\
\texttt{"var"}           & $\actiontoken(\tokenid)$ \\
\texttt{"when"}, \texttt{"where"}, \texttt{"while"}, \texttt{"with"}          & $\actiontoken(\tokenid)$ \\
\hline
\end{tabular}
\end{center}

The following list represents forbidden lexemes in ASL:
\begin{center}
  \begin{tabular}{ll}
    \textbf{Lexical Regular Expression} & \textbf{Lexeme Action}\\
    \hline
    $\REforbiddenhexfirst$ & $\lexicalerror$ \\
    $\REforbiddenhexremaining$ & $\lexicalerror$ \\
    $\REforbiddenrealfirst$ & $\lexicalerror$ \\
    $\REforbiddenrealremaining$ & $\lexicalerror$ \\
    \hline
  \end{tabular}
\end{center}

\ExampleDef{Illegal Numeric Lexemes}
\begin{center}
\begin{tabular}{ll}
\textbf{illegal lexeme} & \textbf{reason}\\
\hline
\verb|5.a| & fails $\REforbiddenrealfirst$\\
\verb|5.2a| & fails $\REforbiddenrealremaining$\\
\verb|0xs| & fails $\REforbiddenhexfirst$\\
\verb|0xa1s| & fails $\REforbiddenhexremaining$\\
\end{tabular}
\end{center}
\begin{center}
\begin{tabular}{ll}

\textbf{Lexical Regular Expression} & \textbf{Lexeme Action}\\
\hline
$\REidentifier$   & $\actiontoken(\toidentifier)$ \\
$\eof$            & $\eoftoken$ \\
\hline
\end{tabular}
\end{center}

\subsection{Scanning Strings}
\hypertarget{def-scanstring}{}
To scan string literals, we define the following specialized scanning function.
The function
\[
\scanstring : \overname{\KleeneStar{\REasciichar}}{\buff} \times \overname{\KleeneStar{\REasciichar}}{s} \aslto (\KleeneStar{\Token} \cup \{\LexicalErrorConfig\})
\]
scans string with the $\specstring$ specification while building the final string literal in $\buff$.
It is defined via the following rules:
\begin{mathpar}
\inferrule[no\_match]{
  \maxmatches(\specstring, s) = \emptylist
}{
  \scanstring(\buff, s) \scanarrow \LexicalErrorConfig
}
\end{mathpar}

\begin{mathpar}
\inferrule[char]{
  \maxmatches(R, s) = [(e_1,a_1),\ldots,(e_n,a_n)]\\
  \remaxmatch(s, e_1) = (s_1, s_2)\\
  a_1(s_1, s_2) = \Tstringchar(t)\\
  \scanstring(\buff \concat t, s_2) \scanarrow \tstwo \terminateas \LexicalErrorConfig
}{
  \scanstring(\buff, s) \scanarrow \tstwo
}
\end{mathpar}

\begin{mathpar}
\inferrule[end]{
  \maxmatches(R, s) = [(e_1,a_1),\ldots,(e_n,a_n)]\\
  \remaxmatch(s, e_1) = (s_1, s_2)\\
  a_1(s_1, s_2) = \Tstringend\\
  \aslscan(\spectoken, s_2) \scanarrow \tstwo \terminateas \LexicalErrorConfig
}{
  \scanstring(\buff, s) \scanarrow [\Tstringlit(\buff)] \concat \tstwo
}
\end{mathpar}

We also employ the following lexeme actions:
\begin{itemize}
\item
\hypertarget{def-stringchar}{}
The lexeme action
\[
\stringchar(s_1, s_2) \triangleq \Tstringchar(s_1)
\]
returns $s_1$, which is always a single character, as a $\Tstringchar$
token, which is added to the characters that make up the final string
literal.

\item
\hypertarget{def-stringescape}{}
The lexeme action
\[
\stringescape(s_1, s_2) \triangleq \begin{cases}
  \Tstringchar(10) & s_1 = \texttt{\textbackslash\;\;n}\\
  \Tstringchar(9) & s_1 = \texttt{\textbackslash\;\;t}\\
  \Tstringchar(34) & s_1 = \texttt{\textbackslash\;\;"}\\
  \Tstringchar(92) & s_1 = \texttt{\textbackslash\;\;\textbackslash}\\
\end{cases}
\]
returns the ASCII character for the corresponding escape string, in decimal encoding,
as a $\Tstringchar$ token, which is added to the characters that make up the final string
literal.

\item
\hypertarget{def-stringfinish}{}
The lexeme action
\[
\stringfinish(s_1, s_2) \triangleq \Tstringend
\]
signals that the string literal has ended, which makes $\scanstring$
switch to scanning via $\aslscan$ and $\spectoken$.
\end{itemize}

\hypertarget{def-specstring}{}
The lexical specification for string literals --- $\specstring$ --- is given by the following table:

\begin{center}
\begin{tabular}{ll}
\textbf{Lexical Regular Expression} & \textbf{Lexeme Action}\\
\hline
\anycharacter{\backslash\ } \anycharacter{{\color{white}\backslash}\texttt{n }}  &  $\stringescape$\\
\anycharacter{\backslash\ } \anycharacter{{\color{white}\backslash}\texttt{t }}  &  $\stringescape$\\
\anycharacter{\backslash\ } \anycharacter{{\color{white}\backslash}\texttt{" }}  &  $\stringescape$\\
\anycharacter{\backslash\ } \anycharacter{\ \backslash\ }  & $\stringescape$ \\
\anycharacter{{\color{white}\backslash}\texttt{" }}   &  $\stringfinish$\\
$\REchar$                                             &  $\stringchar$\\
\hline
\end{tabular}
\end{center}

\subsection{Scanning Multi-line Comments\label{sec:scanningmultilinecomments}}
The lexeme action
\hypertarget{def-discardcommentchar}{}
\[
\discardcommentchar(s_1, s_2) \triangleq \aslscan(\spectoken, s_2)
\]
discards the string $s_1$ (which is always a single character) and continues scanning $s_2$ with $\speccomment$.
This is the same as $\discard$, except that $s_2$ is scanned with $\speccomment$ instead of $\spectoken$.

\hypertarget{def-speccomment}{}
The lexical specification for multi-line comments --- $\speccomment$ --- is given by the table below.
%
Notice that here, $\actiondiscard$ below is used to discard the closing of the multi-line comment and to switch
to scanning with $\spectoken$.

\begin{center}
\begin{tabular}{ll}
\textbf{Lexical Regular Expression} & \textbf{Lexeme Action}\\
\hline
\texttt{"*/"} & $\actiondiscard$ \\
$\REchar$     & $\discardcommentchar$ \\
\hline
\end{tabular}
\end{center}
