\section{BET1}

The following changes have been made.

\subsection{ASL-857: \texttt{for}-loop limits}

Loop limits for \texttt{for}-loops previously triggered one iteration too early.
For example, the following previously produced a dynamic error, but now evaluates successfully.

\begin{lstlisting}
for i = 1 to 1 looplimit 1 do
  pass;
end;
\end{lstlisting}

\subsection{ASL-860: purity of print statements}

Printing statements (\secref{PrintStatements}) are now considered neither \pureterm{} nor \readonlyterm{}.

\subsection{ASL-861: recurse limits for procedures}

Recurse limits are now permitted on procedures.
Their syntax mirrors recurse limits on functions:
\begin{lstlisting}
func foo() recurselimit 10
begin
  ...
end;
\end{lstlisting}

\subsection{ASL-864: binary operator precedence}

Ambiguous expressions such as \texttt{a - b - c} are now forbidden.
In particular, expressions of the form \texttt{x op y op z} are forbidden if \texttt{op} is not \emph{associative}.
Concatenation (\Tcoloncolon) is made associative for bit vectors by factoring string concatenation into a separate operator, \Tplusplus.
The associative operators are then:
\Tplus, \Tmul, \Tband, \Tbor, \Tand, \Tor, \Txor, \Tbeq, \Tplusplus, \Tcoloncolon.
See \RequirementRef{OperatorPrecedence} for more details.
