\chapter{Types\label{chap:Types}}

\identi{BYVL}
ASL allows specifying types for storage elements such as variables, constants, function arguments,
and they are used to define the set of their allowed values.
%
The type system infers types for expressions, which are represented by \typedast{} nodes derived
from the non-terminal $\ty$.

\ExampleDef{Types}
\listingref{Types} shows some examples of types.
\ASLListing{Examples of Types}{Types}{\definitiontests/Types.asl}

\ChapterOutline
\begin{itemize}
  \item \FormalRelationsRef{Types} defines the formal relations for types;
  \item \AbstractSyntaxRef{Types} defines how the abstract syntax nodes for types
    are generated;
  \item \secref{IntegerTypes} defines the \integertypeterm{};
  \item \secref{RealType} defines the \realtypeterm{};
  \item \secref{StringType} defines the \stringtypeterm{};
  \item \secref{BooleanType} defines the \booleantypeterm{};
  \item \secref{BitvectorTypes} defines the \bitvectortypeterm{};
  \item \secref{TupleTypes} defines \Tupletypesterm{};
  \item \secref{ParenthesizedTypes} defines parenthesized types;
  \item \secref{EnumerationTypes} defines \Enumerationtypesterm{};
  \item \secref{ArrayTypes} defines array types;
  \item \secref{RecordTypes} defines record types;
  \item \secref{ExceptionTypes} defines exception types;
  \item \secref{CollectionTypes} defines collection types;
  \item \secref{NamedTypes} defines named types;
  \item \secref{DeclaredTypes} defines \emph{declared types} and restrictions over them;
  \item \secref{DomainOfValuesForTypes} defines the semantics of types by associating a value to each type;
  \item \secref{BasicTypeAttributes} assigns basic properties to types, which are useful
        in classifying them;
  \item \secref{RelationsOnTypes} defines relations on types that are needed to typecheck
        expressions and statements; and
  \item \secref{BaseValues} defines how to generate a \emph{base value} --- an expression
        to initialize a storage element of a given type for which no initializing
        expression is given.
\end{itemize}

%%%%%%%%%%%%%%%%%%%%%%%%%%%%%%%%%%%%%%%%%%%%%%%%%%%%%%%%%%%%%%%%%%%%%%%%%%%%%%%%
\FormalRelationsDef{Types}
\paragraph{Syntax:} \Anonymoustypes{} are grammatically derived from $\Nty$
and types that must be declared and named are grammatically derived from $\Ntydecl$;

\paragraph{Abstract Syntax:} Types are derived in the abstract syntax from $\ty$,
  and generated by $\buildty$, $\buildtydecl$, and $\buildtyorcollection$.

\paragraph{Typing:}
\hypertarget{def-annotatetype}{}
The function
\[
  \annotatetype(\overname{\Bool}{\vdecl} \aslsep \overname{\staticenvs}{\tenv} \aslsep \overname{\ty}{\tty})
  \aslto (\overname{\ty}{\newty} \times \overname{\TSideEffectSet}{\vses}) \cup \overname{\typeerror}{\TypeErrorConfig}
\]
typechecks a type $\tty$ in a \staticenvironmentterm{} $\tenv$,
resulting in a \typedast\ $\newty$ and a \sideeffectsetterm\ $\vses$.
The flag $\decl$ indicates whether $\tty$ is a type currently being declared or not,
and makes a difference only when $\tty$ is an \enumerationtypeterm{} or a \structuredtype.
\ProseOtherwiseTypeError

\paragraph{Semantics:} Types are not evaluated dynamically.
  However, the dynamic semantics of types is given by their \emph{domain of values},
  which is defined in \secref{DomainOfValuesForTypes}.
  In addition, the dynamic semantics of an \atcexpressionterm{}
  checks whether a \nativevalueterm{} belongs to the domain of values of a given type
  (see \SemanticsRuleRef{ATC} and \SemanticsRuleRef{IsValOfType}).

%%%%%%%%%%%%%%%%%%%%%%%%%%%%%%%%%%%%%%%%%%%%%%%%%%%%%%%%%%%%%%%%%%%%%%%%%%%%%%%%
\AbstractSyntaxDef{Types}
\hypertarget{build-ty}{}
The function
\[
  \buildty(\overname{\parsenode{\Nty}}{\vparsednode}) \;\aslto\; \overname{\ty}{\vastnode}
  \cup \overname{\TBuildError}{\BuildErrorConfig}
\]
transforms an anonymous type parse node $\vparsednode$ into the corresponding AST node $\vastnode$.
\ProseOtherwiseBuildError

We define $\buildty$ per type in the following sections.

\hypertarget{build-tydecl}{}
The function
\[
  \buildtydecl(\overname{\parsenode{\Ntydecl}}{\vparsednode}) \;\aslto\; \overname{\ty}{\vastnode}
  \cup \overname{\TBuildError}{\BuildErrorConfig}
\]
transforms a \namedtype\ parse node $\vparsednode$ into an AST node $\vastnode$.
\ProseOtherwiseBuildError

We define $\buildtydecl$ per the corresponding type in the following sections.

\hypertarget{build-tyorcollection}{}
The function
\[
  \buildtyorcollection(\overname{\parsenode{\Ntyorcollection}}{\vparsednode}) \;\aslto\; \overname{\ty}{\vastnode}
  \cup \overname{\TBuildError}{\BuildErrorConfig}
\]
transforms a type annotation parse node $\vparsednode$ into a type AST node $\vastnode$.
\ProseOtherwiseBuildError

We define $\buildtyorcollection$ per the corresponding type in the following sections.

\hypertarget{build-as-ty}{}
The function
\[
  \buildasty(\overname{\parsenode{\Nasty}}{\vparsednode}) \;\aslto\; \overname{\ty}{\vastnode}
  \cup \overname{\TBuildError}{\BuildErrorConfig}
\]
transforms a type annotation parse node $\vparsednode$ into a type AST node $\vastnode$.
\ProseOtherwiseBuildError

Formally:
\begin{mathpar}
\inferrule{
  \buildty(\vt) \astarrow \astversion{\vt}
} {
  \buildasty(\Tcolon, \namednode{\vt}{\Nty}) \astarrow \astversion{\vt}
}
\end{mathpar}

\hypertarget{integertypeterm}{}
\section{Integer Types\label{sec:IntegerTypes}}
The \emph{\integertypesterm{}} represent mathematical integer value.

There are four kinds of integer types, and we
use the term \integertypeterm{} to refer to them collectively:
\emph{\unconstrainedintegertypes},
\emph{\wellconstrainedintegertypes},
\emph{\pendingconstrainedintegertypes},
and \emph{\parameterizedintegertypes}.

\subsection{Unconstrained Integer Types}
\hypertarget{def-unconstrainedintegertype}{}
The type \verb|integer| represents all integer values.
\identi{HJBH}%
There is no bound on the minimum and maximum integer value that can be represented.

\ExampleDef{Unconstrained Integer Types}
\listingref{typing-unconstrained} shows examples of unconstrained integer types.
\ASLListing{Well-typed unconstrained integer types}{typing-unconstrained}{\typingtests/TypingRule.TIntUnConstrained.asl}

\subsection{Well-constrained Integer Types}
\identr{GWCP}%
The type \texttt{integer\{$c_1,\ldots,c_n$\}} represents the
union of sets of integers represented by the \emph{integer constraints} $c_1,\ldots,c_n$.
\hypertarget{def-exactconstraintterm}{}
\hypertarget{def-rangeconstraintterm}{}
A constraint can either be an \emph{\exactconstraintterm}, consisting of a single expression like \texttt{4},
or a \emph{\rangeconstraintterm}, consisting of a pair of expressions like \texttt{1..10}.

\ExampleDef{Well-constrained Integer Types}
\hypertarget{def-wellconstrainedintegertype}{}
The well-typed specification in \listingref{constrainedintegers1} shows examples of \wellconstrainedintegertypes.
\ASLListing{Well-typed well-constrained integer types}{constrainedintegers1}{\definitiontests/ConstrainedIntegers1.asl}

The well-typed specification in \listingref{constrainedintegers2} shows examples of \wellconstrainedintegertypes{}
and \unconstrainedintegertypes{} for \verb|config|s.
\ASLListing{Well-constrained integer types and \texttt{config}s}{constrainedintegers2}{\definitiontests/ConstrainedIntegers2.asl}

The specifications in \listingref{constrainedintegers-bad}
and \listingref{constrainedintegers-bad2}
show examples of ill-typed
assignment between two \wellconstrainedintegertypes{}.
\ASLListing{Ill-typed well-constrained integer types}{constrainedintegers-bad}{\definitiontests/ConstrainedIntegers.bad.asl}
\ASLListing{Ill-typed well-constrained integer types}{constrainedintegers-bad2}{\definitiontests/ConstrainedIntegers.bad2.asl}

\subsection{Pending-constrained Integer Types}
\hypertarget{def-pendingconstrainedintegertype}{}
The type \verb|integer{-}| represents a well-constrained integer type whose
constraints have yet to be determined.
These constraints are inferred by the type system based on the expression used to initialize
the storage element (see \TypingRuleRef{InheritIntegerConstraints}).

\RequirementDef{PendingConstrainedLHS}
Pending-constrained integer types may only appear on the left-hand-side
of local and global storage element declarations.
They may not appear in \texttt{config} declarations.
%
\listingref{config-pending-constrained} shows an ill-typed specification.

\ExampleDef{Well-typed pending-constrained types}
\listingref{typing-pendingconstrained} shows examples of well-typed pending-constrained
integer types.
\ASLListing{Well-typed pending-constrained integer types}{typing-pendingconstrained}{\typingtests/TypingRule.InheritIntegerConstraints.asl}

\subsection{Parameterized Integer Types}
\hypertarget{def-parameterizedintegertype}{}
Subprogram parameters are implicitly \emph{parameterized integer types},
which represent a singleton set for the integer passed to the parameter
at the call site.

\ExampleDef{Parameterized Integer Types}
\listingref{typing-parameterized} shows examples of well-typed parameterized
integer types.
Notice that the type of the parameter \texttt{M} of the function \texttt{bar}
is a parameterized integer type, \underline{not} an unconstrained integer type.
\ASLListing{Well-typed parameterized integer types}{typing-parameterized}{\typingtests/TypingRule.TIntParameterized.asl}

\subsection{Syntax\label{sec:IntegerTypesSyntax}}
\begin{flalign*}
\Nty \derives\ & \Tinteger \parsesep \Nconstraintkindopt &\\
\Nconstraintkindopt \derives \ & \Nconstraintkind \;|\; \emptysentence &\\
\Nconstraintkind \derives \ & \Tlbrace \parsesep \ClistOne{\Nintconstraint} \parsesep \Trbrace &\\
|\ & \Tlbrace \parsesep \Trbrace &\\
\Nintconstraint \derives \ & \Nexpr &\\
|\ & \Nexpr \parsesep \Tslicing \parsesep \Nexpr &
\end{flalign*}

\subsection{Abstract Syntax\label{sec:IntegerTypesAST}}
\RenderTypes[remove_hypertargets]{ty_int_constraint_and_kind}
\BackupOriginalAST{
  \begin{flalign*}
\ty \derives\ & \TInt(\constraintkind)\\
\constraintkind \derives\ & \unconstrained
& \\
|\ & \wellconstrained(\intconstraint^{+})
& \\
|\ & \pendingconstrained{}
& \\
|\ & \parameterized(\overtext{\identifier}{parameter}) &\\
\intconstraint \derives\ & \ConstraintExact(\expr)
& \\
|\ & \ConstraintRange(\overtext{\expr}{start}, \overtext{\expr}{end})&
\end{flalign*}
}

\ASTRuleDef{Ty.TInt}
\begin{mathpar}
\inferrule{}{
  \buildty(\Nty(\Tinteger, \punnode{\Nconstraintkindopt})) \astarrow
  \overname{\TInt(\astof{\vconstraintkindopt})}{\vastnode}
}
\end{mathpar}

\ASTRuleDef{IntConstraintsOpt}
\hypertarget{build-constraintkindopt}{}
The function
\[
  \buildconstraintkindopt(\overname{\parsenode{\Nconstraintkindopt}}{\vparsednode}) \;\aslto\; \overname{\constraintkind}{\vastnode}
\]
transforms a parse node $\vparsednode$ into an AST node $\vastnode$.

\begin{mathpar}
\inferrule[constrained]{}{
  {
    \begin{array}{r}
  \buildconstraintkindopt(\Nconstraintkindopt(\punnode{\Nconstraintkind})) \astarrow \\
  \overname{\astof{\vconstraintkind}}{\vastnode}
    \end{array}
  }
}
\end{mathpar}

\begin{mathpar}
\inferrule[unconstrained]{}{
  \buildconstraintkindopt(\Nconstraintkindopt(\emptysentence)) \astarrow
  \overname{\Unconstrained}{\vastnode}
}
\end{mathpar}

\subsection{ASTRule.IntConstraints\label{sec:ASTRule.IntConstraints}}
\hypertarget{build-constraintkind}{}
The function
\[
  \buildconstraintkind(\overname{\parsenode{\Nconstraintkind}}{\vparsednode}) \;\aslto\; \overname{\constraintkind}{\vastnode}
\]
transforms a parse node $\vparsednode$ into an AST node $\vastnode$.

\begin{mathpar}
\inferrule[well\_constrained]{
  \buildclist[\buildintconstraint](\cs) \astarrow \vcsasts
}{
  {
    \begin{array}{r}
  \buildconstraintkind(\Nconstraintkind(\Tlbrace, \namednode{\cs}{\ClistOne{\Nintconstraint}}, \Trbrace)) \astarrow\\
  \overname{\WellConstrained(\vcsasts)}{\vastnode}
    \end{array}
  }
}
\end{mathpar}

\begin{mathpar}
\inferrule[pending\_constrained]{}{
  \buildconstraintkind(\Nconstraintkind(\Tlbrace, \Trbrace)) \astarrow
  \overname{\PendingConstrained}{\vastnode}
}
\end{mathpar}

\ASTRuleDef{IntConstraint}
\hypertarget{build-intconstraint}{}
The function
\[
  \buildintconstraint(\overname{\parsenode{\Nintconstraint}}{\vparsednode}) \;\aslto\; \overname{\intconstraint}{\vastnode}
\]
transforms a parse node $\vparsednode$ into an AST node $\vastnode$.

\begin{mathpar}
\inferrule[exact]{}{
  \buildintconstraint(\Nintconstraint(\punnode{\Nexpr})) \astarrow
  \overname{\ConstraintExact(\astof{\vexpr})}{\vastnode}
}
\end{mathpar}

\begin{mathpar}
\inferrule[range]{
  \buildexpr(\vfromexpr) \astarrow \astversion{\vfromexpr}\\
  \buildexpr(\vtoexpr) \astarrow \astversion{\vtoexpr}\\
}{
  {
    \begin{array}{r}
  \buildintconstraint(\Nintconstraint(\namednode{\vfromexpr}{\Nexpr}, \Tslicing, \namednode{\vtoexpr}{\Nexpr})) \astarrow\\
  \overname{\ConstraintRange(\astversion{\vfromexpr}, \astversion{\vtoexpr})}{\vastnode}
    \end{array}
  }
}
\end{mathpar}

\subsection{Typing Integer Types\label{sec:TypingIntegerTypes}}
\hypertarget{def-isunconstrainedinteger}{}
\hypertarget{def-isparameterizedinteger}{}
\hypertarget{def-iswellconstrainedinteger}{}
We use the following helper predicates to classify integer types:
\[
  \begin{array}{rcl}
  \isunconstrainedinteger(\overname{\ty}{\vt}) &\aslto& \Bool\\
  \isparameterizedinteger(\overname{\ty}{\vt}) &\aslto& \Bool\\
  \iswellconstrainedinteger(\overname{\ty}{\vt}) &\aslto& \Bool
  \end{array}
\]
Those are defined as follows:
\[
  \begin{array}{rcl}
  \isunconstrainedinteger(\vt) &\triangleq& \vt = \TInt(c) \land \astlabel(c)=\Unconstrained\\
  \isparameterizedinteger(\vt) &\triangleq& \vt = \TInt(c) \land \astlabel(c)=\Parameterized\\
  \iswellconstrainedinteger(\vt) &\triangleq& \vt = \TInt(c) \land \astlabel(c)=\WellConstrained\\
\end{array}
\]
\identd{ZTPP} \identr{WJYH} \identr{HJPN} \identr{CZTX} \identr{TPHR}

\hypertarget{def-unconstrainedinteger}{}
We use the shorthand notation $\unconstrainedinteger \triangleq \TInt(\Unconstrained)$
for unconstrained integers.

\RequirementDef{ConstraintSymbolicallyConstrained}
The expressions appearing in integer constraints must be both
\symbolicallyevaluable{} and \constrainedinteger{} types.
%
In \listingref{annotate-constraint-unconstrained}, the constraint
\verb|x..x+1| is ill-typed, since \texttt{x} is not \symbolicallyevaluable{}.
\ASLListing{Ill-typed constraint}{annotate-constraint-unconstrained}{\typingtests/TypingRule.AnnotateConstraint.bad.asl}

\TypingRuleDef{TInt}
\ExampleDef{Ill-typed pending-constrained integer type}
\listingref{config-pending-constrained}
and \listingref{rhs-pending-constrained}
correspond to \CaseName{pending\_constrained}.
\ASLListing{Ill-typed pending-constrained integer type}{config-pending-constrained}
{\typingtests/TypingRule.TInt.config_pending_constrained.bad.asl}

\ASLListing{Ill-typed pending-constrained integer type}{rhs-pending-constrained}
{\typingtests/TypingRule.TInt.rhs_pending_constrained.bad.asl}

\ProseParagraph
\OneApplies
\begin{itemize}
  \item \AllApplyCase{pending\_constrained}
    \begin{itemize}
      \item $\tty$ is a \pendingconstrainedintegertype;
      \item the result is a \typingerrorterm{} (\UnexpectedType).
    \end{itemize}
  \item \AllApplyCase{well\_constrained}
    \begin{itemize}
      \item $\tty$ is the well-constrained integer type constrained by
        constraints $\vc_i$, for $u=1..k$;
      \item annotating each constraint $\vc_i$, for $i=1..k$,
      yields $(\newc_i, \vxs_i)$\ProseOrTypeError;
      \item $\newconstraints$ is the list of annotated constraints $\newc_i$,
      for $i=1..k$;
      \item $\newty$ is the well-constrained integer type constrained
        by $\newconstraints$ with $\PrecisionFull$;
      \item define $\vses$ as the union of all $\vxs_i$, for $i=1..k$.
    \end{itemize}

    \item \AllApplyCase{parameterized}
    \begin{itemize}
      \item $\tty$ is a \parameterizedintegertype\ for $\name$;
      \item define $\vses$ as the \sideeffectsetterm{} containing a $\Pure$ \LocalEffectTerm{} and an \ImmutabilityTerm{} with immutability $\True$.
      \item $\newty$ is the unconstrained integer type.
    \end{itemize}

    \item \AllApplyCase{unconstrained}
    \begin{itemize}
      \item $\tty$ is an \unconstrainedintegertype;
      \item $\newty$ is the unconstrained integer type;
      \item define $\vses$ as the empty set.
    \end{itemize}
  \end{itemize}

\FormallyParagraph
\begin{mathpar}
\inferrule[pending\_constrained]{}{
  {
    \begin{array}{r}
  \annotatetype(\overname{\Ignore}{\vdecl}, \tenv, \overname{\TInt(\PendingConstrained)}{\tty}) \typearrow
  \TypeErrorVal{\UnexpectedType}
    \end{array}
  }
}
\end{mathpar}

\begin{mathpar}
\inferrule[well\_constrained]{
  \constraints \eqname \vc_{1..k}\\
  i=1..k: \annotateconstraint(\vc_i) \typearrow (\newc_i, \vxs_i) \OrTypeError\\\\
  \newconstraints \eqdef \newc_{1..k}\\
  \vses \eqdef \bigcup_{i=1..k} \vxs_i
}{
  {
    \begin{array}{r}
  \annotatetype(\overname{\Ignore}{\vdecl}, \tenv, \overname{\TInt(\WellConstrained(\constraints))}{\tty}) \typearrow \\
  (\overname{\TInt(\WellConstrained(\newconstraints, \PrecisionFull))}{\newty}, \vses)
    \end{array}
  }
}
\end{mathpar}

\begin{mathpar}
\inferrule[parameterized]{
  \tty \eqname \TInt(\Parameterized(\name))\\
  \vses \eqdef \{ \LocalEffect(\Pure), \Immutability(\True) \}
}{
  \annotatetype(\overname{\Ignore}{\vdecl}, \tenv, \tty) \typearrow (\overname{\tty}{\newty}, \vses)
}
\end{mathpar}

\begin{mathpar}
\inferrule[unconstrained]{
  \tty \eqname \unconstrainedinteger
}{
  \annotatetype(\overname{\Ignore}{\vdecl}, \tenv, \tty) \typearrow (\overname{\tty}{\newty}, \overname{\emptyset}{\vses})
}
\end{mathpar}
\CodeSubsection{\TIntBegin}{\TIntEnd}{../Typing.ml}

\TypingRuleDef{AnnotateConstraint}
\hypertarget{def-annotateconstraint}{}
The function
\[
\begin{array}{r}
\annotateconstraint(\overname{\staticenvs}{\tenv} \aslsep \overname{\intconstraint}{\vc})
\aslto (\overname{\intconstraint}{\newc} \times \overname{\TSideEffectSet}{\vses})\ \cup \\
\overname{\typeerror}{\TypeErrorConfig}
\end{array}
\]
annotates an integer constraint $\vc$ in the \staticenvironmentterm{} $\tenv$ yielding the annotated
integer constraint $\newc$ and \sideeffectsetterm\ $\vses$.
\ProseOtherwiseTypeError

\listingref{annotate-constraint} shows examples of \wellconstrainedintegertypes{}
and the resulting annotated constraints in comments.
The annotated constraints inline the constant \texttt{N} and the right-hand-side
expressions of \texttt{let} storage elements.
\ASLListing{Annotated constraints}{annotate-constraint}{\typingtests/TypingRule.AnnotateConstraint.asl}

\ProseParagraph
\OneApplies
\begin{itemize}
  \item \AllApplyCase{exact}
  \begin{itemize}
    \item $\vc$ is the exact integer constraint for the expression $\ve$, that is, \\ $\ConstraintExact(\ve)$;
    \item applying $\annotatesymbolicconstrainedinteger$ to $\ve$ in $\tenv$ yields \\
          $(\vep, \vses)$\ProseOrTypeError;
    \item define $\newc$ as the exact integer constraint for $\vep$, that is, $\ConstraintExact(\vep)$.
  \end{itemize}

  \item \AllApplyCase{range}
  \begin{itemize}
    \item $\vc$ is the range integer constraint for expressions $\veone$ and $\vetwo$, that is, \\ $\ConstraintRange(\veone, \vetwo)$;
    \item applying $\annotatesymbolicconstrainedinteger$ to $\veone$ in $\tenv$ yields\\ $(\veonep, \vsesone)$\ProseOrTypeError;
    \item applying $\annotatesymbolicconstrainedinteger$ to $\vetwo$ in $\tenv$ yields\\ $(\vetwop, \vsestwo)$\ProseOrTypeError;
    \item define $\newc$ as the range integer constraint for expressions $\veonep$ and $\vetwop$, that is, $\ConstraintRange(\veonep, \vetwop)$;
    \item define $\vses$ as the union of $\vsesone$ and $\vsestwo$.
  \end{itemize}
\end{itemize}

\FormallyParagraph
\begin{mathpar}
\inferrule[exact]{
  \annotatesymbolicconstrainedinteger(\tenv, \ve) \typearrow (\vep, \vses) \OrTypeError
}{
  \annotateconstraint(\tenv, \overname{\ConstraintExact(\ve)}{\vc}) \typearrow (\overname{\ConstraintExact(\vep)}{\newc}, \vses)
}
\end{mathpar}

\begin{mathpar}
\inferrule[range]{
  \annotatesymbolicconstrainedinteger(\tenv, \veone) \typearrow (\veonep, \vsesone) \OrTypeError\\\\
  \annotatesymbolicconstrainedinteger(\tenv, \vetwo) \typearrow (\vetwop, \vsestwo) \OrTypeError\\\\
  \vses \eqdef \vsesone \cup \vsestwo
}{
  {
  \begin{array}{r}
    \annotateconstraint(\tenv, \overname{\ConstraintRange(\veone, \vetwo)}{\vc}) \typearrow \\
    (\overname{\ConstraintRange(\veonep, \vetwop)}{\newc}, \vses)
  \end{array}
  }
}
\end{mathpar}

\section{The Real Type\label{sec:RealType}}
\hypertarget{realtypeterm}{}
The \emph{\realtypeterm{}} represents mathematical rational number values.
There is no bound on the minimum and maximum rational value that can be represented,
and there is no bound on their precision.
%
There is no mechanism in the language to generate an irrational value of \realtypeterm.

Conversions from an \integertypeterm{} value to a \realtypeterm{} value are performed
using the \stdlibfunc{Real}.
%
Conversions from a \realtypeterm{} value an \integertypeterm{} value to are performed
using the \stdlibfunc{RoundDown}, \stdlibfunc{RoundUp}. and \stdlibfunc{RoundTowardsZero}.

\ExampleDef{Well-typed Real Types}
In \listingref{typing-treal}, all the uses of the \realtypeterm{} are well-typed.
\ASLListing{Well-typed real types}{typing-treal}{\typingtests/TypingRule.TReal.asl}

\subsection{Syntax}
\begin{flalign*}
\Nty \derives\ & \Treal &
\end{flalign*}

\subsection{Abstract Syntax}
\RenderTypes[remove_hypertargets]{ty_real}
\BackupOriginalAST{
\begin{flalign*}
\ty \derives\ & \TReal &
\end{flalign*}
}

\ASTRuleDef{TReal}
\begin{mathpar}
\inferrule{}{
  \buildty(\Nty(\Treal)) \astarrow
  \overname{\TReal}{\vastnode}
}
\end{mathpar}

\subsection{Typing the Real Type\label{sec:TypingRealType}}
\TypingRuleDef{TReal}
See \ExampleRef{Well-typed Real Types} for examples of well-typed \realtypeterm{}.

\ProseParagraph
\AllApply
\begin{itemize}
  \item $\tty$ is the \realtypeterm{}, $\TReal$.
  \item $\newty$ is the \realtypeterm{}, $\TReal$;
  \item define $\vses$ as the empty set.
\end{itemize}

\FormallyParagraph
\begin{mathpar}
\inferrule{}
{
  \annotatetype(\overname{\Ignore}{\vdecl}, \tenv, \overname{\TReal}{\tty}) \typearrow (\overname{\TReal}{\newty}, \overname{\emptyset}{\vses})
}
\end{mathpar}
\CodeSubsection{\TRealBegin}{\TRealEnd}{../Typing.ml}

\section{The String Type\label{sec:StringType}}
\hypertarget{stringtypeterm}{}
\hypertarget{stringtypesterm}{}
The \emph{\stringtypeterm{}} represents strings of characters.

Strings play relatively little role in specifications and the only operations
on strings are equality and inequality tests.
Strings are useful in \printstatementsterm{} for debugging and diagnostic purposes
on runtimes that support printing.

\ExampleDef{Well-typed String Types}
In \listingref{typing-tstring}, all the uses of the \stringtypeterm{} are well-typed.
\ASLListing{Well-typed string types}{typing-tstring}{\typingtests/TypingRule.TString.asl}

\subsection{Syntax}
\begin{flalign*}
\Nty \derives\ & \Tstring &
\end{flalign*}

\subsection{Abstract Syntax}
\RenderTypes[remove_hypertargets]{ty_string}
\BackupOriginalAST{
\begin{flalign*}
\ty \derives\ & \TString&
\end{flalign*}
}

\ASTRuleDef{Ty.String}
\begin{mathpar}
\inferrule{}{
  \buildty(\Nty(\Tstring)) \astarrow
  \overname{\TString}{\vastnode}
}
\end{mathpar}

\subsection{Typing the String Type\label{sec:TypingStringType}}
\TypingRuleDef{TString}
See \ExampleRef{Well-typed String Types} for examples of well-typed \stringtypesterm.

\ProseParagraph
\AllApply
\begin{itemize}
  \item $\tty$ is the \stringtypeterm{}, $\TString$.
  \item $\newty$ is the \stringtypeterm{}, $\TString$.
  \item \Proseeqdef{$\vses$}{the empty set}.
\end{itemize}

\FormallyParagraph
\begin{mathpar}
\inferrule{}
{
  \annotatetype(\overname{\Ignore}{\vdecl}, \tenv, \overname{\TString}{\tty}) \typearrow (\overname{\TString}{\newty}, \overname{\emptyset}{\vses})
}
\end{mathpar}
\CodeSubsection{\TStringBegin}{\TStringEnd}{../Typing.ml}

\section{The Boolean Type\label{sec:BooleanType}}
\hypertarget{booleantypeterm}{}
The \emph{\booleantypeterm{}} represents Booleans.

\ExampleDef{Well-typed Boolean Types}
In \listingref{typing-tbool}, all the uses of the \booleantypeterm{} are well-typed.
\ASLListing{Well-typed Boolean types}{typing-tbool}{\typingtests/TypingRule.TBool.asl}

\subsection{Syntax}
\begin{flalign*}
\Nty \derives\ & \Tboolean &
\end{flalign*}

\subsection{Abstract Syntax}
\RenderTypes[remove_hypertargets]{ty_bool}
\BackupOriginalAST{
\begin{flalign*}
\ty \derives\ & \TBool &
\end{flalign*}
}

\ASTRuleDef{Ty.BoolType}
\begin{mathpar}
\inferrule{}{
  \buildty(\Nty(\Tboolean)) \astarrow
  \overname{\TBool}{\vastnode}
}
\end{mathpar}

\subsection{Typing the Boolean Type\label{sec:TypingBooleanType}}
\TypingRuleDef{TBool}
See \ExampleRef{Well-typed Boolean Types} for examples of well-typed \booleantypesterm.

\ProseParagraph
\AllApply
\begin{itemize}
  \item $\tty$ is the boolean type, $\TBool$;
  \item $\newty$ is the boolean type, $\TBool$;
  \item define $\vses$ as the empty set.
\end{itemize}

\FormallyParagraph
\begin{mathpar}
\inferrule{}
{
  \annotatetype(\overname{\Ignore}{\vdecl}, \tenv, \overname{\TBool}{\tty}) \typearrow (\overname{\TBool}{\newty}, \overname{\emptyset}{\vses})
}
\end{mathpar}
\CodeSubsection{\TBoolBegin}{\TBoolEnd}{../Typing.ml}

\section{Bitvector Types\label{sec:BitvectorTypes}}
\hypertarget{bitvectortypeterm}{}
\emph{Bitvectors} represent sequences of $0$ and $1$ bits.
%
The \texttt{bits(N)} type represents a bitvector of width \texttt{N},
where \texttt{N} may specify a fixed width or a constrained width.

\identr{RXYN}%
Bitvectors can be converted to unsigned integers and signed integer
via the standard library functions
\verb|UInt| and \verb|SInt|, respectively.
Converting integers into bitvectors can be done via slicing
(see \ExampleRef{Evaluation of Slicing Expressions}).

\identi{KGMC}%
The syntax for \bitvectortypesterm{} has an optional $\Nbitfields$,
which allows specifying \emph{\bitfieldsterm} ---
\bitslicesterm{} of bitvectors --- to be treated as named
fields that can be read or written.
\chapref{Bitfields} defines \bitfieldsterm{}
and \chapref{BitvectorSlicing} defines \bitslicesterm{}.

\SyntacticSugarDef{Bit}
The \Tbit{} type is \syntacticsugar{} for \texttt{bits(1)},
as exemplified by \listingref{Bit}.
It is \desugared{} by \ASTRuleCaseRef{Ty.TBits}{BIT}.

\ASLListing{Syntactic sugar for single bit bitvector types}{Bit}{\definitiontests/Bit.asl}

\RequirementDef{BitvectorsMSB}
In a bitvector literal, the most significant bit (\ProseMSB) is the leftmost value
and the least significant bit (\ProseLSB) is the rightmost value.
% INLINED_EXAMPLE
For example, in \verb|'10'| the \ProseMSB{} is \verb|1| and the \ProseLSB{} is \verb|0|.

\RequirementDef{BitvectorWidthImmutable}
The width of a bitvector cannot be modified.

In \listingref{BitvectorSlices}, slicing expressions such as \verb|bv[5:0]|
and bitvector concatenation expressions such as \verb|bv[5:5] :: bv[4:4]|
create new bitvector values without affecting the widths (or values)
of existing bitvector values.

\RequirementDef{BitvectorWidthBounds}
There is no bound on the maximum bitvector width allowed, although an implementation may specify an upper
limit.
%The minimum bound is zero.
It is recognized that zero-width bitvectors might not be supported in systems
to which ASL might be translated (such as SMT solvers),
and an implementation might need to lower bitvector
expressions to a form where zero-width bitvectors do not exist.

In \listingref{BitvectorWidthBounds}, any number can be used instead of \verb|2^20| for
\verb|large_bitvector|, and \verb|zero_width_bitvector| is an example of a zero-width bitvector.

\ASLListing{Large and small bitvectors}{BitvectorWidthBounds}{\definitiontests/GuideRule.BitvectorWidthBounds.asl}

\RequirementDef{BitvectorWidthKind}
The width of a \bitvectortypeterm{} can be either \staticallyevaluable{}
or \emph{constrained}. That is, a \symbolicallyevaluable{} \constrainedinteger{}.

\ExampleDef{Symbolic and Constrained Bitwidth}
\listingref{symbolic_bitwidth} shows an example of a bitvector whose width
is symbolic --- \verb|2 * N| and a bitvector whose width is determined
by a constrained type \verb|integer{4, 8}|.
\ASLListing{Constrained bitwidth and symbolic bitwidth}{symbolic_bitwidth}{\definitiontests/symbolic_bitwidth.asl}

\listingref{constrained_bitwidth} shows more examples of constrained bitwidths.
\ASLListing{Constrained bitwidths}{constrained_bitwidth}{\definitiontests/constrained_bitwidth.asl}

\ExampleDef{Rotating a Bitvector}
\listingref{bits-rotate} shows a specification where the width of the bitvector type
\texttt{bv} is a literal (\verb|bits(5)|), and bitvector types where the width is
constrained (\verb|bits(N)|, \verb|bits(i)|, and \verb|bits(N-i)|),
and related operations,
followed by the output to the console.
\ASLListing{Rotating a bitvector}{bits-rotate}{\definitiontests/Bitvector_rotate.asl}
% CONSOLE_BEGIN aslref \definitiontests/Bitvector_rotate.asl
\begin{Verbatim}[fontsize=\footnotesize, frame=single]
bv=0x14, rotated twice=0x05
\end{Verbatim}
% CONSOLE_END

\subsection{Syntax}
\begin{flalign*}
\Nty \derives\ & \Tbit &\\
            |\ & \Tbits \parsesep \Tlpar \parsesep \Nexpr \parsesep \Trpar \parsesep \option{\Nbitfields} &\\
\Nbitfields \derives \ & \Tlbrace \parsesep \TClistZero{\Nbitfield} \parsesep \Trbrace &\\
\Nbitfield \derives \ & \Nslices \parsesep \Tidentifier &\\
                  |\ & \Nslices \parsesep \Tidentifier \parsesep \Nbitfields &\\
                  |\ & \Nslices \parsesep \Tidentifier \parsesep \Tcolon \parsesep \Nty &\\
\end{flalign*}

\subsection{Abstract Syntax}
\RenderTypes[remove_hypertargets]{ty_bits}
\BackupOriginalAST{
\begin{flalign*}
\ty \derives\ & \TBits(\overtext{\expr}{width}, \bitfield^{*}) &
\end{flalign*}
}

\ASTRuleDef{Ty.TBits}
\begin{mathpar}
\inferrule[bit]{}{
  \buildty(\Nty(\Tbit)) \astarrow
  \overname{\TBits(\ELiteral(\LInt(1)), \emptylist)}{\vastnode}
}
\end{mathpar}

\begin{mathpar}
\inferrule[bits]{
  \buildlist[\buildbitfield](\vbitfields) \astarrow \vbitfieldasts
}{
  {
    \begin{array}{r}
  \buildty(\Nty(\Tbits, \Tlpar, \punnode{\Nexpr}, \Trpar, \namednode{\vbitfields}{\maybeemptylist{\Nbitfields}})) \astarrow\\
  \overname{\TBits(\astof{\vexpr}, \vbitfieldasts)}{\vastnode}
    \end{array}
  }
}
\end{mathpar}

\subsection{Typing Bitvector Types}
\TypingRuleDef{TBits}
\ExampleDef{Well-typed Bitvector Types}
In \listingref{typing-tbits}, all the uses of bitvector types are well-typed.
\ASLListing{Well-typed Bitvector types}{typing-tbits}{\typingtests/TypingRule.TBits.asl}

The specification in \listingref{typing-tbits-bad} is ill-typed, since widths need
to be \constrainedintegers.
\ASLListing{An ill-typed Bitvector type}{typing-tbits-bad}{\typingtests/TypingRule.TBits.bad.asl}

\ExampleRef{A bitvector type with bitfields} shows a well-typed \bitvectortypeterm{} with bitfields.

\ProseParagraph
\AllApply
\begin{itemize}
  \item $\tty$ is the \bitvectortypeterm{} with width given by the expression
    $\ewidth$ and the bitfields given by $\bitfields$, that is, $\TBits(\ewidth, \bitfields)$;
  \item annotating the expression $\ewidth$ yields $(\twidth, \ewidthp, \seswidth)$\ProseOrTypeError;
  \item \Prosechecksymbolicallyevaluable{\seswidth};
  \item \Prosecheckconstrainedinteger{$\tenv$}{$\twidth$};
  \item \OneApplies
  \begin{itemize}
    \item \AllApplyCase{with\_bitfields}
    \begin{itemize}
      \item $\bitfields$ is not empty;
      \item checking that $\seswidth$ is \pure{} via $\sesispure$ yields $\True$\ProseTerminateAs{\SideEffectViolation};
      \item annotating the bitfields $\bitfields$ yields \\
            $(\bitfieldsp, \vsesbitfields)$\ProseOrTypeError;
      \item \Prosestaticeval{$\tenv$}{$\ewidthp$}{$\LInt(\vwidth)$};
      \item \Prosecheckcommonbitfieldsalign{$\tenv$}{$\bitfieldsp$}{$\vwidth$}\ProseOrTypeError;
      \item \Proseeqdef{$\newty$}{the \bitvectortypeterm{} with width given by the expression
            $\ewidthp$ and the bitfields given by $\bitfieldsp$, that is, \\
            $\TBits(\ewidthp, \bitfieldsp)$};
      \item \Proseeqdef{$\vses$}{the union of $\seswidth$ and $\vsesbitfields$}.
    \end{itemize}

    \item \AllApplyCase{no\_bitfields}
    \begin{itemize}
      \item $\bitfields$ is empty;
      \item \Proseeqdef{$\newty$}{the \bitvectortypeterm{} with width given by the expression
            $\ewidthp$ and an empty list of bitfields, that is,
            $\TBits(\ewidthp, \emptylist)$};
      \item \Proseeqdef{$\vses$}{$\seswidth$}.
    \end{itemize}
  \end{itemize}
\end{itemize}

\FormallyParagraph
\begin{mathpar}
\inferrule[with\_bitfields]{
  \annotateexpr(\tenv, \ewidth) \typearrow (\twidth, \ewidthp, \seswidth) \OrTypeError\\\\
  \checksymbolicallyevaluable(\seswidth) \typearrow \True \OrTypeError\\\\
  \checkconstrainedinteger(\tenv, \twidth) \typearrow \True \OrTypeError\\\\
  \commonprefixline\\\\
  \bitfields \neq \emptylist\\\\
  \checktrans{\sesispure(\seswidth)}{\SideEffectViolation} \typearrow \True \OrTypeError\\\\
  {
  \begin{array}{r}
    \annotatebitfields(\tenv, \ewidthp, \bitfields) \typearrow \\
    (\bitfieldsp, \vsesbitfields) \OrTypeError
  \end{array}
  }\\
  \staticeval(\tenv, \ewidthp) \typearrow \LInt(\vwidth) \OrTypeError\\\\
  \checkcommonbitfieldsalign(\tenv, \bitfieldsp, \vwidth) \typearrow \True \OrTypeError\\\\
  \vses \eqdef \seswidth \cup \vsesbitfields
}{
  {
    \begin{array}{r}
  \annotatetype(\overname{\Ignore}{\vdecl}, \tenv, \TBits(\ewidth, \bitfields)) \typearrow \\
  (\overname{\TBits(\ewidthp, \bitfieldsp)}{\newty}, \vses)
    \end{array}
  }
}
\end{mathpar}

\begin{mathpar}
\inferrule[no\_bitfields]{
  \annotateexpr(\tenv, \ewidth) \typearrow (\twidth, \ewidthp, \seswidth) \OrTypeError\\\\
  \checksymbolicallyevaluable(\seswidth) \typearrow \True \OrTypeError\\\\
  \checkconstrainedinteger(\tenv, \twidth) \typearrow \True \OrTypeError\\\\
  \commonprefixline\\\\
  \bitfields = \emptylist
}{
  {
    \begin{array}{r}
  \annotatetype(\overname{\Ignore}{\vdecl}, \tenv, \TBits(\ewidth, \bitfields)) \typearrow \\
  (\overname{\TBits(\ewidthp, \bitfieldsp)}{\newty}, \overname{\seswidth}{\seswidth})
    \end{array}
  }
}
\end{mathpar}
\CodeSubsection{\TBitsBegin}{\TBitsEnd}{../Typing.ml}

\section{Tuple Types\label{sec:TupleTypes}}
\hypertarget{tupletypeterm}{}

Types can be combined into \tupletypesterm{} whose values consist of tuples of values of those types.
For example, the expression \verb|(TRUE, Zeros{32})| has type \verb|(boolean, bits(32))|.

\ExampleDef{Well-typed Tuples}
In \listingref{typing-ttuple}, all the uses of \tupletypesterm{} are well-typed.
\ASLListing{Well-typed tuple types}{typing-ttuple}{\typingtests/TypingRule.TTuple.asl}

\RequirementDef{TupleLength}
A \tupletypeterm{} must contain at least two elements.

In \listingref{TupleLength}, both \verb|x| and \verb|y| have tuple types,
whereas \verb|w| and \verb|z| are of the \integertypeterm, since \verb|(5)|
is considered a parenthesized expression, not a tuple expression.
\ASLListing{Tuples and parenthesized expressions}{TupleLength}{\definitiontests/GuideRule.TupleLength.asl}

In \listingref{EmptyTuple}, \verb|()| produces a parsing error, as tuples cannot be empty.
\ASLListing{Invalid empty tuple}{EmptyTuple}{\definitiontests/EmptyTuple.asl}

\RequirementDef{TupleImmutability}
The value and type of tuple elements cannot be modified.

\listingref{TupleImmutability} demonstrates how variables of a \tupletypeterm{} may be assigned,
but the tuple values they store may not be modified.
\ASLListing{Immutability of tuple values}{TupleImmutability}{\definitiontests/GuideRule.TupleImmutability.asl}

\RequirementDef{TupleElementAccess}
The $k+1$ element of a tuple \verb|t| with $n>1$ elements
can be accessed via the \texttt{t.item$k$} notation,
as long as $0 \leq k < n$.

\listingref{TupleElementAccess} shows examples of accessing the elements
of the tuple stored in \verb|x|.
\ASLListing{Accessing tuple elements}{TupleElementAccess}{\definitiontests/GuideRule.TupleElementAccess.asl}

\listingref{TupleElementAccess-bad} shows an example of an illegal access to a non-existing
element index.
\ASLListing{Invalid access to a tuple element}{TupleElementAccess-bad}{\definitiontests/GuideRule.TupleElementAccess.bad.asl}

\subsection{Syntax}
\begin{flalign*}
\Nty \derives\ & \Plisttwo{\Nty} &
\end{flalign*}

\subsection{Abstract Syntax}
\RenderTypes[remove_hypertargets]{ty_tuple}
\BackupOriginalAST{
\begin{flalign*}
\ty \derives\ & \TTuple(\ty^{*}) &
\end{flalign*}
}

\ASTRuleDef{Ty.TTuple}
\begin{mathpar}
\inferrule{
  \buildplist[\buildty](\vtypes) \astarrow \vtypeasts
}{
  \buildty(\Nty(\namednode{\vtypes}{\Plisttwo{\Nty}})) \astarrow
  \overname{\TTuple(\vtypeasts)}{\vastnode}
}
\end{mathpar}

\subsection{Typing Tuple Types\label{sec:TypingTupleTypes}}
\TypingRuleDef{TTuple}
See \ExampleRef{Well-typed Tuples} for examples of well-typed \tupletypesterm.

\ProseParagraph
\AllApply
\begin{itemize}
  \item $\tty$ is the \Prosetupletype{$\tys$}, that is, $\TTuple(\tys)$;
  \item $\tys$ is the list $\tty_i$, for $i=1..k$ and $k>1$;
  \item annotating each type $\tty_i$ in $\tenv$, for $i=1..k$,
        yields $(\ttyp_i, \vxs_i)$\ProseOrTypeError;
  \item $\newty$ is the \Prosetupletype{$\ttyp_i$}, for $i=1..k$;
  \item define $\vses$ as the union of all $\vxs_i$, for $i=1..k$.
\end{itemize}

\FormallyParagraph
\begin{mathpar}
\inferrule{
  k \geq 2\\
  \tys \eqname \tty_{1..k}\\
  i=1..k: \annotatetype(\False, \tenv, \tty_i) \typearrow (\ttyp_i, \vxs_i) \OrTypeError\\\\
  \vses \eqdef \bigcup_{i=1..k} \vxs_i
}{
  \annotatetype(\overname{\Ignore}{\vdecl}, \tenv, \TTuple(\tys)) \typearrow (\overname{\TTuple(\tysp)}{\newty}, \vses)
}
\end{mathpar}
\CodeSubsection{\TTupleBegin}{\TTupleEnd}{../Typing.ml}

\section{Parenthesized Types\label{sec:ParenthesizedTypes}}
A single type inside parentheses is not considered to be a tuple, but rather the element
inside the parenthesis.
Parenthesizing a type can be used to improve readability.

In \listingref{ParenthesizedTypes}, \verb|(integer)| is a parenthesized type, equivalent to simply \verb|integer| (without parentheses).
\ASLListing{Parenthesized types}{ParenthesizedTypes}{\definitiontests/ParenthesizedTypes.asl}

\subsection{Syntax}
\begin{flalign*}
\Nty \derives\  & \Tlpar \parsesep \Nty \parsesep \Trpar &\\
\end{flalign*}

\subsection{Abstract Syntax}
A parenthesized type produces the abstract syntax of the type itself.
In other words, we ignore the parenthesis.

\ASTRuleDef{ParenType}
\begin{mathpar}
  \inferrule{}{
  \buildexpr(\overname{\Nty(\Tlpar, \punnode{\Nty}, \Trpar)}{\vparsednode}) \astarrow
  \overname{\astof{\tty}}{\vastnode}
}
\end{mathpar}

\section{Enumeration Types\label{sec:EnumerationTypes}}
\hypertarget{enumerationtypeterm}{}
The \emph{\enumerationtypeterm} defines a list of enumeration literals,
also referred to as \emph{labels}, that act
as global constants that can be compared for equality and inequality and used
as indices in enumeration-indexed arrays.
%
The type of an enumeration literal is the anonymous \enumerationtypeterm{}
that defined the literal.
\identd{YZBQ} \identr{HJYJ}

\identi{PRPY}%
Unlike many languages, there is no ordering defined for enumeration literals
and therefore enumeration types do not support ordering comparisons such as \verb|<=|.

\ExampleDef{Well-typed Enumeration Types}
\listingref{typing-tenum} and \listingref{typing-tenum-subtypes} show examples of well-typed
enumeration type declarations.

\ASLListing{Well-typed enumeration type}{typing-tenum-subtypes}{\typingtests/TypingRule.TEnumDecl.subtypes.asl}
\ASLListing{Well-typed enumeration type}{typing-tenum}{\typingtests/TypingRule.TEnumDecl.asl}

\identr{DWSP} \identr{QMWT} \identi{MZXL}%
\RequirementDef{LabelNamespace}
Enumeration literals exist in the same namespace as all other declared identifiers except subprograms (see \RequirementRef{GlobalNamespace}),
including storage elements and named types, so no other declared identifier
may have the same name in the same scope.
In particular, this means that an enumeration literal can be declared in
at most one \enumerationtypeterm{} declaration.

See \ExampleRef{Well-typed Enumeration Types} and \ExampleRef{Ill-typed Enumeration Type Declarations}.

\RequirementDef{AnonymousEnumerations}
Enumeration types are only allowed in declarations.

\listingref{AnonymousEnumerations} shows an illegal specification
where an enumeration is used outside of a type definition.
\ASLListing{Anonymous enumerations are illegal}{AnonymousEnumerations}{\definitiontests/GuideRule.AnonymousEnumerations.bad.asl}

\subsection{Syntax}
\begin{flalign*}
\Ntydecl \derives\ & \Tenumeration \parsesep \Tlbrace \parsesep \TClistOne{\Tidentifier} \parsesep \Trbrace &
\end{flalign*}

\subsection{Abstract Syntax}
\RenderTypes[remove_hypertargets]{ty_enum}
\BackupOriginalAST{
\begin{flalign*}
\ty \derives\ & \TEnum(\overtext{\Identifier^{*}}{labels}) &
\end{flalign*}
}

\ASTRuleDef{TyDecl.TEnum}
\begin{mathpar}
\inferrule{
  \buildtclist[\buildidentity](\vids) \astarrow \vidasts
}{
  {
    \begin{array}{r}
  \buildtydecl(\Ntydecl(\Tenumeration, \Tlbrace, \namednode{\vids}{\TClistOne{\Tidentifier}}, \Trbrace)) \astarrow\\
  \overname{\TEnum(\vidasts)}{\vastnode}
\end{array}
  }
}
\end{mathpar}

\subsection{Typing Enumeration Types\label{sec:TypingEnumerationTypes}}
\TypingRuleDef{TEnumDecl}
See \ExampleRef{Well-typed Enumeration Types} for examples of well-typed \enumerationtypesterm{}
declarations.

\ExampleDef{Ill-typed Enumeration Type Declarations}
\listingref{typing-tenum-bad},
\listingref{typing-tenum-bad2},
\listingref{typing-tenum-bad3}, and
\listingref{typing-tenum-bad4},
show examples of ill-typed enumeration type declarations.
\ASLListing{Ill-typed enumeration types}{typing-tenum-bad}{\typingtests/TypingRule.TEnumDecl.bad.asl}
\ASLListing{Ill-typed enumeration types}{typing-tenum-bad2}{\typingtests/TypingRule.TEnumDecl.bad2.asl}
\ASLListing{Ill-typed enumeration types}{typing-tenum-bad3}{\typingtests/TypingRule.TEnumDecl.bad3.asl}
\ASLListing{Ill-typed enumeration types}{typing-tenum-bad4}{\typingtests/TypingRule.TEnumDecl.bad4.asl}

\ProseParagraph
\AllApply
\begin{itemize}
  \item $\tty$ is the \enumerationtypeterm{} with enumeration literals
        $\vli$, that is, $\TEnum(\vli)$;
  \item $\decl$ is $\True$, indicating that $\tty$ should be considered in the context of a declaration;
  \item determining that $\vli$ does not contain duplicates yields $\True$\ProseOrTypeError;
  \item determining that none of the labels in $\vli$ is declared in the global environment
  yields $\True$\ProseOrTypeError;
  \item $\newty$ is the \enumerationtypeterm{} $\tty$;
  \item define $\vses$ as the empty set.
\end{itemize}

\FormallyParagraph
\begin{mathpar}
\inferrule{
  \checknoduplicates(\vli) \typearrow \True \OrTypeError\\\\
  \vl \in \vli: \checkvarnotingenv(G^\tenv, \vl) \typearrow \True \OrTypeError
}{
  \annotatetype(\True, \tenv, \TEnum(\vli)) \typearrow (\overname{\TEnum(\vli)}{\newty}, \overname{\emptyset}{\vses})
}
\end{mathpar}
\CodeSubsection{\TEnumDeclBegin}{\TEnumDeclEnd}{../Typing.ml}

\section{Array Types\label{sec:ArrayTypes}}
\hypertarget{arraytypeterm}{}
\identr{DFXJ}%
Arrays are sequences of values of a single given type.
The syntax \verb|array [[expr]] of ty| declares a single-dimensional array of type \texttt{ty}
with an index type derived from the expression \texttt{expr}.
%
\identr{YHNV}%
ASL offers two kinds of arrays:
\hypertarget{intarraytypeterm}{}
\hypertarget{enumarraytypeterm}{}
\begin{description}
  \item[\Intarraytypeterm] represents a consecutive list of elements at positions $0$ to the size
      specified for the array. The array elements can be accessed via an \integertypeterm{}
      that specifies the $0$-based position of the element to read/update.
  \item[\Enumarraytypeterm] represents a dictionary-like data type where the keys are defined
      by a given \enumerationtypeterm{}. The array elements can be accessed via values of the
      \enumerationtypeterm{} specified for the array type.
\end{description}

\RequirementDef{ArrayLengthImmutable}
The length of an array cannot be modified.
% NO_EXAMPLE

\RequirementDef{ArrayLengthExpression}
The length expression of an \intarraytypeterm{} must be a \symbolicallyevaluable{}
expression whose \underlyingtype{} is an \integertypeterm{} .

\ExampleDef{Well-typed Array Types}
In \listingref{typing-tarray}, all the uses of array types are well-typed.
\ASLListing{Well-typed array types}{typing-tarray}{\typingtests/TypingRule.TArray.asl}

\subsection{Syntax}
\begin{flalign*}
\Nty \derives\ & \Tarray \parsesep \Tllbracket \parsesep \Nexpr \parsesep \Trrbracket \parsesep \Tof \parsesep \Nty &
\end{flalign*}

\subsection{Abstract Syntax}
\RenderTypes[remove_hypertargets]{ty_array}
\BackupOriginalAST{
\begin{flalign*}
\ty \derives\ & \TArray(\arrayindex, \ty) &\\
\end{flalign*}
}

\RenderType[remove_hypertargets]{array_index}

\ASTRuleDef{Ty.TArray}
\begin{mathpar}
\inferrule{}{
  {
  \begin{array}{r}
    \buildty(\Nty(\Tarray, \Tllbracket, \punnode{\Nexpr}, \Trrbracket, \Tof, \punnode{\Nty})) \astarrow\\
    \overname{\TArray(\ArrayLengthExpr(\astof{\vexpr}), \astof{\tty})}{\vastnode}
  \end{array}
  }
}
\end{mathpar}
\subsection{Typing Array Types\label{sec:TypingArrayTypes}}
\TypingRuleDef{TArray}
See \ExampleRef{Well-typed Array Types} for examples of well-typed
array types.

\ExampleDef{Ill-typed Array Types}
In \listingref{typing-tarray-bad}, the array type for \verb|illegal_array1|
is ill-typed, since the expression \verb|non_symbolically_evaluable|
is not \symbolicallyevaluable{}.
\ASLListing{Ill-typed array types}{typing-tarray-bad}{\typingtests/TypingRule.TArray.bad.asl}
Similarly in \listingref{typing-tarray-bad2}, the array type for \verb|illegal_array2| is ill-typed, since the expression \verb|non_constrained| is not a \constrainedinteger{}.
\ASLListing{Ill-typed array types}{typing-tarray-bad2}{\typingtests/TypingRule.TArray.bad2.asl}

\ProseParagraph
\AllApply
\begin{itemize}
  \item $\tty$ is the array type with element type $\vt$;
  \item Annotating the type $\vt$ in $\tenv$ yields $(\vtp, \vsest)$\ProseOrTypeError;
  \item \OneApplies
  \begin{itemize}
    \item \AllApplyCase{expr\_is\_enum}
    \begin{itemize}
      \item the array index is $\ve$ and determining whether $\ve$ corresponds to an enumeration in $\tenv$
      via $\getvariableenum$ yields the enumeration variable
      name $\vs$ of size $\vi$, that is, $\langle \vs, \vi \rangle$\ProseOrTypeError;
      \item $\newty$ is the array type indexed by an \enumerationtypeterm{}
      named $\vs$ of length $\vi$ and of elements of type $\vtp$, that is, $\TArray(\ArrayLengthEnum(\vs, \vi), \vtp)$;
      \item define $\vses$ as $\vsest$.
    \end{itemize}

    \item \AllApplyCase{expr\_not\_enum}
    \begin{itemize}
      \item the array index is $\ve$ and determining whether $\ve$ corresponds to an enumeration in $\tenv$
      via $\getvariableenum$ yields $\None$ (meaning it does not
      correspond to an enumeration)\ProseOrTypeError;
      \item annotating the \symbolicallyevaluable{} \constrainedinteger{} expression $\ve$ yields\\
      $(\vep, \vsesindex)$\ProseOrTypeError;
      \item $\newty$ the array type indexed by integer bounded by
      the expression $\vep$ and of elements of type $\vtp$, that is,
      $\TArray(\ArrayLengthExpr(\vep), \vtp)$;
      \item define $\vses$ as the union of $\vsest$ and $\vsesindex$.
    \end{itemize}
  \end{itemize}
\end{itemize}

\FormallyParagraph
\begin{mathpar}
\inferrule[expr\_is\_enum]{
  \annotatetype(\False, \tenv, \vt) \typearrow (\vtp, \vsest) \OrTypeError\\\\
  \commonprefixline\\\\
  \getvariableenum(\tenv, \ve) \typearrow \langle \vs, \vlabels \rangle
}{
  \annotatetype(\overname{\Ignore}{\vdecl}, \tenv, \overname{\AbbrevTArrayLengthExpr{\ve}{\vt}}{\tty}) \typearrow
  (\overname{\AbbrevTArrayLengthEnum{\ve}{\vlabels}{\vtp}}{\newty}, \overname{\emptyset}{\vsest})
}
\end{mathpar}

\begin{mathpar}
\inferrule[expr\_not\_enum]{
  \annotatetype(\False, \tenv, \vt) \typearrow (\vtp, \vsest) \OrTypeError\\\\
  \commonprefixline\\\\
  \getvariableenum(\tenv, \ve) \typearrow \None\\
  \annotatesymbolicconstrainedinteger(\tenv, \ve) \typearrow (\vep, \vsesindex) \OrTypeError\\\\
  \vses \eqdef \vsest \cup \vsesindex
}{
  \annotatetype(\overname{\Ignore}{\vdecl}, \tenv, \overname{\AbbrevTArrayLengthExpr{\ve}{\vt}}{\tty}) \typearrow
  (\overname{\AbbrevTArrayLengthExpr{\vep}{\vtp}}{\newty}, \vses)
}
\end{mathpar}
\CodeSubsection{\TArrayBegin}{\TArrayEnd}{../Typing.ml}

\TypingRuleDef{GetVariableEnum}
\hypertarget{def-getvariableenum}{}
The function
\[
\getvariableenum(\overname{\staticenvs}{\tenv} \aslsep \overname{\expr}{\ve}) \aslto
\langle (\overname{\Identifier}{\vx}, \overname{\KleenePlus{\Identifier}}{\vlabels})\rangle
\]
tests whether the expression $\ve$ represents a variable of an \enumerationtypeterm{}.
If so, the result is $\vx$ --- the name of the variable and the list of labels $\vlabels$,
declared for the \enumerationtypeterm{}.
Otherwise, the result is $\None$.

\ExampleDef{Retrieving Enumeration Labels from Variable Expressions}
\listingref{typing-getvariableenum} shows examples of retrieving
enumeration labels from variable expressions.
\ASLListing{Retrieving enumeration labels from expressions}{typing-getvariableenum}{\typingtests/TypingRule.GetVariableEnum.asl}

\ProseParagraph
\OneApplies
\begin{itemize}
  \item \AllApplyCase{not\_evar}
  \begin{itemize}
    \item $\ve$ is not a variable expression;
    \item the result is $\None$.
  \end{itemize}

  \item \AllApplyCase{no\_declared\_type}
  \begin{itemize}
    \item $\ve$ is a variable expression for $\vx$, that is, $\EVar(\vx)$;
    \item $\vx$ is not associated with a type in the global environment of $\tenv$;
    \item the result is $\None$.
  \end{itemize}

  \item \AllApplyCase{declared\_enum}
  \begin{itemize}
    \item $\ve$ is a variable expression for $\vx$, that is, $\EVar(\vx)$;
    \item $\vx$ is associated with a type $\vt$ in the global environment of $\tenv$;
    \item obtaining the \underlyingtype\ of $\vt$ in $\tenv$ yields an \enumerationtypeterm{} with labels $\vlabels$;
    \item the result is the pair consisting of $\vx$ and $\vlabels$.
  \end{itemize}

  \item \AllApplyCase{declared\_not\_enum}
  \begin{itemize}
    \item $\ve$ is a variable expression for $\vx$, that is, $\EVar(\vx)$;
    \item $\vx$ is associated with a type $\vt$ in the global environment of $\tenv$;
    \item obtaining the \underlyingtype\ of $\vt$ in $\tenv$ yields a type that is not an \enumerationtypeterm{};
    \item the result is $\None$.
  \end{itemize}
\end{itemize}

\FormallyParagraph
\begin{mathpar}
\inferrule[not\_evar]{
  \astlabel(\ve) \neq \EVar
}{
  \getvariableenum(\tenv, \ve) \typearrow \None
}
\end{mathpar}

\begin{mathpar}
\inferrule[no\_declared\_type]{
  G^\tenv.\declaredtypes(\vx) = \bot
}{
  \getvariableenum(\tenv, \overname{\EVar(\vx)}{\ve}) \typearrow \None
}
\end{mathpar}

\begin{mathpar}
\inferrule[declared\_enum]{
  G^\tenv.\declaredtypes(\vx) = (\vt, \Ignore)\\
  \makeanonymous(\tenv, \vt) \typearrow \TEnum(\vlabels)
}{
  \getvariableenum(\tenv, \overname{\EVar(\vx)}{\ve}) \typearrow \langle(\vx, \vlabels)\rangle
}
\end{mathpar}

\begin{mathpar}
\inferrule[declared\_not\_enum]{
  G^\tenv.\declaredtypes(\vx) = (\vt, \Ignore)\\
  \makeanonymous(\tenv, \vt) \typearrow \vtone\\
  \astlabel(\vtone) \neq \TEnum
}{
  \getvariableenum(\tenv, \overname{\EVar(\vx)}{\ve}) \typearrow \None
}
\end{mathpar}

\identi{PKXK} \identd{YYDW}%
\RequirementDef{SymbolicallyEvaluable}
An expression is \symbolicallyevaluable{} if its evaluation only involves
the use of immutable values.

See \ExampleRef{Annotating Symbolically Evaluable Expressions}.

\TypingRuleDef{AnnotateSymbolicallyEvaluableExpr}
\hypertarget{def-annotatesymbolicallyevaluableexpr}{}
The function
\[
\begin{array}{r}
  \annotatesymbolicallyevaluableexpr(\overname{\staticenvs}{\tenv} \aslsep \overname{\expr}{\ve}) \aslto \\
  (\overname{\ty}{\vt}\times\overname{\expr}{\vep}\times\overname{\TSideEffectSet}{\vses}) \cup \overname{\typeerror}{\TypeErrorConfig}
\end{array}
\]
annotates the expression $\ve$ in the \staticenvironmentterm{} $\tenv$ and checks that it is \symbolicallyevaluable,
yielding the type in $\vt$, the annotated expression $\vep$ and the \sideeffectsetterm{} $\vses$.
\ProseOtherwiseTypeError

\ExampleDef{Annotating Symbolically Evaluable Expressions}
\listingref{typing-annotatesymbolicallyevaluableexpr} shows examples of
expressions and classifies them as either \symbolicallyevaluable{} or not.
\ASLListing{Annotating symbolically evaluable Expressions}{typing-annotatesymbolicallyevaluableexpr}
{\typingtests/TypingRule.AnnotateSymbolicallyEvaluableExpr.asl}

\ProseParagraph
\AllApply
\begin{itemize}
  \item \Proseannotateexpr{$\tenv$}{$\ve$}{$(\vt, \vep, \vses)$};
  \item \Prosechecksymbolicallyevaluable{$\vses$}.
\end{itemize}

\FormallyParagraph
\begin{mathpar}
\inferrule{
  \annotateexpr(\tenv, \ve) \typearrow (\vt, \vep, \vses) \OrTypeError\\\\
  \checksymbolicallyevaluable(\vses) \typearrow \True \OrTypeError
}{
  \annotatesymbolicallyevaluableexpr(\tenv, \ve) \typearrow (\vt, \vep, \vses)
}
\end{mathpar}
\CodeSubsection{\AnnotateSymbolicallyEvaluableExprBegin}{\AnnotateSymbolicallyEvaluableExprEnd}{../Typing.ml}

\hypertarget{def-checkunderlyinginteger}{}
\TypingRuleDef{CheckUnderlyingInteger}
The function
\[
  \checkunderlyinginteger(\overname{\staticenvs}{\tenv} \aslsep \overname{\ty}{\vt}) \aslto
  \{\True\} \cup \typeerror
\]
returns $\True$ is $\vt$ is has the \underlyingtype{} of an \integertypeterm{}.
\ProseOtherwiseTypeError

\ExampleDef{Checking for an Underlying Integer Type}
All of the expressions appearing on the right-hand-side of the assignments in
\listingref{typing-annotatesymbolicallyevaluableexpr}
have an \integertypeterm{} as their \underlyingtype{}.
This includes the expression \verb|i - 3|, since the \underlyingtype{}
of \verb|i| is the \unconstrainedintegertype.

\ProseParagraph
\AllApply
\begin{itemize}
  \item determining the \underlyingtype{} of $\vt$ yields $\vtp$\ProseOrTypeError;
  \item checking that $\vtp$ is an \integertypeterm{} yields $\True$\ProseTerminateAs{\UnexpectedType};
  \item the result is $\True$;
\end{itemize}

\FormallyParagraph
\begin{mathpar}
\inferrule{
  \makeanonymous(\tenv, \vt) \typearrow \vtp \OrTypeError\\\\
  \checktrans{\astlabel(\vtp) = \TInt}{\UnexpectedType} \typearrow \True \OrTypeError
}{
  \checkunderlyinginteger(\tenv, \vt) \typearrow \True
}
\end{mathpar}
\CodeSubsection{\CheckUnderlyingIntegerBegin}{\CheckUnderlyingIntegerEnd}{../Typing.ml}

\section{Record Types\label{sec:RecordTypes}}
\hypertarget{recordtypeterm}{}
\identd{WGQS}%
A record is a \structuredtype{} consisting of a list of field identifiers which denote individual storage elements.
\identr{DXWN}%
A record type is described by specifying for each field identifier its type.

\ExampleDef{Well-typed Record Types}
In \listingref{typing-trecord}, all the uses of record types are well-typed.
\ASLListing{Well-typed structured types}{typing-trecord}{\typingtests/TypingRule.TRecordDecl.asl}

In \listingref{typing-trecord-bad}, the \recordtypeterm{} \verb|MyRecord|
is ill-typed, since the field \verb|v| repeats.
\ASLListing{A record type with repeated fields}{typing-trecord-bad}{\typingtests/TypingRule.TRecordDecl.bad.asl}

\subsection{Syntax}
\begin{flalign*}
\Ntydecl \derives\ & \Trecord \parsesep \Nfields &
\end{flalign*}

\subsection{Abstract Syntax}
\RenderTypes[remove_hypertargets]{ty_record}
\BackupOriginalAST{
\begin{flalign*}
\ty \derives\ & \TRecord(\field^{*}) &
\end{flalign*}
}

\ASTRuleDef{TyDecl.TRecord}
\begin{mathpar}
\inferrule{}{
  \buildtydecl(\Ntydecl(\Trecord, \punnode{\Nfields})) \astarrow
  \overname{\TRecord(\astof{\vfields})}{\vastnode}
}
\end{mathpar}

\subsection{Typing Record Types\label{sec:TypingRecordTypes}}
\TypingRuleDef{TStructuredDecl}
\ExampleRef{Well-typed Record Types} shows examples of well-typed \recordtypesterm.

\ProseParagraph
\AllApply
\begin{itemize}
  \item $\tty$ is a \structuredtype\ with AST label $L$;
  \item the list of fields of $\tty$ is $\fields$;
  \item $\decl$ is $\True$, indicating that $\tty$ should be considered in the context of a declaration;
  \item $\fields$ is a list of pairs where the first element is an identifier and the second is a type --- $(\vx_i, \vt_i)$, for $i=1..k$;
  \item checking that the list of field identifiers $\vx_{1..k}$ does not contain duplicates
  yields $\True$\ProseOrTypeError;
  \item annotating each field type $\vt_i$, for $i=1..k$, yields $(\vtp_i, \vxs_i)$
        \ProseOrTypeError;
  \item $\fieldsp$ is the list with $(\vx_i, \vtp_i)$, for $i=1..k$;
  \item $\newty$ is the AST node with AST label $L$ (either record type or exception type,
  corresponding to the type $\tty$) and fields $\fieldsp$;
  \item define $\vses$ as the union of all $\vxs_i$, for $i=1..k$.
\end{itemize}

\FormallyParagraph
\begin{mathpar}
\inferrule{
  L \in \{\TRecord, \TException\}\\
  \fields \eqname [i=1..k: (\vx_i, \vt_i)]\\
  \checknoduplicates(\vx_{1..k}) \typearrow \True \OrTypeError\\\\
  i=1..k: \annotatetype(\False, \tenv, \vt_i) \typearrow (\vtp_i, \vxs_i) \OrTypeError\\\\
  \fieldsp \eqdef [i=1..k: (\vx_i, \vtp_i)]\\
  \vses \eqdef \bigcup_{i=1..k} \vxs_i
}{
  \annotatetype(\True, \tenv, \overname{L(\fields)}{\tty}) \typearrow (\overname{L(\fieldsp)}{\newty}, \vses)
}
\end{mathpar}
\CodeSubsection{\TStructuredDeclBegin}{\TStructuredDeclEnd}{../Typing.ml}

\section{Exception Types\label{sec:ExceptionTypes}}
\hypertarget{exceptiontypeterm}{}
An exception is a \structuredtype{} consisting of a list of field identifiers
which denote individual storage elements.

\ExampleDef{Well-typed Exception Types}
In \listingref{typing-texception}, all the uses of exception types are well-typed.
\ASLListing{Well-typed exception types}{typing-texception}{\typingtests/TypingRule.TExceptionDecl.asl}

\subsection{Syntax}
\begin{flalign*}
\Ntydecl \derives\ & \Texception \parsesep \Nfields &
\end{flalign*}

\subsection{Abstract Syntax}
\RenderTypes[remove_hypertargets]{ty_exception}
\BackupOriginalAST{
\begin{flalign*}
\ty \derives\ & \TException(\field^{*}) &
\end{flalign*}
}

\ASTRuleDef{TyDecl.TException}
\begin{mathpar}
\inferrule{}{
  \buildtydecl(\Ntydecl(\Texception, \punnode{\Nfields})) \astarrow
  \overname{\TException(\astof{\vfields})}{\vastnode}
}
\end{mathpar}

\subsection{Typing Exception Types}
The rule for typing exception types is \TypingRuleRef{TStructuredDecl}.

\section{Collection Types\label{sec:CollectionTypes}}
\hypertarget{collectiontypeterm}{}

\ASLListing{Collection types}{TCollection}{\typingtests/TypingRule.TCollection.asl}

A collection is a \structuredtype{} consisting of a list of field identifiers
which denote individual storage elements.

\RequirementDef{CollectionsGlobal}
Collections can only be used for global storage elements.
See \listingref{checkisnotcollection} for an ill-typed specification.

\subsection{Syntax}
\begin{flalign*}
  \Ntyorcollection \derives\ & \Nty \;|\; \Tcollection \parsesep \Nfields & \\
\end{flalign*}

\subsection{Abstract Syntax}
\RenderTypes[remove_hypertargets]{ty_collection}
\BackupOriginalAST{
\begin{flalign*}
\ty \derives\ & \TCollection(\field^{*}) &
\end{flalign*}
}

\ASTRuleDef{TyDecl.TCollection}
\begin{mathpar}
\inferrule{}{
  \buildtyorcollection(\Ntyorcollection(\Tcollection, \punnode{\Nfields})) \astarrow
  \overname{\TCollection(\astof{\vfields})}{\vastnode}
}
\and
\inferrule{}{
  \buildtyorcollection(\Ntyorcollection(\Nty)) \astarrow
  \overname{\astof{\Nty}}{\vastnode}
}
\end{mathpar}

\section{Typing Collection Types}
\ExampleDef{Typing Collection Types}
\listingref{TCollection} shows examples of well-typed collection types
and ill-typed collection types in comments.
In addition, \listingref{TCollection} shows well-typed storage declarations
utilizing collection types and ill-typed storage declarations utilizing
collection types in comments.

\ProseParagraph
\AllApply
\begin{itemize}
  \item $\tty$ is a \collectiontypeterm{} with the list of fields of $\fields$;
  \item $\decl$ is $\False$, indicating that $\tty$ should not be considered in the context of a declaration;
  \item $\fields$ is a list of pairs where the first element is an identifier and the second is a type --- $(\vx_i, \vt_i)$, for $i=1..k$;
  \item checking that the list of field identifiers $\vx_{1..k}$ does not contain duplicates
  yields $\True$\ProseOrTypeError;
  \item annotating each field type $\vt_i$, for $i=1..k$, yields $(\vtp_i, \vxs_i)$
        \ProseOrTypeError;
  \item $\fieldsp$ is the list with $(\vx_i, \vtp_i)$, for $i=1..k$;
  \item checking that the \structure{} of the type $\vtp_i$ in the \staticenvironmentterm{} $\tenv$
        is a \bitvectortypeterm, for every $i$ in $1..k$, yields $\True$\ProseOrTypeError;
  \item $\newty$ is the AST node with AST label $L$ (either record type or exception type,
        corresponding to the type $\tty$) and fields $\fieldsp$;
  \item define $\vses$ as the union of all $\vxs_i$, for $i=1..k$.
\end{itemize}

\FormallyParagraph
\begin{mathpar}
\inferrule{
  \fields \eqname [i=1..k: (\vx_i, \vt_i)]\\
  \checknoduplicates(\vx_{1..k}) \typearrow \True \OrTypeError\\\\
  i=1..k: \annotatetype(\False, \tenv, \vt_i) \typearrow (\vtp_i, \vxs_i) \OrTypeError\\\\
  \fieldsp \eqdef [i=1..k: (\vx_i, \vtp_i)]\\
  i=1..k: \checkstructurelabel(\tenv, \vtp_i, \TBits) \typearrow \True \OrTypeError\\\\
  \vses \eqdef \bigcup_{i=1..k} \vxs_i
}{
  \annotatetype(\False, \tenv, \overname{\TCollection(\fields)}{\tty}) \typearrow (\overname{\TCollection(\fieldsp)}{\newty}, \vses)
}
\end{mathpar}

\section{Named Types\label{sec:NamedTypes}}
Named type declarations allow declaring new types names with a given type.
The intent is that, by default, any named type should not be assignable to or from any other named type, even
if they coincidentally have the same \structure. Where assignment between named types is desired it is usually
achieved by grouping related types via a common supertype.

The specification in \listingref{named-types2}, declares two unique types with the same
\structure. Note that the two types are not related in
any way and are not interchangeable. See also \TypingRuleRef{TypeSatisfaction}.
\ASLListing{Well-typed named types}{named-types2}{\definitiontests/NamedTypes2.asl}

In \listingref{named-types}, \verb|TypeC.f| and \verb|TypeD.f| have the same type: \verb|TypeB| (and not integer).
\ASLListing{Two unique well-typed types}{named-types}{\definitiontests/NamedTypes.asl}

In \listingref{named-types3}, the assignments \verb|addr = physical;|
and \verb|physical = addr;| would be illegal.
\ASLListing{Two unique well-typed types}{named-types3}{\definitiontests/NamedTypes3.asl}

The well-typed specification in \listingref{named-types4}
demonstrates how a constant can be given a named type.
\ASLListing{Naming types for constants}{named-types4}{\definitiontests/NamedTypes4.asl}

The specification in \listingref{named-types-bad} is ill-typed
due to an assignment between two different named types (\verb|Char| and \verb|Byte|).
\ASLListing{Named types and illegal assignments}{named-types-bad}{\definitiontests/NamedTypes.bad.asl}

\hypertarget{namedtypeterm}{}
\subsection{Syntax}
\begin{flalign*}
\Nty \derives\ & \Tidentifier &
\end{flalign*}

\subsection{Abstract Syntax}
\RenderTypes[remove_hypertargets]{ty_named}
\BackupOriginalAST{
\begin{flalign*}
\ty \derives\ & \TNamed(\overtext{\Identifier}{type name}) &
\end{flalign*}
}

\ASTRuleDef{Ty.TNamed}
\begin{mathpar}
\inferrule{}{
  \buildty(\Nty(\Tidentifier(\id))) \astarrow
  \overname{\TNamed(\id)}{\vastnode}
}
\end{mathpar}

\subsection{Typing Named Types\label{sec:TypingNamedTypes}}
\TypingRuleDef{TNamed}
\ExampleDef{Well-typed Named Types}
In \listingref{typing-tnamed}, all the uses of \texttt{MyType} are well-typed.
\ASLListing{Well-typed named types}{typing-tnamed}{\typingtests/TypingRule.TNamed.asl}

The specifications in \listingref{typing-tnamed-bad1} and \listingref{typing-tnamed-bad2}
show examples of ill-typed named types.
\ASLListing{Ill-typed named types}{typing-tnamed-bad1}{\typingtests/TypingRule.TNamed.bad1.asl}
\ASLListing{Ill-typed named types}{typing-tnamed-bad2}{\typingtests/TypingRule.TNamed.bad2.asl}

\ProseParagraph
\AllApply
\begin{itemize}
  \item $\tty$ is the named type $\vx$, that is $\TNamed(\vx)$;
  \item checking whether $\vx$ is bound to any declared type in $\tenv$ yields $\True$\ProseOrTypeError;
  \item $\vx$ is bound to a type with associated \purity{} $\vpurity$;
  \item define $\vses$ as the set of the \GlobalEffectTerm\ with purity $\vpurity$ and the \ImmutabilityTerm\ with immutability $\True$;
  \item $\newty$ is $\tty$.
\end{itemize}

\FormallyParagraph
\begin{mathpar}
\inferrule{
  \checktrans{G^\tenv.\declaredtypes(\vx) \neq \bot}{\UndefinedIdentifier} \typearrow \True \OrTypeError\\\\
  G^\tenv.\declaredtypes(\vx) = (\Ignore, \vpurity)\\
  \vses \eqdef \{\ \GlobalEffectTerm(\vpurity), \Immutability(\True) \}
}{
  \annotatetype(\overname{\Ignore}{\vdecl}, \tenv, \overname{\TNamed(\vx)}{\tty}) \typearrow (\overname{\TNamed(\vx)}{\newty}, \vses)
}
\end{mathpar}
\CodeSubsection{\TNamedBegin}{\TNamedEnd}{../Typing.ml}

\section{Declared Types\label{sec:DeclaredTypes}}
A declared type can be an \enumerationtypeterm{}, a \recordtypeterm, an \exceptiontypeterm, or an \anonymoustype.
\subsection{Syntax}
\begin{flalign*}
\Ntydecl \derives\ & \Nty &
\end{flalign*}

\subsection{Abstract Syntax}
\ASTRuleDef{TyDecl}
\begin{mathpar}
\inferrule{}{
  \buildtydecl(\Ntydecl(\punnode{\Nty})) \astarrow
  \overname{\astof{\tty}}{\vastnode}
}
\end{mathpar}

\subsection{Typing Declared Types}
\identr{RGRVJ}%
\RequirementDef{RestrictionsOnAnonymousTypes}
A declared type for an enumeration, a record type, or an exception type
are only permitted in named type declarations. This is enforced by \TypingRuleRef{TNonDecl}.
%
See \ExampleRef{Ill-typed pending-constrained integer type}.

\TypingRuleDef{TNonDecl}
\ExampleDef{Ill-typed Type Declarations}
In \listingref{typing-trecorderror}, the use of a record type outside of a declaration is erroneous.
\ASLListing{An erroneous use of a record type}{typing-trecorderror}{\typingtests/TypingRule.TNonDecl.asl}

\ProseParagraph
\AllApply
\begin{itemize}
  \item $\tty$ is a \structuredtype\ or an \enumerationtypeterm{};
  \item $\decl$ is $\False$, indicating that $\tty$ should be considered to be outside the context of a declaration
  of $\tty$;
  \item a \typingerrorterm{} is returned, indicating that the use of anonymous form of enumerations, record,
  and exceptions types is not allowed here.
\end{itemize}

\FormallyParagraph
\begin{mathpar}
\inferrule{
  \astlabel(\tty) \in \{\TEnum, \TRecord, \TException\}
}{
  \annotatetype(\False, \tenv, \tty) \typearrow \TypeErrorVal{\UnexpectedType}
}
\end{mathpar}
\CodeSubsection{\TNonDeclBegin}{\TNonDeclEnd}{../Typing.ml}

%%%%%%%%%%%%%%%%%%%%%%%%%%%%%%%%%%%%%%%%%%%%%%%%%%%%%%%%%%%%%%%%%%%%%%%%%%%%%%%%%%%%
\section{Domain of Values for Types\label{sec:DomainOfValuesForTypes}}
%%%%%%%%%%%%%%%%%%%%%%%%%%%%%%%%%%%%%%%%%%%%%%%%%%%%%%%%%%%%%%%%%%%%%%%%%%%%%%%%%%%%
This section formalizes the concept of the set of values for a given type.
The formalism is given in the form of inference rules, although those are not meant
to be implemented.

We define the concept of a \emph{dynamic domain} of a type
and the \emph{static domain} of a type.
Intuitively, domains assign potentially infinite sets of \nativevaluesterm{} to types.
Dynamic domains are used by the semantics to evaluate expressions of the form \texttt{ARBITRARY: t}
by choosing a single value from the dynamic domain of $\vt$ (\SemanticsRuleRef{EArbitrary}).
Static domains are used to define subtype satisfaction (\TypingRuleRef{SubtypeSatisfaction})
via a conservative subsumption test as defined in \chapref{SymbolicDomainSubsetTesting}.

\subsection{Dynamic Domain of a Type\label{sec:DynDomain}}
\hypertarget{def-dyndomain}{}
The dynamic domain is defined via the partial function
\[
  \dynamicdomain : \overname{\envs}{\env} \times \overname{\ty}{\vt}
  \partialto \overname{\pow{\nativevalue}}{\vd}
\]
which assigns the set of values that a type $\vt$ can hold in a given environment $\env$.
%
We say that $\dynamicdomain(\env, \vt)$ is the \emph{dynamic domain} of $\vt$
in the environment $\env$.
%
The \emph{static domain} of a type is the set of values which storage elements of that type may hold
\underline{across all possible dynamic environments}.
%
The reason for this distinction is that the sets of values
of integer types, bitvector types, and array types can depend on the dynamic values of variables.

Types that do not refer to variables whose values are only known dynamically have
a static domain that is equal to any of their dynamic domains.
In those cases, we simply refer to their \emph{domain}.

Associating a set of values to a type is done by evaluating any expression appearing
in the type definitions.
%
Expressions appearing in types are guaranteed to be side-effect-free by the
function $\annotatetype$.
%
Evaluation is defined by the relation $\evalexprsef\empty$.
which evaluates side-effect-free expressions and either returns
a configuration of the form $\ResultExprSEF(\vv,\vg)$ or a dynamic error configuration $\DynErrorConfig$.
In the first case, $\vv$ is a \nativevalueterm{} and $\vg$
is an \emph{execution graph}. Execution graphs are related to the concurrent semantics
and can be ignored in the context of defining dynamic domains.
In the latter case (which can occur if, for example, an expression attempts to divide
\texttt{8} by \texttt{0}), a \DynamicErrorConfigurationTerm{}, for which we use the notation
$\DynErrorConfig$, is returned.
%
The dynamic domain is empty in cases where evaluating side-effect-free expressions
results in a \DynamicErrorConfigurationTerm{}.
%
The dynamic domain is undefined if the type $\vt$ is not well-typed in $\tenv$.
That is, if $\annotatetype(\tenv, \vt) \typearrow \TypeErrorConfig$.

As part of the definition, we also associate dynamic domains to integer constraints
by overloading $\dynamicdomain$:
\[
  \dynamicdomain : \overname{\envs}{\env} \times \overname{\intconstraint}{\vc}
  \partialto \overname{\pow{\nativevalue}}{\vd}
\]

\ProseParagraph
\OneApplies
\begin{itemize}
  \item \AllApplyCase{t\_bool}
  \begin{itemize}
    \item $\vt$ is the Boolean type, $\TBool$;
    \item $\vd$ is the set of native Boolean values, $\tbool$.
  \end{itemize}

  \item \AllApplyCase{t\_string}
  \begin{itemize}
    \item $\vt$ is the \stringtypeterm{}, $\TString$;
    \item $\vd$ is the set of all native string values, $\tstring$.
  \end{itemize}

  \item \AllApplyCase{t\_real}
  \begin{itemize}
    \item $\vt$ is the \realtypeterm{}, $\TReal$;
    \item $\vd$ is the set of all native real values, $\treal$.
  \end{itemize}

  \item \AllApplyCase{t\_enumeration}
  \begin{itemize}
    \item $\vt$ is the \enumerationtypeterm{} with labels $\vl_{1..k}$, that is $\TEnum(\vl_{1..k})$;
    \item $\vd$ is the set of all native labels $\nvlabel(\vl_\vi)$, for $\vi = 1..k$.
  \end{itemize}

  \item \AllApplyCase{t\_int\_unconstrained}
  \begin{itemize}
    \item $\vt$ is the unconstrained integer type, $\unconstrainedinteger$;
    \item $\vd$ is the set of all native integer values, $\tint$.
  \end{itemize}

  \item \AllApplyCase{t\_int\_well\_constrained}
  \begin{itemize}
    \item $\vt$ is the well-constrained integer type $\TInt(\WellConstrained(\vc_{1..k}))$;
    \item $\vd$ is the union of the dynamic domains of each of the constraints $\vc_{1..k}$ in $\env$.
  \end{itemize}

  \item \AllApplyCase{constraint\_exact\_okay}
  \begin{itemize}
    \item $\vc$ is a constraint consisting of a single side-effect-free expression $\ve$, that is, $\ConstraintExact(\ve)$;
    \item evaluating $\ve$ in $\env$ results in a configuration with the native integer for $n$;
    \item $\vd$ is the set containing the single native integer value for $n$.
  \end{itemize}

  \item \AllApplyCase{constraint\_exact\_abnormal}
  \begin{itemize}
    \item $\vc$ is a constraint consisting of a single side-effect-free expression $\ve$, that is, $\ConstraintExact(\ve)$;
    \item evaluating $\ve$ in $\env$ results in either a dynamic error configuration or diverges;
    \item $\vd$ is the empty set.
  \end{itemize}

  \item \AllApplyCase{constraint\_range\_okay}
  \begin{itemize}
    \item $\vc$ is a range constraint consisting of a two side-effect-free expressions $\veone$ and $\vetwo$, that is, $\ConstraintRange(\veone, \vetwo)$;
    \item evaluating $\veone$ in $\env$ results in a configuration with the native integer for $a$;
    \item evaluating $\vetwo$ in $\env$ results in a configuration with the native integer for $b$;
    \item $\vd$ is the set containing all native integer values for integers greater or equal to $a$ and less than or equal to $b$.
  \end{itemize}

  \item \AllApplyCase{constraint\_range\_abnormal1}
  \begin{itemize}
    \item $\vc$ is a range constraint consisting of a two side-effect-free expressions $\veone$ and $\vetwo$, that is, $\ConstraintRange(\veone, \vetwo)$;
    \item evaluating $\veone$ in $\env$ results in either a dynamic error configuration or diverges;
    \item $\vd$ is the empty set.
  \end{itemize}

  \item \AllApplyCase{constraint\_range\_abnormal2}
  \begin{itemize}
    \item $\vc$ is a range constraint consisting of a two side-effect-free expressions $\veone$ and $\vetwo$, that is, $\ConstraintRange(\veone, \vetwo)$;
    \item evaluating $\veone$ in $\env$ results in a configuration with the native integer for $a$;
    \item evaluating $\vetwo$ in $\env$ results in either a dynamic error configuration or diverges;
    \item $\vd$ is the empty set.
  \end{itemize}

  \item \AllApplyCase{t\_int\_parameterized}
  \begin{itemize}
    \item $\vt$ is a \parameterizedintegertypeterm\ for parameter $\id$, \\ $\TInt(\Parameterized(\id))$;
    \item the \nativevalueterm{} associated with $\id$ in the local dynamic environment is the native integer value for $n$;
    \item $\vd$ is the set containing the single integer value for $n$.
  \end{itemize}

  \item \AllApplyCase{t\_bits\_dynamic\_error}
  \begin{itemize}
    \item $\vt$ is a bitvector type with size expression $\ve$, $\TBits(\ve, \Ignore)$;
    \item evaluating $\ve$ in $\env$ results in either a dynamic error configuration or diverges;
    \item $\vd$ is the empty set.
  \end{itemize}

  \item \AllApplyCase{t\_bits\_negative\_width\_error}
  \begin{itemize}
    \item $\vt$ is a bitvector type with size expression $\ve$, $\TBits(\ve, \Ignore)$;
    \item evaluating $\ve$ in $\env$ results in a configuration with the native integer for $k$;
    \item $k$ is negative;
    \item $\vd$ is the empty set.
  \end{itemize}

  \item \AllApplyCase{t\_bits\_empty}
  \begin{itemize}
    \item $\vt$ is a bitvector type with size expression $\ve$, $\TBits(\ve, \Ignore)$;
    \item evaluating $\ve$ in $\env$ results in a configuration with the native integer for $0$;
    \item $\vd$ is the set containing the single \nativevalueterm{} for an empty bitvector.
  \end{itemize}

  \item \AllApplyCase{t\_bits\_non\_empty}
  \begin{itemize}
    \item $\vt$ is a bitvector type with size expression $\ve$, $\TBits(\ve, \Ignore)$;
    \item evaluating $\ve$ in $\env$ results in a configuration with the native integer for $k$;
    \item $k$ is greater than $0$;
    \item $\vd$ is the set containing all \nativevaluesterm{} for bitvectors of size exactly $k$.
  \end{itemize}

  \item \AllApplyCase{t\_tuple}
  \begin{itemize}
    \item $\vt$ is a \tupletypeterm{} over types $\vt_i$, for $i=1..k$, $\TTuple(\vt_{1..k})$;
    \item the domain of each element $\vt_i$ is $D_i$, for $i=1..k$;
    \item evaluating $\ve$ in $\env$ results in a configuration with the native integer for $k$;
    \item $\vd$ is the set containing all native vectors of $k$ values, where the value at position $i$
    is from $D_i$.
  \end{itemize}

  \item \AllApply
  \begin{itemize}
    \item $\vt$ is an integer-indexed array type with length expression $\ve$ and element type $\telem$, $\TArray(\ArrayLengthExpr(\ve), \telem)$;
    \item \OneApplies
      \begin{itemize}
      \item \AllApplyCase{t\_array\_dynamic\_error}
      \begin{itemize}
        \item evaluating $\ve$ in $\env$ results in either a dynamic error configuration or diverges;
        \item $\vd$ is the empty set.
      \end{itemize}

      \item \AllApplyCase{t\_array\_negative\_length\_error}
      \begin{itemize}
        \item evaluating $\ve$ in $\env$ results in a configuration with the native integer for $k$;
        \item $k$ is negative;
        \item $\vd$ is the empty set.
      \end{itemize}

      \item \AllApplyCase{t\_array\_okay}
      \begin{itemize}
        \item evaluating $\ve$ in $\env$ results in a configuration with the native integer for $k$;
        \item $k$ is greater than or equal to $0$;
        \item the domain of $\vtone$ is $D_\telem$;
        \item $\vd$ is the set of all native vectors of $k$ values taken from $D_\telem$.
      \end{itemize}
    \end{itemize}
  \end{itemize}

  \item \AllApplyCase{t\_enum\_array}
  \begin{itemize}
    \item $\vt$ is an enumeration-indexed array type with for the enumeration $\id$ with $k$ labels and element type $\telem$,
          $\TArray(\ArrayLengthEnum(\id, k), \telem)$;
    \item view $\env$ as the pair consisting of the \staticenvironmentterm{} $\tenv$ and a dynamic environment;
    \item the type bound to $\id$ in the $\declaredtypes$ map of the \staticenvironmentterm{} of $\tenv$ is the \enumerationtypeterm{}
          for the labels $\vl_{1..k}$, that is, $\TEnum(\vl_{1..k})$;
    \item the dynamic domain of $\telem$ in $\env$ is $D_\telem$;
    \item $\vd$ is the set of all native records where each $\vl_i$ is mapped to a value taken from $D_\telem$, for $i=1..k$.
  \end{itemize}

  \item \AllApplyCase{t\_structured}
  \begin{itemize}
    \item $\vt$ is a \structuredtypeterm\ with typed fields $(\id_i, \vt_i)$, for $i=1..k$, that is $L([i=1..k: (\id_i,\vt_i))]$
    where $L\in\{\TRecord, \TException\}$;
    \item the domain of each type $\vt_i$ is $D_i$, for $i=1..k$;
    \item $\vd$ is the set containing all native records where $\id_i$ is mapped to a value taken from $D_i$, for $i=1..k$.
  \end{itemize}

  \item \AllApplyCase{t\_named}
  \begin{itemize}
    \item $\vt$ is a named type with name $\id$, $\TNamed(\id)$;
    \item the type associated with $\id$ in $\tenv$ is $\tty$;
    \item $\vd$ is the domain of $\tty$ in $\env$.
  \end{itemize}
\end{itemize}

\FormallyParagraph

\begin{mathpar}
\inferrule[t\_bool]{}{ \dynamicdomain(\env, \overname{\TBool}{\vt}) = \overname{\tbool}{\vd} }
\and
\inferrule[t\_string]{}{ \dynamicdomain(\env, \overname{\TString}{\vt}) = \overname{\tstring}{\vd} }
\and
\inferrule[t\_real]{}{ \dynamicdomain(\env, \overname{\TReal}{\vt}) = \overname{\treal}{\vd} }
\and
\inferrule[t\_enumeration]{}{
  \dynamicdomain(\env, \overname{\TEnum(\vl_{1..k})}{\vt}) = \overname{\{ \vi = 1..k: \nvlabel(\vl_\vi) \}}{\vd}
}
\end{mathpar}

\begin{mathpar}
  \inferrule[t\_int\_unconstrained]{}{
  \dynamicdomain(\env, \overname{\unconstrainedinteger}{\vt}) = \overname{\tint}{\vd}
}
\end{mathpar}

\begin{mathpar}
\inferrule[t\_int\_well\_constrained]{}{
  \dynamicdomain(\env, \overname{\TInt(\WellConstrained(\vc_{1..k}))}{\vt}) = \overname{\bigcup_{i=1}^k \dynamicdomain(\env, \vc_i)}{\vd}
}
\end{mathpar}

\begin{mathpar}
\inferrule[constraint\_exact\_okay]{
  \evalexprsef{\env, \ve} \evalarrow \ResultExprSEF(\nvint(n), \Ignore)
}{
  \dynamicdomain(\env, \overname{\ConstraintExact(\ve)}{\vc}) = \overname{\{ \nvint(n) \}}{\vd}
}
\and
\inferrule[constraint\_exact\_abnormal]{
  \evalexprsef{\env, \ve} \evalarrow C\\
  C \in \TDynError \cup \TDiverging
}{
  \dynamicdomain(\env, \overname{\ConstraintExact(\ve)}{\vc}) = \overname{\emptyset}{\vd}
}
\end{mathpar}

\begin{mathpar}
\inferrule[constraint\_range\_okay]{
  \evalexprsef{\env, \veone} \evalarrow \ResultExprSEF(\nvint(a), \Ignore)\\
  \evalexprsef{\env, \vetwo} \evalarrow \ResultExprSEF(\nvint(b), \Ignore)
}{
  \dynamicdomain(\env, \overname{\ConstraintRange(\veone, \vetwo)}{\vc}) = \overname{\{ \nvint(n) \;|\;  a \leq n \land n \leq b\}}{\vd}
}
\and
\inferrule[constraint\_range\_abnormal1]{
  \evalexprsef{\env, \veone} \evalarrow C\\
  C \in \TDynError \cup \TDiverging
}{
  \dynamicdomain(\env, \overname{\ConstraintRange(\veone, \vetwo)}{\vc}) = \overname{\emptyset}{\vd}
}
\and
\inferrule[constraint\_range\_abnormal2]{
  \evalexprsef{\env, \veone} \evalarrow \ResultExprSEF(\Ignore, \Ignore)\\
  \evalexprsef{\env, \vetwo} \evalarrow C\\
  C \in \TDynError \cup \TDiverging
}{
  \dynamicdomain(\env, \overname{\ConstraintRange(\veone, \vetwo)}{\vc}) = \overname{\emptyset}{\vd}
}
\end{mathpar}

The notation $L^\denv(\id)$ denotes the \nativevalueterm{} associated with the identifier $\id$
in the \emph{local dynamic environment} of $\denv$.
\begin{mathpar}
  \inferrule[t\_int\_parameterized]{
  L^\denv(\id) = \nvint(n)
}{
  \dynamicdomain(\env, \overname{\TInt(\Parameterized(\id))}{\vt}) = \overname{\{ \nvint(n) \}}{\vd}
}
\end{mathpar}

\begin{mathpar}
\inferrule[t\_bits\_abnormal]{
  \evalexprsef{\env, \ve} \evalarrow C\\
  C \in \TDynError \cup \TDiverging
}{
  \dynamicdomain(\env, \overname{\TBits(\ve, \Ignore)}{\vt}) = \overname{\emptyset}{\vd}
}
\and
\inferrule[t\_bits\_negative\_width\_error]{
  \evalexprsef{\env, \ve} \evalarrow \ResultExprSEF(\nvint(k), \Ignore)\\
  k < 0
}{
  \dynamicdomain(\env, \overname{\TBits(\ve, \Ignore)}{\vt}) = \overname{\emptyset}{\vd}
}
\and
\inferrule[t\_bits\_empty]{
  \evalexprsef{\env, \ve} \evalarrow \ResultExprSEF(\nvint(0), \Ignore)
}{
  \dynamicdomain(\env, \overname{\TBits(\ve, \Ignore)}{\vt}) = \overname{\{ \nvbitvector(\emptylist) \}}{\vd}
}
\and
\inferrule[t\_bits\_non\_empty]{
  \evalexprsef{\env, \ve} \evalarrow \ResultExprSEF(\nvint(k), \Ignore)\\
  k > 0
}{
  \dynamicdomain(\env, \overname{\TBits(\ve, \Ignore)}{\vt}) = \overname{\{ \nvbitvector(\vb_{1..k}) \;|\; \vb_1,\ldots,\vb_k \in \Bit \}}{\vd}
}
\end{mathpar}

\begin{mathpar}
\inferrule[t\_tuple]{
  i=1..k: \dynamicdomain(\env, \vt_i) = D_i
}{
  \dynamicdomain(\env, \overname{\TTuple(\vt_{1..k})}{\vt}) =
  \overname{\{ \NVVector(\vv_{1..k}) \;|\; \vv_i \in D_i \}}{\vd}
}
\end{mathpar}

\begin{mathpar}
\inferrule[t\_array\_abnormal]{
  \evalexprsef{\env, \ve} \evalarrow C\\
  C \in \TDynError \cup \TDiverging
}{
  \dynamicdomain(\env, \overname{\TArray(\ArrayLengthExpr(\ve), \telem)}{\vt}) = \overname{\emptyset}{\vd}
}
\and
\inferrule[t\_array\_negative\_length\_error]{
  \evalexprsef{\env, \ve} \evalarrow \ResultExprSEF(\nvint(k), \Ignore)\\
  k < 0
}{
  \dynamicdomain(\env, \overname{\TArray(\ArrayLengthExpr(\ve), \telem)}{\vt}) = \overname{\emptyset}{\vd}
}
\and
\inferrule[t\_array\_okay]{
  \evalexprsef{\env, \ve} \evalarrow \ResultExprSEF(\nvint(k), \Ignore)\\
  k \geq 0\\
  \dynamicdomain(\env, \telem) = D_\telem
}{
  \dynamicdomain(\env, \overname{\TArray(\ArrayLengthExpr(\ve), \telem)}{\vt}) =
  \overname{\{ \NVVector(\vv_{1..k}) \;|\; \vv_{1..k} \in D_{\telem} \}}{\vd}
}
\end{mathpar}

\begin{mathpar}
\inferrule[t\_enum\_array]{
  \env \eqname (\tenv, \Ignore)\\
  G^\tenv.\declaredtypes(\id) = \TEnum(\vl_{1..k})\\
  \dynamicdomain(\env, \telem) = D_\telem
}{
  {
  \begin{array}{c}
    \dynamicdomain(\env, \overname{\TArray(\ArrayLengthEnum(\id, k), \telem)}{\vt}) =\\
    \overname{\{ \NVRecord(\{i=1..k: \vl_i\mapsto \vv_i\}) \;|\; \vv_i \in D_\telem \}}{\vd}
  \end{array}
  }
}
\end{mathpar}

\begin{mathpar}
\inferrule[structured]{
  L \in \{\TRecord, \TException\}\\
  i=1..k: \dynamicdomain(\env, \vt_i) = D_i
}{
  \dynamicdomain(\env, \overname{L([i=1..k: (\id_i,\vt_i))]}{\vt}) = \\
  \overname{\{ \NVRecord(\{i=1..k: \id_i\mapsto \vv_i\}) \;|\; \vv_i \in D_i \}}{\vd}
}
\end{mathpar}

\begin{mathpar}
\inferrule[t\_named]{
  G^\tenv.\declaredtypes(\id)=\tty
}{
  \dynamicdomain(\env, \overname{\TNamed(\id)}{\vt}) = \overname{\dynamicdomain(\env, \tty)}{\vd}
}
\end{mathpar}

\ExampleDef{Type Domains}
The domain of \texttt{integer} is the infinite set of all integers.

The domain of \verb|integer {2,16}| is the set $\{\nvint(2), \nvint(16)\}$.

The domain of \verb|integer{1..3}| is the set $\{\nvint(1), \nvint(2), \nvint(3)\}$.

The domain of \verb|integer{10..1}| is the empty set as there are no integers that are
both greater than $10$ and smaller than $1$.

The domain of \texttt{bits(2)} is the set $\{\nvbitvector(00)$, $\nvbitvector(01),$
$\nvbitvector(10)$, $\nvbitvector(11)\}$.

The domain of \verb|enumeration {GREEN, ORANGE, RED}| is the set \\
$\{\nvlabel(\texttt{GREEN}), \nvlabel(\texttt{ORANGE}), \nvlabel(\texttt{RED})\}$ and so is the domain
of \\
\verb|type TrafficLights of enumeration {GREEN, ORANGE, RED}|.

The domain of \texttt{(integer, integer)} is the set containing all pairs of native integer values.

The domain of \verb|record {a: integer;  b: boolean}| contains all native records
that map \texttt{a} to a native integer value and \texttt{b} to a native Boolean value.

The dynamic domain of a subprogram parameter \texttt{N: integer} is the (singleton) set containing
the native integer value $c$,
which is assigned to \texttt{N} by a given dynamic environment. The static domain of that parameter
is the infinite set of all native integer values.

\identd{BMGM} \identr{PHRL} \identr{PZNR}
\identr{RLQP} \identr{LYDS} \identr{SVDJ} \identi{WLPJ} \identr{FWMM}
\identi{WPWL} \identi{CDVY} \identi{KFCR} \identi{BBQR} \identr{ZWGH}
\identr{DKGQ} \identr{DHZT} \identi{HSWR} \identd{YZBQ}

%%%%%%%%%%%%%%%%%%%%%%%%%%%%%%%%%%%%%%%%%%%%%%%%%%%%%%%%%%%%%%%%%%%%%%%%%%%%%%%%%%%%
\section{Basic Type Attributes\label{sec:BasicTypeAttributes}}
%%%%%%%%%%%%%%%%%%%%%%%%%%%%%%%%%%%%%%%%%%%%%%%%%%%%%%%%%%%%%%%%%%%%%%%%%%%%%%%%%%%%

This section defines some basic predicates for classifying types as well as
functions that inspect the structure of types:
\begin{itemize}
  \item Builtin singular types (\TypingRuleRef{BuiltinSingularType})
  \item Builtin aggregate types (\TypingRuleRef{BuiltinAggregateType})
  \item Builtin types (\TypingRuleRef{BuiltinSingularOrAggregate})
  \item Named types (\TypingRuleRef{NamedType})
  \item Anonymous types (\TypingRuleRef{AnonymousType})
  \item Singular types (\TypingRuleRef{SingularType})
  \item Aggregate types (\TypingRuleRef{AggregateType})
  \item Structured types (\TypingRuleRef{StructuredType})
  \item Non-primitive types (\TypingRuleRef{NonPrimitiveType})
  \item Primitive types (\TypingRuleRef{PrimitiveType})
  \item The structure of a type (\TypingRuleRef{Structure})
  \item The underlying type of a type (\TypingRuleRef{MakeAnonymous})
  \item Checked constrained integers (\TypingRuleRef{CheckConstrainedInteger})
\end{itemize}

\TypingRuleDef{BuiltinSingularType}
\hypertarget{def-isbuiltinsingular}{}
The predicate
\[
  \isbuiltinsingular(\overname{\ty}{\tty}) \;\aslto\; \Bool
\]
tests whether the type $\tty$ is a \emph{builtin singular type}.

\ProseParagraph
The \emph{builtin singular types} are:
\begin{itemize}
\item the \integertypesterm{};
\item the \realtypeterm{};
\item the \stringtypeterm{};
\item the \booleantypeterm{};
\item the \bitvectortypeterm{} (which includes \texttt{bit}, as a special case);
\item the \enumerationtypeterm{}.
\end{itemize}

\ExampleDef{Builtin singular types}
\listingref{typing-builtinsingulartype} defines variables of builtin singular types
\texttt{integer}, \texttt{real},
\texttt{boolean}, \texttt{bits(4)}, and~\texttt{bits(2)}
\ASLListing{Examples of builtin singular types}{typing-builtinsingulartype}{\typingtests/TypingRule.BuiltinSingularTypes.asl}

\ExampleDef{Builtin enumeration types}
In \listingref{typing-builtinenumerationtype},
the builtin singular type \texttt{Color} consists in two constants:
\texttt{RED} and~\texttt{BLACK}.
\ASLListing{An enumeration type}{typing-builtinenumerationtype}{\typingtests/TypingRule.EnumerationType.asl}

\FormallyParagraph
\begin{mathpar}
\inferrule{
  \vb \eqdef \astlabel(\tty) \in \{\TReal, \TString, \TBool, \TBits, \TEnum, \TInt\}
}{
  \isbuiltinsingular(\tty) \typearrow \vb
}
\end{mathpar}
\CodeSubsection{\BuiltinSingularBegin}{\BuiltinSingularEnd}{../types.ml}


\identd{PQCK} \identd{NZWT}

\TypingRuleDef{BuiltinAggregateType}
\hypertarget{def-isbuiltinaggregate}{}
The predicate
\[
  \isbuiltinaggregate(\overname{\ty}{\tty}) \;\aslto\; \Bool
\]
tests whether the type $\tty$ is a \emph{builtin aggregate type}.

\ProseParagraph
The builtin aggregate types are:
\begin{itemize}
\item tuple;
\item \texttt{array};
\item \texttt{record};
\item \texttt{exception};
\item \texttt{collection}.
\end{itemize}

\ExampleDef{Builtin Aggregate Types}
\listingref{typing-builtinaggregatetypes} provides examples of some builtin aggregate types.
\ASLListing{Builtin aggregate types}{typing-builtinaggregatetypes}{\typingtests/TypingRule.BuiltinAggregateTypes.asl}

Type \texttt{Pair} is the type of integer and boolean pairs.

Arrays are declared with indices that are either integer-typed
or enumeration-typed.  In the example above, \texttt{T} is
declared as an array with an integer-typed index (as indicated
by the used of the integer-typed constant \texttt{3}) whereas
\texttt{PointArray} is declared with the index of
\texttt{Coord}, which is an \enumerationtypeterm{}.

Arrays declared with integer-typed indices can be accessed only by integers ranging from $0$ to
the size of the array minus $1$. In the example above, $\texttt{T}$ can be accessed with
one of $0$, $1$, and $2$.

Arrays declared with an enumeration-typed index can only be accessed with labels from the corresponding
enumeration. In the example above, \texttt{PointArray} can only be accessed with one of the labels
\texttt{CX}, \texttt{CY}, and \texttt{CZ}.

The (builtin aggregate) type \verb|{ x : real, y : real, z : real }| is a record type with three fields
\texttt{x}, \texttt{y} and \texttt{z}.

\subsubsection{Builtin Aggregate Exception Types}
\listingref{typing-builtinexceptiontype} defines two (builtin aggregate) exception types:
\begin{itemize}
\item \verb|exception{}| (for \texttt{Not\_found}), which carries no value; and
\item \verb|exception { message:string }| (for \texttt{SyntaxException}), which carries a message.
\end{itemize}
Notice the similarity with record types and that the empty field list \verb|{}| can be
omitted in type declarations, as is the case for \texttt{Not\_found}.

\ASLListing{Exception types}{typing-builtinexceptiontype}{\typingtests/TypingRule.BuiltinExceptionType.asl}

\FormallyParagraph
\begin{mathpar}
\inferrule{ \vb \eqdef \astlabel(\tty) \in \{\TTuple, \TArray, \TRecord, \TException, \TCollection\} }
{ \isbuiltinaggregate(\tty) \typearrow \vb }
\end{mathpar}
\CodeSubsection{\BuiltinAggregateBegin}{\BuiltinAggregateEnd}{../types.ml}

\identd{PQCK} \identd{KNBD}

\TypingRuleDef{BuiltinSingularOrAggregate}
\hypertarget{def-isbuiltin}{}
The predicate
\[
  \isbuiltin(\overname{\ty}{\tty}) \;\aslto\; \overname{\Bool}{\vb}
\]
tests whether the type $\tty$ is a \emph{builtin type}, yielding the result in $\vb$.

\ExampleDef{Builtin Types}
In the specification
\begin{lstlisting}
type ticks of integer;
\end{lstlisting}
the type \texttt{integer} is a builtin type but the type of \texttt{ticks} is not.

\ProseParagraph
\Proseeqdef{$\vb$}{$\True$ if and only if either $\tty$ is singular or $\tty$ is builtin aggregate}.

\FormallyParagraph
\begin{mathpar}
\inferrule{
  \isbuiltinsingular(\tty) \lor \isbuiltinaggregate(\tty)
}{
  \isbuiltin(\tty) \typearrow \vbone \lor \vbtwo
}
\end{mathpar}
\CodeSubsection{\BuiltinSingularOrAggregateBegin}{\BuiltinSingularOrAggregateEnd}{../types.ml}

\TypingRuleDef{NamedType}
\hypertarget{def-isnamed}{}
The predicate
\[
  \isnamed(\overname{\ty}{\tty}) \;\aslto\; \Bool
\]
tests whether the type $\tty$ is a \emph{named type}.

\Enumerationtypesterm{}, record types, collection types, and exception types
must be declared and associated with a named type.

\ExampleDef{Named Types}
In the specification
\begin{lstlisting}
type ticks of integer;
\end{lstlisting}
\texttt{ticks} is a named type.

\ProseParagraph
A named type is a type that is declared by using the \texttt{type ... of ...} syntax.

\FormallyParagraph
\begin{mathpar}
\inferrule{
  \vb \eqdef \astlabel(\tty) = \TNamed
}{
  \isnamed(\tty) \typearrow \vb
}
\end{mathpar}
\CodeSubsection{\NamedBegin}{\NamedEnd}{../types.ml}
\identd{vmzx}

\TypingRuleDef{AnonymousType}
\hypertarget{def-isanonymous}{}
\identd{VMZX} \identi{SBCK}%
The predicate
\[
  \isanonymous(\overname{\ty}{\tty}) \;\aslto\; \Bool
\]
tests whether the type $\tty$ is an \anonymoustype.

\ExampleDef{Anonymous Types}
The well-typed specification in \listingref{AnonymousType} illustrates the use
of anonymous types as permitted by \TypingRuleRef{TypeSatisfaction}.
\ASLListing{Well-typed anonymous types}{AnonymousType}{\typingtests/TypingRule.AnonymousType.asl}

\ProseParagraph
\Anonymoustypes\ are types that are not declared using the \texttt{type ... of ...} syntax:
\integertypesterm{}, the \realtypeterm{}, the \stringtypeterm{}, the \booleantypeterm{},
\bitvectortypesterm{}, \tupletypesterm{}, and \arraytypesterm{}.

\FormallyParagraph
\begin{mathpar}
\inferrule{
  \vb \eqdef \astlabel(\tty) \neq \TNamed
}{
  \isanonymous(\tty) \typearrow \vb
}
\end{mathpar}
\CodeSubsection{\AnonymousBegin}{\AnonymousEnd}{../types.ml}

\TypingRuleDef{SingularType}
\hypertarget{def-issingular}{}
\identr{GVZK}%
The predicate
\[
  \issingular(\overname{\staticenvs}{\tenv} \aslsep \overname{\ty}{\tty}) \;\aslto\;
  \overname{\Bool}{\vb} \cup \overname{\typeerror}{\TypeErrorConfig}
\]
tests whether the type $\tty$ is a \emph{\singulartypeterm} in the \staticenvironmentterm{} $\tenv$,
yielding the result in $\vb$.
\ProseOtherwiseTypeError

\ExampleDef{Singular types}
In the following example, the types \texttt{A}, \texttt{B}, and \texttt{C} are all singular types:
\begin{lstlisting}
type A of integer;
type B of A;
type C of B;
\end{lstlisting}

\ProseParagraph
\AllApply
\begin{itemize}
  \item obtaining the \underlyingtypeterm\ of $\tty$ in the \staticenvironmentterm{} $\tenv$ yields $\vtone$\ProseOrTypeError;
  \item applying $\isbuiltinsingular$ to $\vtone$ yields $\vb$.
\end{itemize}

\FormallyParagraph
\begin{mathpar}
\inferrule{
  \makeanonymous(\tenv, \tty) \typearrow \vtone \OrTypeError\\\\
  \isbuiltinsingular(\vtone) \typearrow \vb
}{
  \issingular(\tenv, \tty) \typearrow \vb
}
\end{mathpar}
\CodeSubsection{\SingularBegin}{\SingularEnd}{../types.ml}

\TypingRuleDef{AggregateType}
\hypertarget{def-isaggregate}{}
The predicate
\[
  \isaggregate(\overname{\staticenvs}{\tenv} \aslsep \overname{\ty}{\tty}) \;\aslto\;
  \overname{\Bool}{\vb} \cup \overname{\typeerror}{\TypeErrorConfig}
\]
tests whether the type $\tty$ is an \emph{aggregate type} in the \staticenvironmentterm{} $\tenv$,
yielding the result in $\vb$.

\ExampleDef{Aggregate Types}
In the following example, the types \texttt{A}, \texttt{B}, and \texttt{C} are all aggregate types:
\begin{lstlisting}
type A of (integer, integer);
type B of A;
type C of B;
\end{lstlisting}

\ProseParagraph
\AllApply
\begin{itemize}
  \item obtaining the \underlyingtypeterm\ of $\tty$ in the environment $\tenv$ yields $\vtone$\ProseOrTypeError;
  \item $\vtone$ is a builtin aggregate.
\end{itemize}

\FormallyParagraph
\begin{mathpar}
\inferrule{
  \makeanonymous(\tenv, \tty) \typearrow \vtone \OrTypeError\\\\
  \isbuiltinaggregate(\vtone) \typearrow \vb
}{
  \isaggregate(\tenv, \tty) \typearrow \vb
}
\end{mathpar}
\identr{GVZK}
\CodeSubsection{\AggregateBegin}{\AggregateEnd}{../types.ml}

\TypingRuleDef{StructuredType}
\hypertarget{def-isstructured}{}
\hypertarget{def-structuredtype}{}
A \emph{\structuredtypeterm} is any type that consists of a list of field
identifiers that denote individual storage elements.
In ASL there are three such types --- record types, collection types, and
exception types.

The predicate
\[
  \isstructured(\overname{\ty}{\tty}) \;\aslto\; \overname{\Bool}{\vb}
\]
tests whether the type $\tty$ is a \structuredtypeterm\ and yields the result in $\vb$.

\ExampleDef{Structured Types}
In the following example, the types \texttt{SyntaxException} and \texttt{PointRecord}
are each an example of a \structuredtypeterm:
\begin{lstlisting}
type SyntaxException of exception {message: string };
type PointRecord of Record {x : real, y: real, z: real};
\end{lstlisting}

\ProseParagraph
The result $\vb$ is $\True$ if and only if $\tty$ is either a record type, a
collection type or an exception type, which is determined via the AST label of
$\tty$.

\FormallyParagraph
\begin{mathpar}
\inferrule{}{
  \isstructured(\tty) \typearrow \overname{\astlabel(\tty) \in \{\TRecord, \TException, \TCollection\}}{\vb}
}
\end{mathpar}

\identd{WGQS} \identd{QXYC}

\TypingRuleDef{NonPrimitiveType}
\hypertarget{def-isnonprimitive}{}
The predicate
\[
  \isnonprimitive(\overname{\ty}{\tty}) \;\aslto\; \overname{\Bool}{\vb}
\]
tests whether the type $\tty$ is a \emph{non-primitive type}.

\ExampleDef{Non-primitive Types}
The following types are non-primitive:

\begin{tabular}{ll}
\textbf{Type definition} & \textbf{Reason for being non-primitive}\\
\hline
\texttt{type A of integer}  & Named types are non-primitive\\
\texttt{(integer, A)}       & The second component, \texttt{A}, has non-primitive type\\
\texttt{array[6] of A}      & Element type \texttt{A} has a non-primitive type\\
\verb|record { a : A }|     & The field \texttt{a} has a non-primitive type
\end{tabular}

\ProseParagraph
\OneApplies
\begin{itemize}
  \item \AllApplyCase{singular}
  \begin{itemize}
  \item $\tty$ is a builtin singular type;
  \item $\vb$ is $\False$.
  \end{itemize}
  \item \AllApplyCase{named}
  \begin{itemize}
    \item $\tty$ is a named type;
    \item $\vb$ is $\True$.
  \end{itemize}
  \item \AllApplyCase{tuple}
  \begin{itemize}
    \item $\tty$ is a \tupletypeterm{} $\vli$;
    \item $\vb$ is $\True$ if and only if there exists a non-primitive type in $\vli$.
  \end{itemize}
  \item \AllApplyCase{array}
    \begin{itemize}
    \item $\tty$ is an array of type $\tty'$
    \item $\vb$ is $\True$ if and only if $\tty'$ is non-primitive.
    \end{itemize}
  \item \AllApplyCase{structured}
    \begin{itemize}
    \item $\tty$ is a \structuredtypeterm\ with fields $\fields$;
    \item $\vb$ is $\True$ if and only if there exists a non-primitive type in $\fields$.
    \end{itemize}
\end{itemize}

\FormallyParagraph
The cases \textsc{tuple} and \textsc{structured} below, use the notation $\vb_\vt$ to name
Boolean variables by using the types denoted by $\vt$ as a subscript.
\begin{mathpar}
\inferrule[singular]{
  \astlabel(\tty) \in \{\TReal, \TString, \TBool, \TBits, \TEnum, \TInt\}
}{
  \isnonprimitive(\tty) \typearrow \False
}
\end{mathpar}

\begin{mathpar}
\inferrule[named]{\astlabel(\tty) = \TNamed}{\isnonprimitive(\tty) \typearrow \True}
\end{mathpar}

\begin{mathpar}
\inferrule[tuple]{
  \vt \in \tys: \isnonprimitive(\vt) \typearrow \vb_{\vt}\\
  \vb \eqdef \bigvee_{\vt \in \tys} \vb_{\vt}
}{
  \isnonprimitive(\overname{\TTuple(\tys)}{\tty}) \typearrow \vb
}
\end{mathpar}

\begin{mathpar}
\inferrule[array]{
  \isnonprimitive(\tty') \typearrow \vb
}{
  \isnonprimitive(\overname{\TArray(\Ignore, \tty')}{\tty}) \typearrow \vb
}
\end{mathpar}

\begin{mathpar}
\inferrule[structured]{
  L \in \{\TRecord, \TException, \TCollection\}\\
  (\Ignore,\vt) \in \fields : \isnonprimitive(\vt) \typearrow \vb_\vt\\
  \vb \eqdef \bigvee_{\vt \in \vli} \vb_{\vt}
}{
  \isnonprimitive(\overname{L(\fields)}{\tty}) \typearrow \vb
}
\end{mathpar}
\CodeSubsection{\NonPrimitiveBegin}{\NonPrimitiveEnd}{../types.ml}
\identd{GWXK}

\TypingRuleDef{PrimitiveType}
\hypertarget{def-isprimitive}{}
The predicate
\[
  \isprimitive(\overname{\ty}{\tty}) \;\aslto\; \Bool
\]
tests whether the type $\tty$ is a \emph{primitive type}.

\ExampleDef{Primitive Types}
The following types are primitive:

\begin{tabular}{ll}
\textbf{Type definition} & \textbf{Reason for being primitive}\\
\hline
\texttt{integer} & Integers are primitive\\
\texttt{(integer, integer)} & All tuple elements are primitive\\
\texttt{array[5] of integer} & The array element type is primitive\\
\verb|record {ticks : integer}| & The single field \texttt{ticks} has a primitive type
\end{tabular}

\ProseParagraph
A type $\tty$ is primitive if it is not non-primitive.

\FormallyParagraph
\begin{mathpar}
\inferrule{
  \isnonprimitive(\tty) \typearrow \vb
}{
  \isprimitive(\tty) \typearrow \neg\vb
}
\end{mathpar}
\CodeSubsection{\PrimitiveBegin}{\PrimitiveEnd}{../types.ml}
\identd{GWXK}

\TypingRuleDef{Structure}
\hypertarget{def-structure}{}
The function
\[
  \tstruct(\overname{\staticenvs}{\tenv} \aslsep \overname{\ty}{\tty}) \aslto \overname{\ty}{\vt} \cup \overname{\typeerror}{\TypeErrorConfig}
\]
assigns a type to its \hypertarget{def-tstruct}{\emph{\structureterm}}, which is the type formed by
recursively replacing named types by their type definition in the \staticenvironmentterm{} $\tenv$.
If a named type is not associated with a declared type in $\tenv$, a \typingerrorterm{} is returned.

\TypingRuleRef{TypeCheckAST} ensures the absence of circular type definitions,
which ensures that \TypingRuleRef{Structure} terminates\footnote{In mathematical terms,
this ensures that \TypingRuleRef{Structure} is a proper \emph{structural induction.}}.

\ExampleDef{The Structure of a Type}
\listingref{Structure} shows examples of types and their structures.
\ASLListing{Types and their structure}{Structure}{\typingtests/TypingRule.Structure.asl}

\ProseParagraph
\OneApplies
\begin{itemize}
\item \AllApplyCase{named}
  \begin{itemize}
  \item $\tty$ is a named type $\vx$;
  \item obtaining the declared type associated with $\vx$ in the \staticenvironmentterm{} $\tenv$ yields $\vtone$\ProseOrTypeError;
  \item obtaining the structure of $\vtone$ \staticenvironmentterm{} $\tenv$ yields $\vt$\ProseOrTypeError;
  \end{itemize}
\item \AllApplyCase{builtin\_singular}
  \begin{itemize}
  \item $\tty$ is a builtin singular type;
  \item $\vt$ is $\tty$.
  \end{itemize}
\item \AllApplyCase{tuple}
  \begin{itemize}
  \item $\tty$ is a \tupletypeterm{} with list of types $\tys$;
  \item the types in $\tys$ are indexed as $\vt_i$, for $i=1..k$;
  \item obtaining the structure of each type $\vt_i$, for $i=1..k$, in $\tys$ in the \staticenvironmentterm{} $\tenv$,
  yields $\vtp_i$\ProseOrTypeError;
  \item $\vt$ is a \tupletypeterm{} with the list of types $\vtp_i$, for $i=1..k$.
  \end{itemize}
\item \AllApplyCase{array}
  \begin{itemize}
    \item $\tty$ is an array type of length $\ve$ with element type $\vt$;
    \item obtaining the structure of $\vt$ yields $\vtone$\ProseOrTypeError;
    \item $\vt$ is an array type with of length $\ve$ with element type $\vtone$.
  \end{itemize}
\item \AllApplyCase{structured}
  \begin{itemize}
  \item $\tty$ is a \structuredtypeterm\ with fields $\fields$;
  \item obtaining the structure for each type $\vt$ associated with field $\id$ yields a type $\vt_\id$\ProseOrTypeError;
  \item $\vt$ is a record, a collection or an exception, in correspondence to $\tty$, with the list of pairs $(\id, \vt\_\id)$;
  \end{itemize}
\end{itemize}

\FormallyParagraph
\begin{mathpar}
\inferrule[named]{
  \declaredtype(\tenv, \vx) \typearrow \vtone \OrTypeError\\\\
  \tstruct(\tenv, \vtone)\typearrow\vt \OrTypeError
}{
  \tstruct(\tenv, \overname{\TNamed(\vx)}{\tty}) \typearrow \vt
}
\end{mathpar}

\begin{mathpar}
\inferrule[builtin\_singular]{
  \isbuiltinsingular(\tty) \typearrow \True
}{
  \tstruct(\tenv, \tty) \typearrow \tty
}
\end{mathpar}

\begin{mathpar}
\inferrule[tuple]{
  \tys \eqname \vt_{1..k}\\
  i=1..k: \tstruct(\tenv, \vt_i) \typearrow \vtp_i \OrTypeError
}{
  \tstruct(\tenv, \overname{\TTuple(\tys)}{\tty}) \typearrow  \TTuple(i=1..k: \vtp_i)
}
\end{mathpar}

\begin{mathpar}
\inferrule[array]{
  \tstruct(\tenv, \vt) \typearrow \vtone \OrTypeError
}{
  \tstruct(\tenv, \overname{\TArray(\ve, \vt)}{\tty}) \typearrow \TArray(\ve, \vtone)
}
\end{mathpar}

\begin{mathpar}
\inferrule[structured]{
  L \in \{\TRecord, \TException, \TCollection\}\\\\
  (\id,\vt) \in \fields : \tstruct(\tenv, \vt) \typearrow \vt_\id \OrTypeError
}{
  \tstruct(\tenv, \overname{L(\fields)}{\tty}) \typearrow
 L([ (\id,\vt) \in \fields : (\id,\vt_\id) ])
}
\end{mathpar}
\CodeSubsection{\StructureBegin}{\StructureEnd}{../types.ml}
\identd{FXQV}

\TypingRuleDef{MakeAnonymous}
\hypertarget{def-makeanonymous}{}
\hypertarget{def-underlyingtype}{}
The function
\[
  \makeanonymous(\overname{\staticenvs}{\tenv} \aslsep \overname{\ty}{\tty}) \aslto \overname{\ty}{\vt} \cup \overname{\typeerror}{\TypeErrorConfig}
\]
returns the \emph{\underlyingtypeterm} --- $\vt$ --- of the type $\tty$ in the \staticenvironmentterm{} $\tenv$ or a \typingerrorterm{}.
Intuitively, $\tty$ is the first non-named type that is used to define $\tty$. Unlike $\tstruct$,
$\makeanonymous$ replaces named types by their definition until the first non-named type is found but
does not recurse further.

\ExampleDef{The Underlying Type of a Type}
Consider the following example:
\begin{lstlisting}
type T1 of integer;
type T2 of T1;
type T3 of (integer, T2);
\end{lstlisting}

The \underlyingtypesterm\ of \texttt{integer}, \texttt{T1}, and \texttt{T2} is \texttt{integer}.

The \underlyingtypeterm{} of \texttt{(integer, T2)} and \texttt{T3} is
\texttt{(integer, T2)}.  Notice how the \underlyingtypeterm{} does not replace
\texttt{T2} with its own \underlyingtypeterm, in contrast to the \structureterm{} of
\texttt{T2}, which is \texttt{(integer, integer)}.

\ProseParagraph
\OneApplies
\begin{itemize}
  \item \AllApplyCase{named}
  \begin{itemize}
    \item $\tty$ is a named type $\vx$;
    \item obtaining the type declared for $\vx$ yields $\vtone$\ProseOrTypeError;
    \item the \underlyingtypeterm\ of $\vtone$ is $\vt$.
  \end{itemize}

  \item \AllApplyCase{non-named}
  \begin{itemize}
    \item $\tty$ is not a named type $\vx$;
    \item $\vt$ is $\tty$.
  \end{itemize}
\end{itemize}

\FormallyParagraph
\begin{mathpar}
\inferrule[named]{
  \tty \eqname \TNamed(\vx) \\
  \declaredtype(\tenv, \vx) \typearrow \vtone \OrTypeError \\\\
  \makeanonymous(\tenv, \vtone) \typearrow \vt
}{
  \makeanonymous(\tenv, \tty) \typearrow \vt
}
\and
\inferrule[non-named]{
  \astlabel(\tty) \neq \TNamed
}{
  \makeanonymous(\tenv, \tty) \typearrow \tty
}
\end{mathpar}
\CodeSubsection{\MakeAnonymousBegin}{\MakeAnonymousEnd}{../types.ml}

\TypingRuleDef{CheckConstrainedInteger}
A type is a \emph{\constrainedintegerterm} if it is either a \wellconstrainedintegertypeterm{}
or a \parameterizedintegertypeterm.

\hypertarget{def-checkconstrainedinteger}{}
The function
\[
  \checkconstrainedinteger(\overname{\staticenvs}{\tenv} \aslsep \overname{\ty}{\tty}) \aslto \{\True\} \cup \overname{\typeerror}{\TypeErrorConfig}
\]
checks whether the type $\vt$ is a \constrainedintegerterm{} type.
If so, the result is $\True$, otherwise the result is a \typingerrorterm.

\ExampleDef{Checking for Constrained Integers}
\listingref{check-constrained-integer}
shows examples of checking whether a type (used for the width of a bitvector type)
is a \constrainedintegerterm{} type.
\ASLListing{Checking for constrained integers}{check-constrained-integer}{\typingtests/TypingRule.CheckConstrainedInteger.asl}

\ProseParagraph
\OneApplies
\begin{itemize}
  \item \AllApplyCase{well-constrained}
  \begin{itemize}
    \item $\vt$ is a well-constrained integer;
    \item the result is $\True$.
  \end{itemize}

  \item \AllApplyCase{parameterized}
  \begin{itemize}
    \item $\vt$ is a \parameterizedintegertypeterm;
    \item the result is $\True$.
  \end{itemize}

  \item \AllApplyCase{unconstrained}
  \begin{itemize}
    \item $\vt$ is an unconstrained integer or pending constrained integer;
    \item the result is a \typingerrorterm{} indicating that a constrained integer type is expected.
  \end{itemize}

  \item \AllApplyCase{conflicting\_type}
  \begin{itemize}
    \item $\vt$ is not an integer type;
    \item the result is a \typingerrorterm{} indicating the type conflict.
  \end{itemize}
\end{itemize}

\FormallyParagraph
\begin{mathpar}
\inferrule[well-constrained]{}
{
  \checkconstrainedinteger(\tenv, \overname{\TInt(\WellConstrained(\Ignore))}{\tty}) \typearrow \True
}
\and
\inferrule[parameterized]{}
{
  \checkconstrainedinteger(\tenv, \overname{\TInt(\Parameterized(\Ignore))}{\tty}) \typearrow \True
}
\and
\inferrule[unconstrained]{
  \astlabel(\vc) = \Unconstrained \;\lor\; \astlabel(\vc) = \PendingConstrained
}{
  \checkconstrainedinteger(\tenv, \overname{\TInt(\vc)}{\tty}) \typearrow \TypeErrorVal{\UnexpectedType}
}
\and
\inferrule[conflicting\_type]{
  \astlabel(\vt) \neq \TInt
}{
  \checkconstrainedinteger(\tenv, \vt) \typearrow \TypeErrorVal{\UnexpectedType}
}
\end{mathpar}
\CodeSubsection{\CheckConstrainedIntegerBegin}{\CheckConstrainedIntegerEnd}{../Typing.ml}

\section{Relations Over Types\label{sec:RelationsOnTypes}}

This section defines the following relations over types and operators:
\begin{itemize}
  \item Subtype (\TypingRuleRef{Subtype})
  \item Subtype Satisfaction (\TypingRuleRef{SubtypeSatisfaction})
  \item Type Satisfaction (\TypingRuleRef{TypeSatisfaction})
  \item The Lowest Common Ancestor of two types (\TypingRuleRef{LowestCommonAncestor})
  \item Applying a unary operator to a type (\TypingRuleRef{ApplyUnopType})
  \item Applying a binary operator to a pair of types (\TypingRuleRef{ApplyBinopTypes})
\end{itemize}

\TypingRuleDef{Subtype}
\hypertarget{def-supertypeterm}{}
The \emph{\subtypeterm} relation is a partial order over \underline{named types}.
The \emph{\supertypeterm} is the inverse relation.
That is, \tty\ is a \supertypeterm{} of \tsy\ if and only if \tsy\ is a \subtypeterm{} of \tty.

\ExampleDef{Subtypes and Supertypes}
The following table determines whether the type \vAbf{}
subtypes the type \vBbf{} with respect to the types
declared in \listingref{subtype}:\\
\begin{tabular}{llll}
  \textbf{type A} & \textbf{type B}   & \textbf{subtypes?}  & \textbf{reason}\\
\hline
  \texttt{subInt}     & \texttt{subInt}   & yes             & subtyping is reflexive for \namedtypesterm{}\\
  \texttt{subInt}     & \texttt{superInt} & yes             & declared as a \subtypeterm{}\\
  \texttt{superInt}   & \texttt{subInt}   & no              & subtyping is anti-symmetric\\
  \texttt{subsubInt}  & \texttt{superInt} & yes             & subtyping is transitive\\
  \texttt{otherInt}   & \texttt{superInt} & no              & no chain of subtyping between the types\\
  \texttt{superInt}   & \texttt{integer}  & no              & \texttt{integer} is not a \namedtypeterm{}\\
  \texttt{integer}    & \texttt{integer}  & no              & \texttt{integer} is not a \namedtypeterm{}\\
\end{tabular}

\ASLListing{Subtypes and Supertypes}{subtype}{\typingtests/TypingRule.Subtype.asl}

\hypertarget{def-subtypesrel}{}
The predicate
\[
  \subtypesrel(\overname{\staticenvs}{\tenv} \aslsep \overname{\ty}{\vtone} \aslsep \overname{\ty}{\vttwo})
  \aslto \overname{\Bool}{\vb}
\]
defines whether the type $\vtone$ subtypes the type $\vttwo$ in the \staticenvironmentterm{} $\tenv$,
yielding the result in $\vb$.

\ProseParagraph
\OneApplies
\begin{itemize}
  \item \AllApplyCase{reflexive}
  \begin{itemize}
    \item $\vtone$ and $\vttwo$ are both the same named type;
    \item $\vb$ is $\True$.
  \end{itemize}

  \item \AllApplyCase{transitive}
  \begin{itemize}
    \item $\vtone$ is a named type with name $\idone$, that is $\TNamed(\idone)$;
    \item $\vttwo$ is a named type with name $\idtwo$, that is $\TNamed(\idtwo)$, such that $\idone$ is different from $\idtwo$;
    \item the \globalstaticenvironmentterm{} maintains that $\idone$ is a subtype of $\idthree$;
    \item testing whether the type named $\idthree$ is a subtype of $\vttwo$ in the \staticenvironmentterm{} $\tenv$
    gives $\vb$.
  \end{itemize}

  \item \AllApplyCase{no\_supertype}
  \begin{itemize}
    \item $\vtone$ is a named type with name $\idone$, that is $\TNamed(\idone)$;
    \item $\vttwo$ is a named type with name $\idtwo$, that is $\TNamed(\idtwo)$, such that $\idone$ is different from $\idtwo$;
    \item the \globalstaticenvironmentterm{} maintains that $\idone$ does subtype any named type;
    \item $\vb$ is $\False$.
  \end{itemize}

  \item \AllApplyCase{not\_named}
  \begin{itemize}
    \item at least one of $\vtone$ and $\vttwo$ is not a named type;
    \item $\vb$ is $\False$.
  \end{itemize}
\end{itemize}

\FormallyParagraph
\begin{mathpar}
\inferrule[reflexive]{}{
  \subtypesrel(\tenv, \TNamed(\id), \TNamed(\id)) \typearrow \True
}
\and
\inferrule[transitive]{
  \idone \neq \idtwo\\
  G^\tenv.\subtypes(\idone) = \idthree\\
  \subtypesrel(\tenv, \TNamed(\idthree), \vttwo) \typearrow \vb
}{
  \subtypesrel(\tenv, \TNamed(\idone), \TNamed(\idtwo)) \typearrow \vb
}
\and
\inferrule[no\_supertype]{
  \idone \neq \idtwo\\
  G^\tenv.\subtypes(\idone) = \bot
}{
  \subtypesrel(\tenv, \TNamed(\idone), \TNamed(\idtwo)) \typearrow \False
}
\and
\inferrule[not\_named]{
  (\astlabel(\vtone) \neq \TNamed \lor \astlabel(\vttwo) \neq \TNamed)
}{
  \subtypesrel(\tenv, \vtone, \vttwo) \typearrow \False
}
\end{mathpar}
\CodeSubsection{\SubtypeBegin}{\SubtypeEnd}{../types.ml}

\identr{NXRX} \identi{KGKS} \identi{MTML} \identi{JVRM} \identi{CHMP}

\TypingRuleDef{SubtypeSatisfaction}
\hypertarget{def-subtypesat}{}
The predicate
\[
  \subtypesat(\overname{\staticenvs}{\tenv} \aslsep \overname{\ty}{\vt} \aslsep \overname{\ty}{\vs})
  \aslto \overname{\Bool}{\vb} \cup \overname{\typeerror}{\TypeErrorConfig}
\]
determines whether a type $\vt$ \emph{subtype-satisfies} a type $\vs$ in environment $\tenv$,
returning the result in $\vb$.
\ProseOtherwiseTypeError

The function assumes that both $\vt$ and $\vs$ are well-typed according to \chapref{Types}.

\ExampleDef{Subtype Satisfaction}
\listingref{subtypesat1} shows examples
where the types of the \rhsexpressions{} \subtypesatisfy{}
the types of the left-hand-side expressions.
\pagebreak
\ASLListing{Subtype Satisfaction}{subtypesat1}{\typingtests/TypingRule.SubtypeSatisfaction1.asl}

\listingref{subtypesat-bad1} and \listingref{subtypesat-bad2}
shows examples of nuanced \typingerrorsterm.
Specifically where a type consisting of a range of values does not \subtypesatisfy{}
a type consisting of one variable expression.
\ASLListing{Subtype satisfaction error 1}{subtypesat-bad1}{\typingtests/TypingRule.SubtypeSatisfaction.bad1.asl}
\ASLListing{Subtype satisfaction error 2}{subtypesat-bad2}{\typingtests/TypingRule.SubtypeSatisfaction.bad2.asl}

\listingref{subtypesat2} shows examples of legal and illegal assignments involving
subtyping.
\ASLListing{More Examples of subtype satisfaction}{subtypesat2}{\typingtests/TypingRule.SubtypeSatisfaction2.asl}

\listingref{subtypesat3} shows more examples of legal and illegal assignments involving
subtyping.
\ASLListing{Even more examples of subtype satisfaction}{subtypesat3}{\typingtests/TypingRule.SubtypeSatisfaction3.asl}

\ProseParagraph
\OneApplies
\begin{itemize}
\item \AllApplyCase{error1}
  \begin{itemize}
  \item obtaining the \underlyingtype\ of $\vt$ gives a \typingerrorterm{};
  \item the rule results in a \typingerrorterm{}.
  \end{itemize}

\item \AllApplyCase{error2}
  \begin{itemize}
    \item obtaining the \underlyingtype\ of $\vt$ gives a type $\vttwo$;
    \item obtaining the \underlyingtype\ of $\vs$ gives a \typingerrorterm{};
    \item the rule results in a \typingerrorterm{}.
    \end{itemize}

\item \AllApplyCase{different\_labels}
  \begin{itemize}
  \item the underlying types of $\vt$ and $\vs$ have different AST labels
  (for example, $\TInt$ and $\TReal$);
  \item $\vb$ is $\False$.
  \end{itemize}

\item \AllApplyCase{simple}
  \begin{itemize}
  \item the \underlyingtype\ of $\vt$, $\vttwo$, is either \realtypeterm{}, \stringtypeterm{}, or \booleantypeterm{};
  \item the \underlyingtype\ of $\vs$, $\vstwo$, is either \realtypeterm{}, \stringtypeterm{}, or \booleantypeterm{};
  \item $\vb$ is $\True$ if and only if both $\vttwo$ and $\vstwo$ have the same ASL label.
  \end{itemize}

\item \AllApplyCase{t\_int}
  \begin{itemize}
  \item the \underlyingtype\ of $\vt$, $\vttwo$, is an \integertypeterm{} (any kind);
  \item the \underlyingtype\ of $\vs$, $\vstwo$, is an \integertypeterm{} (any kind);
  \item applying $\symdomoftype$ to $\tenv$ and $\vs$ yields the \symbolicdomainterm{} $\ds$;
  \item applying $\symdomoftype$ to $\tenv$ and $\vt$ yields the \symbolicdomainterm{} $\dt$;
  \item applying $\symdomsubsetunions$ to $\tenv$, $\ds$, and $\dt$ yields $\vb$.
  \end{itemize}

\item \AllApplyCase{t\_enum}
  \begin{itemize}
  \item the \underlyingtype\ of $\vt$ is an \enumerationtypeterm{} with list of labels $\vlit$, that is, $\TEnum(\vlit)$;
  \item the \underlyingtype\ of $\vs$ is an \enumerationtypeterm{} with list of labels $\vlis$, that is, $\TEnum(\vlis)$;
  \item $\vb$ is $\True$ if and only if $\vlit$ is equal to $\vlis$.
  \end{itemize}

\item \AllApplyCase{t\_bits}
  \begin{itemize}
  \item the \underlyingtype\ of $\vs$ is a bitvector type with width $\ws$ and bit fields $\bfss$, that is $\TBits(\ws, \bfss)$;
  \item the \underlyingtype\ of $\vt$ is a bitvector type with width $\wt$ and bit fields $\bfst$, that is $\TBits(\wt, \bfst)$;
  \item determining whether the bitfields $\bfss$ are included in the bitfields $\bfst$ in $\tenv$ yields $\True$\ProseOrTypeError;
  \item determining whether the \symbolicdomainterm{} of $\ws$ subsumes the \symbolicdomainterm{} of $\wt$ in $\tenv$ yields $\vb$.
  \end{itemize}

\item \AllApplyCase{t\_array\_expr}
  \begin{itemize}
  \item $\vs$ has the \underlyingtype\ of an array with index $\vlengths$ and element type $\vtys$, that is $\TArray(\vlengths, \vtys)$;
  \item $\vt$ has the \underlyingtype\ of an array with index $\vlengtht$ and element type $\vtyt$, that is $\TArray(\vlengtht, \vtyt)$;
  \item determining whether $\vtys$ and $\vtyt$ are \equivalenttypesterm{} in $\tenv$ is either $\True$
  or $\False$, which short-circuits the entire rule with $\vb=\False$;
  \item either the AST labels of $\vlengths$ and $\vlengtht$ are the same or the rule short-circuits with $\vb=\False$;
  \item $\vlengths$ is an array length expression with $\vlengthexprs$, that is \\ $\ArrayLengthExpr(\vlengthexprs)$;
  \item $\vlengtht$ is an array length expression with $\vlengthexprt$, that is \\ $\ArrayLengthExpr(\vlengthexprt)$;
  \item determining whether $\vlengthexprs$ and $\vlengthexprt$ are \equivalentexprsterm{} gives $\vb$.
  \end{itemize}

  \item \AllApplyCase{t\_array\_enum}
  \begin{itemize}
  \item $\vs$ has the \underlyingtype\ of an array with index $\vlengths$ and element type $\vtys$, that is $\TArray(\vlengths, \vtys)$;
  \item $\vt$ has the \underlyingtype\ of an array with index $\vlengtht$ and element type $\vtyt$, that is $\TArray(\vlengtht, \vtyt)$;
  \item determining whether $\vtys$ and $\vtyt$ are \equivalenttypesterm{} in $\tenv$ is either $\True$
  or $\False$, which short-circuits the entire rule with $\vb=\False$;
  \item either the AST labels of $\vlengths$ and $\vlengtht$ are the same or the rule short-circuits with $\vb=\False$;
  \item $\vlengths$ is an array with indices taken from the enumeration $\vnames$, that is $\ArrayLengthEnum(\vnames, \Ignore)$;
  \item $\vlengtht$ is an array with indices taken from the enumeration $\vnamet$, that is $\ArrayLengthEnum(\vnamet, \Ignore)$;
  \item $\vb$ is $\True$ if and only if $\vnames$ and $\vnamet$ are the same.
  \end{itemize}

\item \AllApplyCase{t\_tuple}
  \begin{itemize}
  \item $\vs$ has the \underlyingtype\ of a tuple with type list $\vlis$, that is $\TTuple(\vlis)$;
  \item $\vt$ has the \underlyingtype\ of a tuple with type list $\vlit$, that is $\TTuple(\vlit)$;
  \item equating the lengths of $\vlis$ and $\vlit$ is either $\True$ or $\False$, which short-circuits
  the entire rule with $\vb=\False$;
  \item checking at each index $\vi$ of the list $\vlis$ whether the type $\vlit[\vi]$ \typesatisfies\ the type $\vlis[\vi]$
  yields $\vb_\vi$\ProseOrTypeError;
  \item $\vb$ is $\True$ if and only if all $\vb_\vi$ are $\True$;
  \end{itemize}

\item \AllApplyCase{structured}
  \begin{itemize}
  \item $\vs$ has the \underlyingtype\ $L(\vfieldss)$, which is a \structuredtype;
  \item $\vt$ has the \underlyingtype\ $L(\vfieldst)$, which is a \structuredtype;
  \item since both underlying types have the same AST label they are either both record types or both exception types or both collection types;
  \item $\vb$ is $\True$ if and only if for each field in $\vfieldss$ with type $\vtys$
  there exists a field in $\vfieldst$ with type $\vtyt$ such that both $\vtys$ and $\vtyt$
  are determined to be \typeequivalent\ in $\tenv$.
  \end{itemize}
\end{itemize}

\FormallyParagraph
\begin{mathpar}
\inferrule[error1]{
  \makeanonymous(\tenv, \vt) \typearrow \TypeErrorConfig
}{
  \subtypesat(\tenv, \vt, \vs) \typearrow \TypeErrorConfig
}
\end{mathpar}

\begin{mathpar}
\inferrule[error2]{
  \makeanonymous(\tenv, \vt) \typearrow \vttwo\\
  \makeanonymous(\tenv, \vs) \typearrow \TypeErrorConfig
}{
  \subtypesat(\tenv, \vt, \vs) \typearrow \TypeErrorConfig
}
\end{mathpar}

\begin{mathpar}
\inferrule[different\_labels]{
  \makeanonymous(\tenv, \vt) \typearrow \vttwo\\
  \makeanonymous(\tenv, \vs) \typearrow \vstwo\\
  \astlabel(\vttwo) \neq \astlabel(\vstwo)
}{
  \subtypesat(\tenv, \vt, \vs) \typearrow \False
}
\end{mathpar}

\begin{mathpar}
  \inferrule[simple]{
    \makeanonymous(\tenv, \vt) \typearrow \vttwo\\
    \makeanonymous(\tenv, \vs) \typearrow \vstwo\\
    \astlabel(\vttwo) \in \{\TReal, \TString, \TBool\}\\
    \vb \eqdef \astlabel(\vstwo) = \astlabel(\vttwo)
  }{
    \subtypesat(\tenv, \vt, \vs) \typearrow \vb
  }
\end{mathpar}

\begin{mathpar}
\inferrule[t\_int]{
  \makeanonymous(\tenv, \vt) \typearrow \vttwo\\
  \makeanonymous(\tenv, \vs) \typearrow \vstwo\\
  \astlabel(\vttwo) = \astlabel(\vstwo) = \TInt\\
  \symdomoftype(\tenv, \vs) \typearrow \ds \\
  \symdomoftype(\tenv, \vt) \typearrow \dt \\
  \symdomsubsetunions(\tenv, \ds, \dt) \typearrow \vb
}{
  \subtypesat(\tenv, \vt, \vs) \typearrow \vb
}
\end{mathpar}

\begin{mathpar}
\inferrule[t\_enum]{
  \makeanonymous(\tenv, \vt) \typearrow \TEnum(\vlit)\\
  \makeanonymous(\tenv, \vs) \typearrow \TEnum(\vlis)\\
  \vb \eqdef \vlit = \vlis
}{
  \subtypesat(\tenv, \vt, \vs) \typearrow \vb
}
\end{mathpar}

\begin{mathpar}
\inferrule[t\_bits]{
  \makeanonymous(\tenv, \vs) \typearrow \TBits(\ws, \bfss)\\
  \makeanonymous(\tenv, \vt) \typearrow \TBits(\wt, \bfst)\\
  \bitfieldsincluded(\tenv, \bfss, \bfst) \typearrow \True \OrTypeError \\
  \symdomofwidthexpr(\tenv, \ws) \typearrow \ds \\
  \symdomofwidthexpr(\tenv, \wt) \typearrow \dt \\
  \symdomsubsetunions(\tenv, \ds, \dt) \typearrow \vb
}{
  \subtypesat(\tenv, \vt, \vs) \typearrow \vb
}
\end{mathpar}

\begin{mathpar}
\inferrule[t\_array\_expr]{
  \makeanonymous(\tenv, \vs) \typearrow \TArray(\vlengths,\vtys) \\
  \makeanonymous(\tenv, \vt) \typearrow \TArray(\vlengtht,\vtyt) \\
  \typeequal(\tenv, \vtys, \vtyt) \typearrow \True \terminateas \False\\
  \booltrans(\astlabel(\vlengths) = \astlabel(\vlengtht)) \booltransarrow \True \terminateas \False\\
  \vlengths \eqname \ArrayLengthExpr(\vlengthexprs)\\
  \vlengtht \eqname \ArrayLengthExpr(\vlengthexprt)\\
  \exprequal(\tenv, \vlengthexprs, \vlengthexprt) \typearrow \vb
}{
  \subtypesat(\tenv, \vt, \vs) \typearrow \vb
}
\end{mathpar}

\begin{mathpar}
\inferrule[t\_array\_enum]{
  \makeanonymous(\tenv, \vs) \typearrow \TArray(\vlengths,\vtys) \\
  \makeanonymous(\tenv, \vt) \typearrow \TArray(\vlengtht,\vtyt) \\
  \typeequal(\tenv, \vtys, \vtyt) \typearrow \True\\
  \booltrans(\astlabel(\vlengths) = \astlabel(\vlengtht)) \typearrow \True \terminateas \False \\
  \vlengths \eqname \ArrayLengthEnum(\vnames, \Ignore)\\
  \vlengtht \eqname \ArrayLengthEnum(\vnamet, \Ignore)\\
  \vb \eqdef \vnames = \vnamet
}{
  \subtypesat(\tenv, \vt, \vs) \typearrow \vb
}
\end{mathpar}

\begin{mathpar}
\inferrule[t\_tuple]
{ \makeanonymous(\tenv, \vs) \typearrow\TTuple(\vlis)\\
  \makeanonymous(\tenv, \vt) \typearrow\TTuple(\vlit)\\
  \equallength(\vlis, \vlit) \typearrow\True \terminateas \False\\
  \vi\in\listrange(\vlis): \typesat(\tenv, \vlit[\vi], \vlis[\vi]) \typearrow \vb_i \terminateas \typeerror\\
  \vb \eqdef \bigwedge_{\vi=1}^k \vb_\vi
}{
  \subtypesat(\tenv, \vt, \vs) \typearrow \vb
}
\end{mathpar}

\hypertarget{def-fieldnames}{}
For a list of typed fields $\fields$, we define the set of its field identifiers as:
\[
  \fieldnames(\fields) \triangleq \{ \id \;|\; (\id, \vt) \in \fields\}
\]
\hypertarget{def-fieldtype}{}
We define the type associated with the field name $\id$ in a list of typed fields $\fields$,
if there is a unique one, as follows:
\[
  \fieldtype(\fields, \id) \triangleq
  \begin{cases}
  \vt  & \text{ if }\{ \vtp \;|\; (\id,\vtp) \in \fields\} = \{\vt\}\\
  \bot & \text{ otherwise}
  \end{cases}
\]

\begin{mathpar}
\inferrule[structured]{
  L \in \{\TRecord, \TException, \TCollection\}\\
  \makeanonymous(\tenv, \vs)\typearrow L(\vfieldss) \\
  \makeanonymous(\tenv, \vt)\typearrow L(\vfieldst) \\
  \vnamess \eqdef \fieldnames(\vfieldss)\\
  \vnamest \eqdef \fieldnames(\vfieldst)\\
  \booltrans(\vnamess \subseteq \vnamest) \booltransarrow \True \terminateas \False\\
  (\id,\vtys)\in\vfieldss: \typeequal(\tenv, \vtys, \fieldtype(\vfieldst, \id)) \typearrow \vb_\id\\
  \vb \eqdef \bigwedge_{\id \in \vnamess} \vb_\id
}{
  \subtypesat(\tenv, \vs, \vt) \typearrow \vb
}
\end{mathpar}
\identd{TRVR} \identi{SJDC} \identi{MHYB} \identi{TWTZ} \identi{GYSK} \identi{KXSD} \identi{KNXJ}
\CodeSubsection{\SubtypeSatisfactionBegin}{\SubtypeSatisfactionEnd}{../types.ml}

\TypingRuleDef{TypeSatisfaction}
\identr{FMXK}
\hypertarget{def-typesatisfies}{}
The predicate
\[
  \typesat(\overname{\staticenvs}{\tenv} \aslsep \overname{\ty}{\vt} \aslsep \overname{\ty}{\vs})
  \aslto \overname{\Bool}{\vb} \cup \overname{\typeerror}{\TypeErrorConfig}
\]
determines whether a type $\vt$ \emph{\typesatisfies} a type $\vs$ in environment $\tenv$,
returning the result $\vb$.
\ProseOtherwiseTypeError

The function assumes that both $\vt$ and $\vs$ are well-typed according to \secref{Types}.

\ExampleDef{Type-satisfaction Examples}
In \listingref{typing-typesat1},
\texttt{var pair: pairT = (1, dataT1)} is legal since the right-hand-side has
anonymous, non-primitive type \texttt{(integer, T1)}.
\ASLListing{Type satisfaction example}{typing-typesat1}{\typingtests/TypingRule.TypeSatisfaction1.asl}

\ExampleDef{More Type-satisfaction Examples}
In \listingref{typing-typesat2},
\texttt{pair = (1, dataAsInt);} is legal since the right-hand-side has anonymous,
primitive type \texttt{(integer, integer)}.
\ASLListing{Type satisfaction example}{typing-typesat2}{\typingtests/TypingRule.TypeSatisfaction2.asl}

\ExampleDef{Failing Type-satisfaction}
In \listingref{typing-typesat3},
\texttt{pair = (1, dataT2);} is illegal since the right-hand-side has anonymous,
non-primitive type \texttt{(integer, T2)} which does not subtype-satisfy named
type \texttt{pairT}.
\ASLListing{Type satisfaction example}{typing-typesat3}{\typingtests/TypingRule.TypeSatisfaction3.asl}

The specification in \listingref{typing-typesat-bad1} is ill-typed,
since \verb|integer{0..N}| does not \typesatisfy{} \verb|integer{0..M}|.
\ASLListing{Failing type satisfaction example}{typing-typesat-bad1}{\typingtests/TypingRule.TypeSatisfaction.bad1.asl}

\ProseParagraph
\OneApplies
 \begin{itemize}
  \item \AllApplyCase{subtypes}
    \begin{itemize}
    \item $\vt$ subtypes $\vs$ in $\tenv$ ;
    \item $\vb$ is $\True$.
  \end{itemize}

  \item \AllApplyCase{anonymous}
  \begin{itemize}
    \item $\vt$ does not subtype $\vs$ in $\tenv$;
    \item at least one of $\vt$ and $\vs$ is an anonymous type in $\tenv$;
    \item determining whether $\vt$ \subtypesatisfies\ $\vs$ in $\tenv$ yields $\True$\ProseOrTypeError;
    \item $\vb$ is $\True$.
  \end{itemize}

  \item \AllApplyCase{t\_bits}
  \begin{itemize}
    \item $\vt$ does not subtype $\vs$ in $\tenv$;
    \item determining whether $\vt$ is anonymous yields $\vbone$;
    \item determining whether $\vs$ is anonymous yields $\vbtwo$;
    \item determining whether $\vt$ \subtypesatisfies\ $\vs$ in $\tenv$ yields $\vbthree$;
    \item $(\vbone \lor \vbtwo) \land \vbthree$ is $\False$;
    \item $\vt$ is a bitvector type with width $\widtht$ and no bitfields;
    \item obtaining the \structure\ of $\vs$ in $\tenv$ yields a bitvector type with width \\
          $\widths$\ProseOrTypeError;
    \item determining whether $\widtht$ and $\widths$ are \bitwidthequivalent\ yields $\vb$.
  \end{itemize}

  \item \AllApplyCase{otherwise1}
  \begin{itemize}
    \item $\vt$ does not subtype $\vs$ in $\tenv$;
    \item determining whether $\vt$ is anonymous yields $\vbone$;
    \item determining whether $\vs$ is anonymous yields $\vbtwo$;
    \item determining whether $\vt$ \subtypesatisfies\ $\vs$ in $\tenv$ yields $\vbthree$;
    \item $(\vbone \lor \vbtwo) \land \vbthree$ is $\False$;
    \item obtaining the \structure\ of $\vs$ in $\tenv$ yields a $\vsstruct$\ProseOrTypeError;
    \item at least one of $\vt$ and $\vsstruct$ is not a bitvector type;
  \end{itemize}

  \item \AllApplyCase{otherwise2}
  \begin{itemize}
    \item $\vt$ does not subtype $\vs$ in $\tenv$;
    \item determining whether $\vt$ is anonymous yields $\vbone$;
    \item determining whether $\vs$ is anonymous yields $\vbtwo$;
    \item determining whether $\vt$ \subtypesatisfies\ $\vs$ in $\tenv$ yields $\vbthree$;
    \item $(\vbone \lor \vbtwo) \land \vbthree$ is $\False$;
    \item obtaining the \structure\ of $\vs$ in $\tenv$ yields a $\vsstruct$\ProseOrTypeError;
    \item both $\vt$ and $\vsstruct$ are bitvector types;
    \item the bitvector type $\vt$ has a non-empty list of bitfields;
    \item $\vb$ is $\False$;
  \end{itemize}
\end{itemize}

\FormallyParagraph
\begin{mathpar}
\inferrule[subtypes]{
  \subtypesrel(\tenv, \vt, \vs) \typearrow \True
}{
  \typesat(\tenv, \vt, \vs) \typearrow \True
}
\end{mathpar}

\begin{mathpar}
\inferrule[anonymous]{
  \subtypesrel(\tenv, \vt, \vs) \typearrow \False\\
  \isanonymous(\tenv, \vt) \typearrow \vbone\\
  \isanonymous(\tenv, \vs) \typearrow \vbtwo\\
  \vbone \lor \vbtwo\\
  \subtypesat(\tenv, \vt, \vs) \typearrow \True
}{
  \typesat(\tenv, \vt, \vs) \typearrow \True
}
\end{mathpar}

\begin{mathpar}
\inferrule[t\_bits]{
  \subtypesrel(\tenv, \vt, \vs) \typearrow \False\\
  \isanonymous(\tenv, \vt) \typearrow \vbone\\
  \isanonymous(\tenv, \vs) \typearrow \vbtwo\\
  \subtypesat(\tenv, \vt, \vs) \typearrow \vbthree\\
  \neg((\vbone \lor \vbtwo) \land \vbthree)\\
  \vt = \TBits(\widtht, \emptylist)\\
  \tstruct(\tenv, \vs) \typearrow \TBits(\widths, \Ignore) \OrTypeError\\\\
  \bitwidthequal(\tenv, \widtht, \widths) \typearrow \vb
}{
  \typesat(\tenv, \vt, \vs) \typearrow \vb
}
\end{mathpar}

\begin{mathpar}
\inferrule[otherwise1]{
  \subtypesrel(\tenv, \vt, \vs) \typearrow \False\\
  \isanonymous(\tenv, \vt) \typearrow \vbone\\
  \isanonymous(\tenv, \vs) \typearrow \vbtwo\\
  \subtypesat(\tenv, \vt, \vs) \typearrow \vbthree\\
  \neg((\vbone \lor \vbtwo) \land \vbthree)\\
  \tstruct(\tenv, \vs) \typearrow \vsstruct\\
  \astlabel(\vt) \neq \TBits \lor \astlabel(\vsstruct) \neq \TBits
}{
  \typesat(\tenv, \vt, \vs) \typearrow \overname{\False}{\vb}
}
\end{mathpar}

\begin{mathpar}
\inferrule[otherwise2]{
  \subtypesrel(\tenv, \vt, \vs) \typearrow \False\\
  \isanonymous(\tenv, \vt) \typearrow \vbone\\
  \isanonymous(\tenv, \vs) \typearrow \vbtwo\\
  \subtypesat(\tenv, \vt, \vs) \typearrow \vbthree\\
  \neg((\vbone \lor \vbtwo) \land \vbthree)\\
  \tstruct(\tenv, \vs) \typearrow \vsstruct\\
  \astlabel(\vt) = \TBits \land \astlabel(\vsstruct) = \TBits\\
  \vt = \TBits(\widtht, \bitfields)\\
  \bitfields \neq \emptylist
}{
  \typesat(\tenv, \vt, \vs) \typearrow \overname{\False}{\vb}
}
\end{mathpar}
\CodeSubsection{\TypeSatisfactionBegin}{\TypeSatisfactionEnd}{../types.ml}

\TypingRuleDef{CheckTypeSatisfaction}
\hypertarget{def-checktypesat}{}
We also define
\[
  \checktypesat(\overname{\staticenvs}{\tenv} \aslsep \overname{\ty}{\vt} \aslsep \overname{\ty}{\vs})
  \aslto \{\True\} \cup \overname{\typeerror}{\TypeErrorConfig}
\]
which is the same as $\typesat$, but yields a \typingerrorterm{} when \\ $\typesat(\tenv, \vt, \vs)$ is $\False$.

The function assumes that both $\vt$ and $\vs$ are well-typed according to \secref{Types}.

\ExampleDef{Checking Type Satisfaction}
In \listingref{typing-typesat1},
checking whether \verb|(integer, T1)| \typesatisfies{} \verb|pairT|
for the assignment \verb|var pair: pairT = (1, dataT1)| yields $\True$.

In \listingref{typing-typesat3}, checking whether \verb|(intege, T2)|
\typesatisfies{} \verb|pairT| for the assignment \verb|pair = (1, dataT2);|
yields a \typingerrorterm.

\ProseParagraph
\OneApplies
\begin{itemize}
  \item \AllApplyCase{okay}
  \begin{itemize}
    \item \ProsetypesatTrue{$\tenv$}{$\vt$}{$\vs$};
    \item the result is $\True$.
  \end{itemize}

  \item \AllApplyCase{error}
  \begin{itemize}
    \item \ProsetypesatFalse{$\tenv$}{$\vt$}{$\vs$}..
    \item the result is a \typingerrorterm{} (\TypeSatisfactionFailure).
  \end{itemize}
\end{itemize}

\FormallyParagraph
\begin{mathpar}
\inferrule[okay]{
  \typesat(\tenv, \vt, \vs) \typearrow \True
}{
  \checktypesat(\tenv, \vt, \vs) \typearrow \True
}
\end{mathpar}

\begin{mathpar}
\inferrule[error]{
  \typesat(\tenv, \vt, \vs) \typearrow \False
}{
  \checktypesat(\tenv, \vt, \vs) \typearrow \TypeErrorVal{\TypeSatisfactionFailure}
}
\end{mathpar}

\TypingRuleDef{LowestCommonAncestor}
\hypertarget{def-lowestcommonancestor}{}
Annotating a conditional expression (see \TypingRuleRef{ECond}),
requires finding a single type that can be used to annotate the results of both subexpressions.
We refer to such a type as a \emph{\Proselca}, or LCA, for short, and define it next.

The function
\[
  \lca(\overname{\staticenvs}{\tenv} \aslsep \overname{\ty}{\vt} \aslsep \overname{\ty}{\vs})
  \aslto \overname{\ty}{\tty} \cup \overname{\typeerror}{\TypeErrorConfig}
\]
returns the \Proselca{} of types $\vt$ and $\vs$ in $\tenv$ --- $\tty$.
The result is a \typingerrorterm{} if a \Proselca{} does not exist or a \typingerrorterm{} is detected.

\ExampleDef{Lowest Common Ancestor Examples}
\listingref{example-lca} shows examples of conditional expressions and the resulting
\emph{\Proselca}.
\pagebreak
\ASLListing{Lowest common ancestor}{example-lca}{\typingtests/TypingRule.LowestCommonAncestor.asl}

\listingref{example-lca2} shows an example of a lowest common ancestor of two \bitvectortypesterm{}
of different widths.
\ASLListing{Lowest common ancestor of two bitvectors}{example-lca2}{\typingtests/TypingRule.LowestCommonAncestor2.asl}

\ProseParagraph
\OneApplies
\begin{itemize}
  \item \AllApplyCase{type\_equal}
  \begin{itemize}
    \item $\vt$ is \typeequal\ to $\vs$ in $\tenv$;
    \item $\tty$ is $\vs$ (can as well be $\vt$).
  \end{itemize}

  \item \AllApply
  \begin{itemize}
    \item $\vt$ is not \typeequal\ to $\vs$ in $\tenv$;
    \item \OneApplies
    \begin{itemize}
      \item \AllApplyCase{named\_subtype1}
      \begin{itemize}
        \item $\vt$ is a named type with identifier $\namesubt$, that is, $\TNamed(\namesubt)$;
        \item $\vs$ is a named type with identifier $\namesubs$, that is, $\TNamed(\namesubs)$;
        \item there is no \namedlowestcommonancestor\ of $\namesubs$ and $\namesubt$ in $\tenv$;
        \item obtaining the \underlyingtype\ of $\vs$ yields $\vanons$\ProseOrTypeError;
        \item obtaining the \underlyingtype\ of $\vt$ yields $\vanont$\ProseOrTypeError;
        \item obtaining the lowest common ancestor of $\vanons$ and $\vanont$ in $\tenv$ yields $\tty$\ProseOrTypeError.
      \end{itemize}

      \item \AllApplyCase{named\_subtype2}
      \begin{itemize}
        \item $\vt$ is a named type with identifier $\namesubt$, that is, $\TNamed(\namesubt)$;
        \item $\vs$ is a named type with identifier $\namesubs$, that is, $\TNamed(\namesubs)$;
        \item the \namedlowestcommonancestor\ of $\namesubs$ and $\namesubt$ in $\tenv$ is \\
              $\name$\ProseOrTypeError;
        \item $\tty$ is the named type with identifier $\name$, that is, $\TNamed(\name)$.
      \end{itemize}

      \item \AllApplyCase{one\_named1}
      \begin{itemize}
        \item only one of $\vt$ or $\vs$ is a named type;
        \item obtaining the \underlyingtype\ of $\vs$ yields $\vanons$\ProseOrTypeError;
        \item obtaining the \underlyingtype\ of $\vt$ yields $\vanont$\ProseOrTypeError;
        \item $\vanont$ is \typeequal\ to $\vanons$;
        \item $\tty$ is $\vt$ if it is a named type (that is, $\astlabel(\vt)=\TNamed$), and $\vs$ otherwise.
      \end{itemize}

      \item \AllApplyCase{one\_named2}
      \begin{itemize}
        \item only one of $\vt$ or $\vs$ is a named type;
        \item obtaining the \underlyingtype\ of $\vs$ yields $\vanons$\ProseOrTypeError;
        \item obtaining the \underlyingtype\ of $\vt$ yields $\vanont$\ProseOrTypeError;
        \item $\vanont$ is not \typeequal\ to $\vanons$;
        \item the lowest common ancestor of $\vanont$ and $\vanons$ in $\tenv$ is $\tty$\ProseOrTypeError.
      \end{itemize}

      \item \AllApplyCase{t\_int\_unconstrained}
      \begin{itemize}
        \item both $\vt$ and $\vs$ are integer types;
        \item at least one of $\vt$ or $\vs$ is an unconstrained integer type;
        \item $\tty$ is the unconstrained integer type.
      \end{itemize}

      \item \AllApplyCase{t\_int\_parameterized}
      \begin{itemize}
        \item neither $\vt$ nor $\vs$ are the unconstrained integer type;
        \item one of $\vt$ and $\vs$ is a \parameterizedintegertype;
        \item the \wellconstrainedversion\ of $\vt$ is $\vtone$;
        \item the \wellconstrainedversion\ of $\vs$ is $\vsone$;
        \item $\tty$ the lowest common ancestor of $\vtone$ and $\vsone$ in $\tenv$ is $\tty$\ProseOrTypeError.
      \end{itemize}

      \item \AllApplyCase{t\_int\_wellconstrained}
      \begin{itemize}
        \item $\vt$ is a well-constrained integer type with constraints $\cst$ and \Proseprecisionlossindicator{} $\vpone$;
        \item $\vs$ is a well-constrained integer type with constraints $\css$ and \Proseprecisionlossindicator{} $\vpone$;
        \item applying $\precisionjoin$ on $\vpone$ and $\vptwo$ yields $\vp$;
        \item $\tty$ is the well-constrained integer type with constraints $\cst \concat \css$ and \Proseprecisionlossindicator{} $\vp$.
      \end{itemize}

      \item \AllApplyCase{t\_bits}
      \begin{itemize}
        \item $\vt$ is a bitvector type with length expression $\vet$, that is, $\TBits(\vet, \Ignore)$;
        \item $\vs$ is a bitvector type with length expression $\ves$, that is, $\TBits(\ves, \Ignore)$;
        \item applying $\typeequal$ to $\vt$ and $\vs$ in $\tenv$ yields $\False$;
        \item applying $\exprequal$ to $\vet$ and $\ves$ in $\tenv$ yields $\vbequal$;
        \item checking whether $\vbequal$ is $\True$ yields $\True$\ProseTerminateAs{\NoLCA};
        \item $\tty$ is a bitvector type with length expression $\vet$ and an empty bitfield list, that is, $\TBits(\vet, \emptylist)$.
      \end{itemize}

      \item \AllApplyCase{t\_array}
      \begin{itemize}
        \item $\vt$ is an array type with width expression $\widtht$ and element type $\vtyt$;
        \item $\vs$ is an array type with width expression $\widths$ and element type $\vtys$;
        \item applying $\arraylengthequal$ to $\widtht$ and $\widths$ in $\tenv$ to equate the array lengths,
              yields $\vbequallength$\ProseOrTypeError;
        \item checking that $\vbequallength$ is $\True$ yields $\True$\ProseTerminateAs{\NoLCA};
        \item the lowest common ancestor of $\vtyt$ and $\vtys$ is $\vtone$\ProseOrTypeError;
        \item $\tty$ is an array type with width expression $\widths$ and element type $\vtone$.
      \end{itemize}

      \item \AllApplyCase{t\_tuple}
      \begin{itemize}
        \item $\vt$ is a \tupletypeterm{} with type list $\vlit$;
        \item $\vs$ is a \tupletypeterm{} with type list $\vlis$;
        \item checking whether $\vlit$ and $\vlis$ have the same number of elements yields $\True$
              or a \typingerrorterm{}, which short-circuits the entire rule (indicating that the number of elements in both tuples is expected
              to be the same and thus there is no lowest common ancestor);
        \item applying $\lca$ to $\vlit[\vi]$ and $\vlis[\vi]$ in $\tenv$, for every position of $\vlit$,
              yields $\vt_\vi$\ProseOrTypeError;
        \item define $\vli$ to be the list of types $\vt_\vi$, for every position of $\vlit$;
        \item define $\tty$ as the \tupletypeterm{} with list of types $\vli$, that is, $\TTuple(\vli)$.
      \end{itemize}

      \item \AllApplyCase{error}
      \begin{itemize}
        \item either the AST labels of $\vt$ and $\vs$ are different, or one of them is $\TEnum$, $\TRecord$, $\TCollection$, or $\TException$;
        \item the result is a \typingerrorterm{} indicating the lack of a lowest common ancestor.
      \end{itemize}
    \end{itemize}
  \end{itemize}
\end{itemize}
\CodeSubsection{\LowestCommonAncestorBegin}{\LowestCommonAncestorEnd}{../types.ml}

\FormallyParagraph
Since we do not impose a canonical representation on types (e.g., \verb|integer {1, 2}| is \typeequivalent{} to \verb|integer {1..2}|),
the lowest common ancestor is not unique.
We define $\lca(\tenv, \vt, \vs)$ to be any type $\vtp$ that is \typeequivalent\ to the lowest common ancestor of $\vt$ and $\vs$.

\begin{mathpar}
\inferrule[type\_equal]{
  \typeequal(\tenv, \vt, \vs) \typearrow \True
}{
  \lca(\tenv, \vt, \vs) \typearrow \overname{\vs}{\tty}
}
\end{mathpar}

\begin{mathpar}
\inferrule[named\_subtype1]{
  \vt = \TNamed(\namesubs)\\
  \vs = \TNamed(\namesubt)\\
  \typeequal(\tenv, \vt, \vs) \typearrow \False\\
  \namedlca(\tenv, \namesubs, \namesubt) \typearrow \None \OrTypeError\\\\
  \makeanonymous(\tenv, \vs) \typearrow \vanons \OrTypeError\\\\
  \makeanonymous(\tenv, \vt) \typearrow \vanont \OrTypeError\\\\
  \lca(\tenv, \vanont, \vanons) \typearrow \tty \OrTypeError
}{
  \lca(\tenv, \vt, \vs) \typearrow \tty
}
\end{mathpar}

\begin{mathpar}
\inferrule[named\_subtype2]{
  \vt = \TNamed(\namesubs)\\
  \vs = \TNamed(\namesubt)\\
  \typeequal(\tenv, \vt, \vs) \typearrow \False\\
  \namedlca(\tenv, \namesubs, \namesubt) \typearrow \langle\name\rangle \OrTypeError\\
}{
  \lca(\tenv, \vt, \vs) \typearrow \overname{\TNamed(\name)}{\tty}
}
\end{mathpar}

\begin{mathpar}
\inferrule[one\_named1]{
  \typeequal(\tenv, \vt, \vs) \typearrow \False\\
  (\astlabel(\vt) = \TNamed \lor \astlabel(\vs) = \TNamed)\\
  \astlabel(\vt) \neq \astlabel(\vs)\\
  \makeanonymous(\tenv, \vs) \typearrow \vanons \OrTypeError\\\\
  \makeanonymous(\tenv, \vt) \typearrow \vanont \OrTypeError\\\\
  \typeequal(\tenv, \vanont, \vanons) \typearrow \True\\
  \tty \eqdef \choice{\astlabel(\vt) = \TNamed}{\vt}{\vs}
}{
  \lca(\tenv, \vt, \vs) \typearrow \tty
}
\end{mathpar}

\begin{mathpar}
\inferrule[one\_named2]{
  \typeequal(\tenv, \vt, \vs) \typearrow \False\\
  (\astlabel(\vt) = \TNamed \lor \astlabel(\vs) = \TNamed)\\
  \astlabel(\vt) \neq \astlabel(\vs)\\
  \makeanonymous(\tenv, \vs) \typearrow \vanons \OrTypeError\\\\
  \makeanonymous(\tenv, \vt) \typearrow \vanont \OrTypeError\\\\
  \typeequal(\tenv, \vanont, \vanons) \typearrow \False\\
  \lca(\tenv, \vanont, \vanons) \typearrow \tty \OrTypeError
}{
  \lca(\tenv, \vt, \vs) \typearrow \tty
}
\end{mathpar}

\begin{mathpar}
\inferrule[t\_int\_unconstrained]{
  \typeequal(\tenv, \vt, \vs) \typearrow \False\\
  \astlabel(\vt) = \astlabel(\vs) = \TInt\\
  \isunconstrainedinteger(\vt) \lor \isunconstrainedinteger(\vs)
}{
  \lca(\tenv, \vt, \vs) \typearrow \overname{\unconstrainedinteger}{\tty}
}
\and
\inferrule[t\_int\_parameterized]{
  \typeequal(\tenv, \vt, \vs) \typearrow \False\\
  \astlabel(\vt) = \astlabel(\vs) = \TInt\\
  \neg\isunconstrainedinteger(\vt)\\
  \neg\isunconstrainedinteger(\vs)\\
  \isparameterizedinteger(\vt) \lor \isparameterizedinteger(\vs)\\
  \towellconstrained(\tenv, \vt) \typearrow \vtone\\
  \towellconstrained(\tenv, \vs) \typearrow \vsone\\
  \lca(\tenv, \vtone, \vsone) \typearrow \tty \OrTypeError
}{
  \lca(\tenv, \vt, \vs) \typearrow \tty
}
\and
\inferrule[t\_int\_wellconstrained]
{
  \typeequal(\tenv, \vt, \vs) \typearrow \False\\
  \vt = \TInt(\WellConstrained(\cst, \vpone))\\
  \vs = \TInt(\WellConstrained(\css, \vptwo))\\
  \vp \eqdef \precisionjoin(\vpone, \vptwo)
}{
  \lca(\tenv, \vt, \vs) \typearrow \overname{\TInt(\WellConstrained(\cst \concat \css, \vp))}{\tty}
}
\end{mathpar}

\begin{mathpar}
\inferrule[t\_bits]{
  \typeequal(\tenv, \vt, \vs) \typearrow \False\\
  \exprequal(\tenv, \vet, \ves) \typearrow \vbequal\\
  \checktrans{\vbequal}{\NoLCA} \checktransarrow \True \OrTypeError
}{
  \lca(\tenv, \overname{\TBits(\vet, \Ignore)}{\vt}, \overname{\TBits(\ves, \Ignore)}{\vs}) \typearrow \overname{\TBits(\vet, \emptylist)}{\tty}
}
\end{mathpar}

\begin{mathpar}
\inferrule[t\_array]{
  \typeequal(\tenv, \vt, \vs) \typearrow \False\\
  \arraylengthequal(\tenv, \widtht, \widths) \typearrow \vbequallength \OrTypeError\\\\
  \checktrans{\vbequallength}{\NoLCA} \checktransarrow \True \OrTypeError\\\\
  \lca(\tenv, \vtyt, \vtys) \typearrow \vtone \OrTypeError
}{
  {
  \begin{array}{r}
  \lca(\tenv, \overname{\TArray(\widtht, \vtyt)}{\vt}, \overname{\TArray(\widths, \vtys)}{\vs}) \typearrow \\
  \overname{\TArray(\widtht, \vtone)}{\tty}
  \end{array}
  }
}
\end{mathpar}

\begin{mathpar}
\inferrule[t\_tuple]{
  \typeequal(\tenv, \vt, \vs) \typearrow \False\\
  \equallength(\vlit, \vlis) \typearrow \vb\\
  \checktrans{\vb}{\NoLCA} \typearrow \True \OrTypeError\\\\
  {
    \begin{array}{r}
  \vi\in\listrange(\vlit): \lca(\tenv, \vlit[\vi], \vlis[\vi]) \typearrow \\
   \vt_\vi \OrTypeError
    \end{array}
  }\\
  \vli \eqdef [\vi\in\listrange(\vlit): \vt_\vi]
}{
  \lca(\tenv, \overname{\TTuple(\vlit)}{\vt}, \overname{\TTuple(\vlis)}{\vs}) \typearrow \overname{\TTuple(\vli)}{\tty}
}
\end{mathpar}

\begin{mathpar}
\inferrule[error]{
  \typeequal(\tenv, \vt, \vs) \typearrow \False\\
  (\astlabel(\vt) \neq \astlabel(\vs)) \lor
  \astlabel(\vt) \in \{\TEnum, \TRecord, \TException, \TCollection\}
}{
  \lca(\tenv, \vt, \vs) \typearrow \TypeErrorVal{\NoLCA}
}
\end{mathpar}

\identr{YZHM}

\TypingRuleDef{ApplyUnopType}
\hypertarget{def-applyunoptype}{}
The function
\[
  \applyunoptype(\overname{\staticenvs}{\tenv} \aslsep \overname{\unop}{\op} \aslsep \overname{\ty}{\vt})
  \aslto \overname{\ty}{\vs} \cup \overname{\typeerror}{\TypeErrorConfig}
\]
determines the result type of applying a unary operator when the type of its operand is known.
Similarly, we determine the negation of integer constraints.
\ProseOtherwiseTypeError

\ExampleDef{Applying Unary Operations to Types}
\listingref{apply-unop-type} shows examples of typing applications of unary operations.
\ASLListing{Applying unary operations to types}{apply-unop-type}{\typingtests/TypingRule.ApplyUnopType.asl}

\ProseParagraph
\OneApplies
\begin{itemize}
\item \AllApplyCase{bnot\_t\_bool}
  \begin{itemize}
    \item $\op$ is $\BNOT$;
    \item determining whether $\vt$ \typesatisfies\ $\TBool$ yields $\True$\ProseOrTypeError;
    \item $\vs$ is $\TBool$;
  \end{itemize}

\item \AllApplyCase{neg\_error}
\begin{itemize}
  \item $\op$ is $\NEG$;
  \item determining whether $\vt$ \typesatisfies\ $\TReal$ yields $\False$\ProseOrTypeError;
  \item determining whether $\vt$ \typesatisfies\ $\unconstrainedinteger$ yields $\False$\ProseOrTypeError;
  \item the result is a \typingerrorterm{} indicating the $\NEG$ is appropriate only for the \realtypeterm{} and the \integertypeterm{};
\end{itemize}

\item \AllApplyCase{neg\_t\_real}
\begin{itemize}
  \item $\op$ is $\NEG$;
  \item determining whether $\vt$ \typesatisfies\ $\TReal$ yields $\True$;
  \item $\vs$ is $\TReal$;
\end{itemize}

\item \AllApplyCase{neg\_t\_int\_unconstrained}
\begin{itemize}
  \item $\op$ is $\NEG$;
  \item obtaining the \wellconstrainedstructure\ of $\vt$ yields $\unconstrainedinteger$\ProseOrTypeError;
  \item $\vs$ is $\unconstrainedinteger$;
\end{itemize}

\item \AllApplyCase{neg\_t\_int\_well\_constrained}
\begin{itemize}
  \item $\op$ is $\NEG$;
  \item obtaining the \wellconstrainedstructure\ of $\vt$ yields the well-constrained integer type with constraints $\vcs$ and \Proseprecisionlossindicator{} $\vp$\ProseOrTypeError;
  \item negating the constraints in $\vcs$ (see $\negateconstraint$) yields $\vcsnew$;
  \item $\vs$ is the well-constrained integer type with constraints $\vcsnew$ and \Proseprecisionlossindicator{} $\vp$, that is, \\
  $\TInt(\WellConstrained(\vcsnew, \vp))$;
\end{itemize}

\item \AllApplyCase{not\_t\_bits}
  \begin{itemize}
  \item $\op$ is $\NOT$;
  \item $\vt$ has the structure of a bitvector;
  \item $\vs$ is $\vt$.
  \end{itemize}
\end{itemize}

\FormallyParagraph
\begin{mathpar}
\inferrule[bnot\_t\_bool]{
  \checktypesat(\tenv, \vtone, \TBool) \typearrow \True \OrTypeError\\
}{
  \applyunoptype(\tenv, \BNOT, \vtone) \typearrow \TBool
}
\end{mathpar}
\CodeSubsection{\ApplyUnopTypeBegin}{\ApplyUnopTypeEnd}{../Typing.ml}

\begin{mathpar}
\inferrule[neg\_error]{
  \typesat(\tenv, \vt, \unconstrainedinteger) \typearrow \False \OrTypeError\\\\
  \typesat(\tenv, \vt, \TReal) \typearrow \False \OrTypeError\\
}{
  \applyunoptype(\tenv, \overname{\NEG}{\op}, \vt) \typearrow \TypeErrorVal{\BadOperands}
}
\end{mathpar}

\begin{mathpar}
\inferrule[neg\_t\_real]{
  \typesat(\tenv, \vt, \TReal) \typearrow \True
}{
  \applyunoptype(\tenv, \overname{\NEG}{\op}, \vt) \typearrow \overname{\TReal}{\vs}
}
\end{mathpar}

\begin{mathpar}
\inferrule[neg\_t\_int\_unconstrained]{
  \getwellconstrainedstructure(\tenv, \vt) \typearrow \unconstrainedinteger \OrTypeError
}{
  \applyunoptype(\tenv, \overname{\NEG}{\op}, \vt) \typearrow \overname{\unconstrainedinteger}{\vs}
}
\end{mathpar}

\begin{mathpar}
\inferrule[neg\_t\_int\_well\_constrained]{
  \getwellconstrainedstructure(\tenv, \vt) \typearrow \TInt(\WellConstrained(\vcs))\\
  \vc \in \vcs: \negateconstraint(\vc) \typearrow \vneg_\vc\\
  \vcsnew \eqdef [\vc \in \vcs: \vneg_\vc]
}{
  \applyunoptype(\tenv, \overname{\NEG}{\op}, \vt) \typearrow \overname{\TInt(\WellConstrained(\vcsnew))}{\vs}
}
\end{mathpar}

\begin{mathpar}
\inferrule[not\_t\_bits]{
  \checkstructurelabel(\tenv, \vt, \TBits) \typearrow \True \OrTypeError
}{
  \applyunoptype(\tenv, \overname{\NOT}{\op}, \vt) \typearrow \vt
}
\end{mathpar}

\TypingRuleDef{NegateConstraint}
\hypertarget{def-negateconstraint}{}
The helper function
\[
  \negateconstraint(\overname{\intconstraint}{\vc}) \aslto \overname{\intconstraint}{\newc}
\]
takes an integer constraint $\vc$ and returns the constraint $\newc$,
which corresponds to the negation of all the values that $\vc$ represents.

\ExampleRef{Applying Unary Operations to Types} shows examples of negating single expression
constraints and range constraints.

\ProseParagraph
\OneApplies
\begin{itemize}
  \item \AllApplyCase{exact}
  \begin{itemize}
    \item $\vc$ is the \Proseexactconstraint{$\ve$};
    \item \Proseeqdef{$\newc$}{the \Proseexactconstraint{the unary expression negating $\ve$}}.
  \end{itemize}

  \item \AllApplyCase{range}
  \begin{itemize}
    \item $\vc$ is the \Proserangeconstraint{$\vstart$}{$\vend$}
    \item \Proseeqdef{$\newc$}{the \Proserangeconstraint{that is the unary expression negating $\vend$}
    {that is the unary expression negating $\vstart$}}.
  \end{itemize}
\end{itemize}

\FormallyParagraph
\begin{mathpar}
\inferrule[exact]{}
{
  \negateconstraint(\overname{\ConstraintExact(\ve)}{\vc}) \typearrow
  \overname{\ConstraintExact(\EUnop(\SUB, \ve))}{\newc}
}
\end{mathpar}

\begin{mathpar}
\inferrule[range]{}
{
  \negateconstraint(\overname{\ConstraintRange(\vstart, \vend)}{\vc}) \typearrow \\
  \overname{\ConstraintRange(\EUnop(\SUB, \vend), \EUnop(\SUB, \vstart))}{\newc}
}
\end{mathpar}

\TypingRuleDef{ApplyBinopTypes}
\hypertarget{def-applybinoptypes}{}
The function
\[
  \applybinoptypes(\overname{\staticenvs}{\tenv} \aslsep \overname{\binop}{\op} \aslsep \overname{\ty}{\vtone}
  \aslsep \overname{\ty}{\vttwo})
  \aslto \overname{\ty}{\vt} \cup \overname{\typeerror}{\TypeErrorConfig}
\]
determines the result type $\vt$ of applying the binary operator $\op$
to operands of type $\vtone$ and $\vttwo$ in the \staticenvironmentterm{} $\tenv$.
\ProseOtherwiseTypeError

\ExampleDef{Filtering Constraints for Binary Operations}
Since binary operations may dynamically fail, the typechecker may remove values
that will definitely lead to \dynamicerrorsterm.
For example, in \listingref{apply-binop-constraints2},
the constraint \verb|integer{-1..1}| serves for the denominator in \verb|x DIV y|,
and since \verb|-1| and \verb|0| will always lead to a \dynamicerrorterm,
the typechecker removes them for the purpose of typing \verb|x DIV y|,
which leaves only \verb|1| and the final type for \verb|z| is therefore
\verb|integer{2, 4}|.

\ASLListing{Removing values from constraints}{apply-binop-constraints2}
{\typingtests/TypingRule.ApplyBinopTypes.constraints2.asl}

\ExampleDef{Applying Binary Operations to Types}
\listingref{apply-binop-type} shows examples of typing binary operations.
\ASLListing{Applying binary operations to types}{apply-binop-type}
          {\typingtests/TypingRule.ApplyBinopTypes.asl}

\ExampleDef{Applying Binary Operations to Constrained Integers}
\listingref{apply-binop-constraints} shows examples of typing binary operations
applied to \constrainedinteger{} types.

Importantly, note that there is no AST for applying binary operations on
constraints.
For example, given a range constraint \verb|A..B| and an exact constraint |2|,
there is no AST to express \verb|(A..B) * 2|. Therefore, the constraints for typing
\verb|ab_times_2| approximate the set of values for \verb|(A..B) * 2| via four
range constraints. More precisely, they are a superset of the values for \verb|(A..B) * 2|.

\pagebreak
\ASLListing{Applying binary operations to constrained integers}{apply-binop-constraints}
          {\typingtests/TypingRule.ApplyBinopTypes.constraints.asl}

\ProseParagraph
\OneApplies
\begin{itemize}
  \item \AllApplyCase{named}
  \begin{itemize}
    \item at least one of $\vtone$ and $\vttwo$ is a \namedtype;
    \item determining the \underlyingtype\ if $\vtone$ yields $\vtoneanon$\ProseOrTypeError;
    \item determining the \underlyingtype\ if $\vttwo$ yields $\vttwoanon$\ProseOrTypeError;
    \item \Proseapplybinoptypes{$\tenv$}{$\op$}{$\vtoneanon$}{$\vttwoanon$}{$\vt$\ProseOrTypeError}.
  \end{itemize}

  \item \AllApplyCase{boolean}
  \begin{itemize}
    \item $\op$ is $\AND$, $\OR$, $\EQ$ or $\IMPL$;
    \item both $\vtone$ and $\vttwo$ are $\TBool$;
    \item $\vt$ is $\TBool$.
  \end{itemize}

  \item \AllApplyCase{bits\_arith}
  \begin{itemize}
    \item $\op$ is one of $\AND$, $\OR$, $\XOR$, $\ADD$, and $\SUB$;
    \item $\vtone$ is a bitvector type with width expression $\vwone$;
    \item $\vttwo$ is a bitvector type with width expression $\vwtwo$;
    \item checking whether $\vtone$ and $\vttwo$ have the \structure\ of bitvector types
          of the same width in $\tenv$ yields $\True$\ProseOrTypeError;
    \item $\vt$ is the bitvector type of width $\vwone$ and empty list of bitfields, that is, \\ $\TBits(\vwone, \emptylist)$.
  \end{itemize}

  \item \AllApplyCase{bits\_int}
  \begin{itemize}
    \item $\op$ is either $\ADD$ or $\SUB$;
    \item $\vtone$ is a bitvector type with width expression $\vw$;
    \item $\vttwo$ is an integer type;
    \item $\vt$ is the bitvector type of width $\vw$ and empty list of bitfields, that is, \\ $\TBits(\vw, \emptylist)$.
  \end{itemize}

  \item \AllApplyCase{bits\_concat}
  \begin{itemize}
    \item $\op$ is $\BVCONCAT$;
    \item $\vtone$ is a bitvector type with width expression $\vwone$;
    \item $\vttwo$ is a bitvector type with width expression $\vwtwo$;
    \item define $\vw$ as the addition of $\vwone$ and $\vwtwo$;
    \item applying \normalize{} to $\vw$ in $\tenv$ yields $\vwp$;
    \item $\vt$ is the bitvector type of width $\vwp$ and empty list of bitfields, that is, \\ $\TBits(\vw, \emptylist)$.
  \end{itemize}

  \item \AllApplyCase{string\_concat}
  \begin{itemize}
    \item $\op$ is $\STRCONCAT$;
    \item $\vtone$ and $\vttwo$ are not both bitvector types;
    \item checking that $\vtone$ is a \Prosesingulartype{} yields $\True$\ProseOrTypeError;
    \item checking that $\vttwo$ is a \Prosesingulartype{} yields $\True$\ProseOrTypeError;
    \item $\vt$ is the string type.
  \end{itemize}

  \item \AllApplyCase{rel}
  \begin{itemize}
    \item the operator $\op$ and types of $\vtone$ and $\vttwo$ match one of the rows in the following table:
    \[
    \begin{array}{lll}
      \mathbf{\op} & \mathbf{\vtone} & \mathbf{\vttwo} \\
      \hline
      \LE  & \TInt    & \TInt\\
      \GE  & \TInt    & \TInt\\
      \GT   & \TInt    & \TInt\\
      \LT   & \TInt    & \TInt\\
      \LE  & \TReal   & \TReal\\
      \GE  & \TReal   & \TReal\\
      \GT   & \TReal   & \TReal\\
      \LT   & \TReal   & \TReal\\
      \EQ & \TInt    & \TInt\\
      \NE  & \TInt    & \TInt\\
      \EQ & \TBool   & \TBool\\
      \NE  & \TBool   & \TBool\\
      \EQ & \TReal   & \TReal\\
      \NE  & \TReal   & \TReal\\
      \EQ & \TString & \TString\\
      \NE  & \TString & \TString
    \end{array}
    \]
    \item $\vt$ is $\TBool$.
  \end{itemize}

  \item \AllApplyCase{eq\_neq\_bits}
  \begin{itemize}
    \item $\op$ is either $\EQ$ or $\NE$;
    \item $\vtone$ is a bitvector type with width expression $\vwone$;
    \item $\vttwo$ is a bitvector type with width expression $\vwtwo$;
    \item checking whether the bitwidth of $\vtoneanon$ and $\vttwoanon$ is the same yields $\True$\ProseOrTypeError;
    \item $\vt$ is $\TBool$.
  \end{itemize}

  \item \AllApplyCase{eq\_neq\_enum}
  \begin{itemize}
    \item $\op$ is either $\EQ$ or $\NE$;
    \item $\vtone$ is $\TEnum(\vlione)$;
    \item $\vttwo$ is $\TEnum(\vlitwo)$;
    \item checking whether $\vlione$ is equal to $\vlitwo$ yields $\True$\ProseOrTypeError;
    \item $\vt$ is $\TBool$.
  \end{itemize}

  \item \AllApplyCase{arith\_t\_int\_unconstrained}
  \begin{itemize}
    \item $\op$ is one of $\{\MUL, \DIV, \DIVRM, \MOD, \SHL,  \SHR, \POW, \ADD, \SUB\}$;
    \item both $\vtone$ and $\vttwo$ are integer types and at least one them is the unconstrained integer type;
    \item $\vt$ is the unconstrained integer type;
  \end{itemize}

  \item \AllApplyCase{arith\_t\_int\_parameterized}
  \begin{itemize}
    \item $\op$ is one of $\{\MUL, \DIV, \DIVRM, \MOD, \SHL,  \SHR, \POW, \ADD, \SUB\}$;
    \item both $\vtone$ and $\vttwo$ are integer types, neither is an unconstrained integer type, and at least one them is a \parameterizedintegertype;
    \item applying $\towellconstrained$ to $\vtone$ yields $\vtonewellconstrained$;
    \item applying $\towellconstrained$ to $\vttwo$ yields $\vttwowellconstrained$;
    \item \Proseapplybinoptypes{$\tenv$}{$\op$}{$\vtonewellconstrained$}{$\vttwowellconstrained$}{$\vt$}.
  \end{itemize}

  \item \AllApplyCase{arith\_t\_int\_wellconstrained}
  \begin{itemize}
    \item $\op$ is one of $\{\MUL, \POW, \ADD, \SUB, \DIVRM, \DIV, \MOD, \SHL, \SHR\}$;
    \item $\vtone$ is the well-constrained integer type with constraints $\csone$ and \Proseprecisionlossindicator{} $\vpone$;
    \item $\vttwo$ is the well-constrained integer type with constraints $\cstwo$ and \Proseprecisionlossindicator{} $\vptwo$;
    \item applying $\annotateconstraintbinop$ to $\op$, $\csone$, and $\cstwo$ in $\tenv$ yields $\vc$ and $\vpthree$;
    \item defining $\vpthree$ as the $\precisionjoin$ of $\vpone$, $\vptwo$, and $\vpthree$;
    \item $\vt$ is the well-constrained integer type with constraints $\vc$ and \Proseprecisionlossindicator{} $\vp$\
  \end{itemize}

  \item \AllApplyCase{arith\_real}
  \begin{itemize}
    \item the operator $\op$ and types of $\vtone$ and $\vttwo$ match one of the rows in the following table:
    \[
    \begin{array}{lll}
      \mathbf{\op} & \mathbf{\vtone} & \mathbf{\vttwo} \\
      \hline
      \ADD  & \TReal    & \TReal\\
      \SUB & \TReal    & \TReal\\
      \MUL   & \TReal    & \TReal\\
      \POW   & \TReal    & \TInt\\
      \RDIV  & \TReal    & \TReal
    \end{array}
    \]
    \item $\vt$ is $\TReal$.
  \end{itemize}

  \item \AllApplyCase{error}
  \begin{itemize}
    \item obtaining the \underlyingtype\ of $\vtone$ in $\tenv$ yields $\vtoneanon$\ProseOrTypeError;
    \item obtaining the \underlyingtype\ of $\vttwo$ in $\tenv$ yields $\vttwoanon$\ProseOrTypeError;
    \item the operator and the AST labels of $\vtoneanon$ and $\vttwoanon$ do not match any of the rows in the following table,
    where $\vlone$ and $\vltwo$ are the AST labels of any \Prosesingulartypes{}:
    \[
    \begin{array}{lll}
      \hline
      \mathbf{\op} & \mathbf{\astlabel(\vtoneanon)} & \mathbf{\astlabel(\vttwoanon)} \\
      \hline
      \AND     & \TBool  & \TBool\\
      \OR      & \TBool  & \TBool\\
      \EQ    & \TBool  & \TBool\\
      \IMPL    & \TBool  & \TBool\\
      %
      \AND     & \TBits  & \TBits\\
      \OR      & \TBits  & \TBits\\
      \XOR     & \TBits  & \TBits\\
      \ADD    & \TBits  & \TBits\\
      \SUB   & \TBits  & \TBits\\
      \BVCONCAT& \TBits  & \TBits\\
      %
      \STRCONCAT& \vlone  & \vltwo\\
      %
      \ADD    & \TBits  & \TInt\\
      \SUB   & \TBits  & \TInt\\
      %
      \LE     & \TInt     & \TInt\\
      \GE     & \TInt     & \TInt\\
      \GT      & \TInt     & \TInt\\
      \LT      & \TInt     & \TInt\\
      \LE     & \TReal    & \TReal\\
      \GE     & \TReal    & \TReal\\
      \GT      & \TReal    & \TReal\\
      \LT      & \TReal    & \TReal\\
      \EQ    & \TInt     & \TInt\\
      \NE     & \TInt     & \TInt\\
      \EQ    & \TBool    & \TBool\\
      \NE     & \TBool    & \TBool\\
      \EQ    & \TReal    & \TReal\\
      \NE     & \TReal    & \TReal\\
      \EQ    & \TString  & \TString\\
      \NE     & \TString  & \TString\\
      %
      \MUL     & \TInt  & \TInt\\
      \DIV     & \TInt  & \TInt\\
      \DIVRM   & \TInt  & \TInt\\
      \MOD     & \TInt  & \TInt\\
      \SHL     & \TInt  & \TInt\\
      \SHR     & \TInt  & \TInt\\
      \POW     & \TInt  & \TInt\\
      \ADD    & \TInt  & \TInt\\
      \SUB   & \TInt  & \TInt\\
      \ADD    & \TReal & \TReal\\
      \SUB   & \TReal & \TReal\\
      \MUL     & \TReal & \TReal\\
      \RDIV    & \TReal & \TReal\\
      \POW     & \TReal & \TInt\\
      %
      \ADD    & \TReal & \TReal\\
      \SUB   & \TReal & \TReal\\
      \MUL     & \TReal & \TReal\\
      \POW     & \TReal & \TInt\\
      \RDIV    & \TReal & \TReal\\
      \hline
    \end{array}
    \]
  \end{itemize}
\end{itemize}

\FormallyParagraph
\begin{mathpar}
\inferrule[named]{
  \astlabel(\vtone) = \TNamed \lor \astlabel(\vttwo) = \TNamed\\
  \makeanonymous(\tenv, \vtone) \typearrow \vtoneanon \OrTypeError\\\\
  \makeanonymous(\tenv, \vttwo) \typearrow \vttwoanon \OrTypeError\\\\
  \applybinoptypes(\tenv, \op, \vtoneanon, \vttwoanon) \typearrow \vt \OrTypeError
}{
  \applybinoptypes(\tenv, \op, \vtone, \vttwo) \typearrow \vt
}
\end{mathpar}

\begin{mathpar}
\inferrule[boolean]{
  \op \in  \{\BAND, \BOR, \IMPL, \EQ\}
}{
  \applybinoptypes(\tenv, \op, \overname{\TBool}{\vtone}, \overname{\TBool}{\vttwo}) \typearrow \overname{\TBool}{\vt}
}
\end{mathpar}

\begin{mathpar}
\inferrule[bits\_arith]{
  \op \in  \{\AND, \OR, \XOR, \ADD, \SUB\}\\
  \checkbitsequalwidth(\tenv, \vtone, \vttwo) \typearrow \True \OrTypeError
}{
  \applybinoptypes(\tenv, \op, \overname{\TBits(\vwone, \Ignore)}{\vtone}, \overname{\TBits(\vwtwo, \Ignore)}{\vttwo})
  \typearrow \overname{\TBits(\vwone, \emptylist)}{\vt}
}
\end{mathpar}

\begin{mathpar}
\inferrule[bits\_int]{
  \op \in  \{\ADD, \SUB\}}{
  \applybinoptypes(\tenv, \op, \overname{\TBits(\vw, \Ignore)}{\vtone}, \overname{\TInt(\Ignore)}{\vttwo}) \typearrow
  \overname{\TBits(\vw, \emptylist)}{\vt}
}
\end{mathpar}

\begin{mathpar}
\inferrule[bits\_concat]{
  \vw \eqdef \EBinop(\ADD, \vwone, \vwtwo) \\
  \normalize(\tenv, \vw) \typearrow \vwp
}{
  \applybinoptypes(\tenv, \BVCONCAT, \overname{\TBits(\vwone, \Ignore)}{\vtone}, \overname{\TBits(\vwtwo, \Ignore)}{\vttwo}) \typearrow
  \overname{\TBits(\vwp, \emptylist)}{\vt}
}
\end{mathpar}

\begin{mathpar}
\inferrule[string\_concat]{
  \vtone \neq \TBits(\Ignore, \Ignore) \lor \vttwo \neq \TBits(\Ignore, \Ignore) \\
  \checktrans{\issingular(\vtone)}{\UnexpectedType} \typearrow \True \OrTypeError \\
  \checktrans{\issingular(\vttwo)}{\UnexpectedType} \typearrow \True \OrTypeError
}{
  \applybinoptypes(\tenv, \STRCONCAT, \vtone, \vttwo) \typearrow
  \overname{\TString}{\vt}
}
\end{mathpar}

\begin{mathpar}
\inferrule[rel]{
  {
    (\op, \vtone, \vttwo) \in \left\{
      \begin{array}{lclcl}
        (\LE     &,& \TInt     &,& \TInt)\\
        (\GE     &,& \TInt     &,& \TInt)\\
        (\GT      &,& \TInt     &,& \TInt)\\
        (\LT      &,& \TInt     &,& \TInt)\\
        (\LE     &,& \TReal    &,& \TReal)\\
        (\GE     &,& \TReal    &,& \TReal)\\
        (\GT      &,& \TReal    &,& \TReal)\\
        (\LT      &,& \TReal    &,& \TReal)\\
        (\EQ    &,& \TInt     &,& \TInt)\\
        (\NE     &,& \TInt     &,& \TInt)\\
        (\EQ    &,& \TBool    &,& \TBool)\\
        (\NE     &,& \TBool    &,& \TBool)\\
        (\EQ    &,& \TReal    &,& \TReal)\\
        (\NE     &,& \TReal    &,& \TReal)\\
        (\EQ    &,& \TString  &,& \TString)\\
        (\NE     &,& \TString  &,& \TString)\\
      \end{array}
      \right\}
  }
}{
  \applybinoptypes(\tenv, \op, \vtone, \vttwo) \typearrow \overname{\TBool}{\vt}
}
\end{mathpar}

\begin{mathpar}
\inferrule[eq\_neq\_bits]{
  \op \in  \{\EQ, \NE\}\\
  \checkbitsequalwidth(\tenv, \vtoneanon, \vttwoanon) \typearrow \True \OrTypeError
}{
  \applybinoptypes(\tenv, \op, \overname{\TBits(\vwone, \Ignore)}{\vtone}, \overname{\TBits(\vwtwo, \Ignore)}{\vttwo}) \typearrow \overname{\TBool}{\vt}
}
\end{mathpar}

\begin{mathpar}
\inferrule[eq\_neq\_enum]{
  \op \in  \{\EQ, \NE\}\\
  \checktrans{\vlione = \vlitwo}{DifferentEnumLabels} \checktransarrow \True \OrTypeError
}{
  \applybinoptypes(\tenv, \op, \overname{\TEnum(\vlione)}{\vtone}, \overname{\TEnum(\vlitwo)}{\vttwo}) \typearrow \overname{\TBool}{\vt}
}
\end{mathpar}

\begin{mathpar}
\inferrule[arith\_t\_int\_unconstrained]{
  \op \in  \{\MUL, \DIV, \DIVRM, \MOD, \SHL,  \SHR, \POW, \ADD, \SUB\}\\
  \vcone = \Unconstrained \lor \vctwo = \Unconstrained
}{
  \applybinoptypes(\tenv, \op, \overname{\TInt(\vcone)}{\vtone}, \overname{\TInt(\vctwo)}{\vttwo}) \typearrow \unconstrainedinteger
}
\end{mathpar}

\begin{mathpar}
\inferrule[arith\_t\_int\_parameterized]{
  \op \in  \{\MUL, \DIV, \DIVRM, \MOD, \SHL,  \SHR, \POW, \ADD, \SUB\}\\
  \astlabel(\vcone) = \Parameterized \lor \astlabel(\vctwo) = \Parameterized\\
  \astlabel(\vcone) \neq \Unconstrained \land \astlabel(\vctwo) \neq \Unconstrained\\
  \towellconstrained(\vtone) \typearrow \vtonewellconstrained\\
  \towellconstrained(\vttwo) \typearrow \vttwowellconstrained\\
  \applybinoptypes(\tenv, \vtonewellconstrained, \vttwowellconstrained) \typearrow \vt \OrTypeError
}{
  \applybinoptypes(\tenv, \op, \overname{\TInt(\vcone)}{\vtone}, \overname{\TInt(\vctwo)}{\vttwo}) \typearrow \vt
}
\end{mathpar}

\begin{mathpar}
\inferrule[arith\_t\_int\_wellconstrained]{
  \op \in  \{\MUL, \POW, \ADD, \SUB, \DIVRM, \DIV, \MOD, \SHL, \SHR\}\\
  \vcone = \WellConstrained(\cstwo, \vpone)\\
  \vctwo = \WellConstrained(\csone, \vptwo)\\
  \annotateconstraintbinop(\tenv, \op, \vcsone, \vcstwo) \typearrow (\cs, \vpthree) \OrTypeError\\\\
  \vp = \precisionjoin(\vpone, \precisionjoin(\vptwo, \vpthree))
}{
      \applybinoptypes(\tenv, \op, \overname{\TInt(\vcone)}{\vtone},
        \overname{\TInt(\vctwo)}{\vttwo}) \typearrow
        \overname{\TInt(\WellConstrained(\cs, \vp))}{\vt}
}
\end{mathpar}

\begin{mathpar}
\inferrule[arith\_real]{
  (\op, \vtone, \vttwo) \in
  {
    \left\{
    \begin{array}{lclcl}
      (\ADD  &,& \TReal &,& \TReal)\\
      (\SUB &,& \TReal &,& \TReal)\\
      (\MUL   &,& \TReal &,& \TReal)\\
      (\POW   &,& \TReal &,& \TInt)\\
      (\RDIV  &,& \TReal &,& \TReal)
    \end{array}
    \right\}
  }
}{
  \applybinoptypes(\tenv, \op, \vtone, \vttwo) \typearrow \overname{\TReal}{\vt}
}
\end{mathpar}

\begin{mathpar}
\inferrule[error]{
  \makeanonymous(\tenv, \vtone) \typearrow \vtoneanon \OrTypeError\\\\
  \makeanonymous(\tenv, \vttwo) \typearrow \vttwoanon \OrTypeError\\\\
  {
  \begin{array}{l}
  (\op, \astlabel(\vtoneanon), \astlabel(\vttwoanon)) \not\in\\
    \left\{
    \begin{array}{lclcl}
      (\AND     &,& \TBool  &,& \TBool)\\
      (\OR      &,& \TBool  &,& \TBool)\\
      (\EQ    &,& \TBool  &,& \TBool)\\
      (\IMPL    &,& \TBool  &,& \TBool)\\
      %
      (\AND     &,& \TBits  &,& \TBits)\\
      (\OR      &,& \TBits  &,& \TBits)\\
      (\XOR     &,& \TBits  &,& \TBits)\\
      (\ADD    &,& \TBits  &,& \TBits)\\
      (\SUB   &,& \TBits  &,& \TBits)\\
      (\BVCONCAT&,& \TBits  &,& \TBits)\\
      %
      (\ADD    &,& \TBits  &,& \TInt)\\
      (\SUB   &,& \TBits  &,& \TInt)\\
      %
      (\LE     &,& \TInt     &,& \TInt)\\
      (\GE     &,& \TInt     &,& \TInt)\\
      (\GT      &,& \TInt     &,& \TInt)\\
      (\LT      &,& \TInt     &,& \TInt)\\
      (\LE     &,& \TReal    &,& \TReal)\\
      (\GE     &,& \TReal    &,& \TReal)\\
      (\GT      &,& \TReal    &,& \TReal)\\
      (\LT      &,& \TReal    &,& \TReal)\\
      (\EQ    &,& \TInt     &,& \TInt)\\
      (\NE     &,& \TInt     &,& \TInt)\\
      (\EQ    &,& \TBool    &,& \TBool)\\
      (\NE     &,& \TBool    &,& \TBool)\\
      (\EQ    &,& \TReal    &,& \TReal)\\
      (\NE     &,& \TReal    &,& \TReal)\\
      (\EQ    &,& \TString  &,& \TString)\\
      (\NE     &,& \TString  &,& \TString)\\
      %
      (\MUL   &,& \TInt  &,& \TInt)\\
      (\DIV   &,& \TInt  &,& \TInt)\\
      (\DIVRM &,& \TInt  &,& \TInt)\\
      (\MOD   &,& \TInt  &,& \TInt)\\
      (\SHL   &,& \TInt  &,& \TInt)\\
      (\SHR   &,& \TInt  &,& \TInt)\\
      (\POW   &,& \TInt  &,& \TInt)\\
      (\ADD  &,& \TInt  &,& \TInt)\\
      (\SUB &,& \TInt  &,& \TInt)\\
      (\ADD  &,& \TReal &,& \TReal)\\
      (\SUB &,& \TReal &,& \TReal)\\
      (\MUL   &,& \TReal &,& \TReal)\\
      (\RDIV  &,& \TReal &,& \TReal)\\
      (\POW   &,& \TReal &,& \TInt)\\
      %
      (\ADD  &,& \TReal &,& \TReal)\\
      (\SUB &,& \TReal &,& \TReal)\\
      (\MUL   &,& \TReal &,& \TReal)\\
      (\POW   &,& \TReal &,& \TInt)\\
      (\RDIV  &,& \TReal &,& \TReal)
    \end{array}
    \right\}\\
    \cup \{ (\STRCONCAT, \astlabel(\vtone), \astlabel(\vttwo)) \;|\; \issingular(\vtone) \land \issingular(\vttwo)\}\\
  \end{array}
  }
}{
  \applybinoptypes(\tenv, \op, \vtone, \vttwo) \typearrow \TypeErrorVal{\BadOperands}
}
\end{mathpar}
\CodeSubsection{\applybinoptypesBegin}{\applybinoptypesEnd}{../Typing.ml}

\identr{BKNT} \identr{ZYWY} \identr{BZKW}
\identr{KFYS} \identr{KXMR} \identr{SQXN} \identr{MRHT} \identr{JGWF}
\identr{TTGQ} \identi{YHML} \identi{YHRP} \identi{VMZF} \identi{YXSY}
\identi{LGHJ} \identi{RXLG}

\TypingRuleDef{FindNamedLCA}
\hypertarget{def-namedlowestcommonancestor}{}
The function
\[
  \namedlca(\overname{\staticenvs}{\tenv} \aslsep \overname{\ty}{\vt} \aslsep \overname{\ty}{\vs})
  \aslto \overname{\ty}{\tty} \cup \overname{\typeerror}{\TypeErrorConfig}
\]
returns the lowest common named super type --- $\tty$ --- of the types $\vt$ and $\vs$ in $\tenv$.

\hypertarget{def-supers}{}
The helper function
\[
  \supers(\overname{\staticenvs}{\tenv} \aslsep \overname{\ty}{\vt})
  \aslto \pow{\ty}
\]
returns the set of \emph{named supertypes} of a type $\vt$
in the $\subtypes$ function of a \globalstaticenvironmentterm{} $\tenv$:
\[
  \supers(\tenv, \vt) \triangleq
  \begin{cases}
    \{\vt\} \cup \supers(\vs) & \text{ if }G^\tenv.\subtypes(\vt) = \vs\\
    \{\vt\}  & \text{ otherwise } (\text{that is, }G^\tenv.\subtypes(\vt) = \bot)\\
  \end{cases}
\]

\ExampleDef{Finding Named Lowest Common Ancestors}
In \listingref{find-named-lca},
the set of named supertypes for \verb|B2| is \verb|{A1, A2, B1, B2}|,
set of named supertypes for \verb|C2| is \verb|{A1, A2, C1, C2}|,
therefore the named lowest common ancestor of \verb|B2| and \verb|C2|
is \verb|A2|, while \verb|B2| and \verb|D1| have no
named lowest common ancestor.

\ASLListing{Finding named lowest common ancestors}{find-named-lca}{\typingtests/TypingRule.FindNamedLCA.asl}

\ProseParagraph
\OneApplies
\begin{itemize}
  \item $\vtsupers$ is in the set of named supertypes of $\vt$;
  \item \AllApplyCase{found}
  \begin{itemize}
    \item $\vs$ is in $\vtsupers$;
    \item $\tty$ is $\vs$;
  \end{itemize}

  \item \AllApplyCase{super}
  \begin{itemize}
    \item $\vs$ is not in $\vtsupers$;
    \item $\vs$ has a named super type in $\tenv$ --- $\vsp$;
    \item $\tty$ is the lowest common named \supertypeterm{} of $\vt$ and $\vsp$ in $\tenv$.
  \end{itemize}

  \item \AllApplyCase{none}
  \begin{itemize}
    \item $\vs$ is not in $\vtsupers$;
    \item $\vs$ has no named super type in $\tenv$;
    \item $\tty$ is $\None$.
  \end{itemize}
\end{itemize}

\FormallyParagraph
\begin{mathpar}
\inferrule[found]{
  \supers(\tenv, \vt) \typearrow \vtsupers\\
  \vs \in \vtsupers
}{
  \namedlca(\tenv, \vt, \vs) \typearrow \vs
}
\and
\inferrule[super]{
  \supers(\tenv, \vt) \typearrow \vtsupers\\
  \vs \not\in \vtsupers\\
  G^\tenv.\subtypes(\vs) = \vsp\\
  \namedlca(\tenv, \vt, \vsp) \typearrow \tty
}{
  \namedlca(\tenv, \vt, \vs) \typearrow \tty
}
\and
\inferrule[none]{
  \supers(\tenv, \vt) \typearrow \vtsupers\\
  \vs \not\in \vtsupers\\
  G^\tenv.\subtypes(\vs) = \bot
}{
  \namedlca(\tenv, \vt, \vs) \typearrow \None
}
\end{mathpar}

\TypingRuleDef{AnnotateConstraintBinop}
\hypertarget{def-annotateconstraintbinop}{}
The function
\[
\annotateconstraintbinop\left(
  \begin{array}{c}
  \overname{\{\Over,\Under\}}{\vapprox} \aslsep\\
  \overname{\staticenvs}{\tenv} \aslsep\\
  \overname{\binop}{\op} \aslsep\\
  \overname{\KleeneStar{\intconstraint}}{\csone} \aslsep\\
  \overname{\KleeneStar{\intconstraint}}{\cstwo}
  \end{array}
\right) \aslto
\begin{array}{r}
\left(
  \begin{array}{c}
  \overname{\KleeneStar{\intconstraint}}{\annotatedcs} \aslsep\\
  \overname{\precisionlossindicator}{\vp}
  \end{array}
\right)
\\ \cup\ \overname{\typeerror}{\TypeErrorConfig}
\end{array}
\]
annotates the application of the binary operation $\op$ to the lists of integer constraints
$\csone$ and $\cstwo$, yielding a list of constraints --- $\annotatedcs$.
%
If the list is empty, the result is either $\CannotUnderapproximate$ or $\CannotOverapproximate$,
based on $\vapprox$ (this function is invoked in the context of approximating lists of constraints).
%
\ProseOtherwiseTypeError\

The operator $\op$ is assumed to be only one of the operators in the following set:
$\{\SHL, \SHR, \POW, \MOD, \DIVRM, \SUB, \MUL, \ADD, \DIV\}$.
The rule employs $\binopisexploding$ to decide whether range constraints can be maintained
as range constraints or have to be converted to a list of exact constraints.

\ExampleDef{Annotating Constraints for Binary Operations}
Applying $\ADD$ to
$\{\AbbrevConstraintRange{\ELInt{2}}{\ELInt{4}}\}$ and
$\{\AbbrevConstraintExact{\ELInt{2}}\}$ results in\\
$\{\AbbrevConstraintRange{\ELInt{4}}{\ELInt{6}}\}$,
since $\binopisexploding(\ADD) \typearrow \False$
while applying $\MUL$ to the same lists of constraints results in
$\{\AbbrevConstraintExact{\ELInt{4}}, \AbbrevConstraintExact{\ELInt{6}}, \AbbrevConstraintExact{\ELInt{8}}\}$,
since $\binopisexploding(\MUL) \typearrow \True$.

Annotating the constraints involves applying symbolic reasoning and in particular filtering out values that
will definitely result in a \dynamicerrorterm{}.

Also see \ExampleRef{Applying Binary Operations to Constrained Integers}.

\ProseParagraph
\AllApply
\begin{itemize}
  \item applying $\binopfilterrhs$ to $\vapprox$, $\op$ $\cstwo$ in $\tenv$, to filter out constraints that will definitely fail dynamically,
        yields $\cstwof$\ProseTerminateAs{\CannotUnderapproximate, \CannotOverapproximate};
  \item \OneApplies
  \begin{itemize}
    \item \AllApplyCase{exploding}
    \begin{itemize}
      \item applying $\binopisexploding$ to $\op$ yields $\True$;
      \item applying $\explodeintervals$ to $\csone$ in $\tenv$ yields $(\csonee \aslsep \vpone)$;
      \item applying $\explodeintervals$ to $\cstwof$ in $\tenv$ yields $(\cstwoe \aslsep \vptwo)$;
      \item applying $\precisionjoin$ to $\vpone$ and $\vptwo$ yields $\vpzero$;
      \item \Proseeqdef{$\vexpectedconstraintlength$}{the number of constraints in \\
            $\cstwoe$ if $\op$ is $\MOD$
            and the multiplication of numbers of constraints in $\csonee$ and $\cstwoe$, respectively};
      \item \Proseeqdef{$(\csonearg, \cstwoarg, \vp)$}{$(\csonee, \cstwoe, \vpzero)$ if \\
            $\vexpectedconstraintlength$ is
            less than $\maxconstraintsize$ and \\
            $(\csone, \cstwof, \PrecisionLost)$, otherwise};
    \end{itemize}

    \item \AllApplyCase{non\_exploding}
    \begin{itemize}
      \item applying $\binopisexploding$ to $\op$ yields $\False$;
      \item \Proseeqdef{$\vp$}{$\PrecisionFull$};
      \item \Proseeqdef{$(\csonearg, \cstwoarg)$}{$(\csone, \cstwof)$};
    \end{itemize}
  \end{itemize}
  \item applying $\constraintbinop$ to $\op$, $\csonearg$, and $\cstwoarg$ yields $\csvanilla$;
  \item applying $\refineconstraintfordiv$ to $\vapprox$, $\op$ and $\csvanilla$ yields\\
        $\refinedcs$\ProseTerminateAs{\CannotUnderapproximate, \CannotOverapproximate};
  \item applying $\reduceconstraints$ to $\refinedcs$ in $\tenv$ yields $\annotatedcs$.
\end{itemize}

\FormallyParagraph
\begin{mathpar}
\inferrule[exploding]{
  \binopfilterrhs(\vapprox, \tenv, \op, \cstwo) \typearrow \cstwof \terminateas \CannotUnderapproximate, \CannotOverapproximate\\\\
  \commonprefixline\\\\
  \binopisexploding(\op) \typearrow \True\\
  \explodeintervals(\tenv, \csone) \typearrow (\csonee, \vpone)\\
  \explodeintervals(\tenv, \cstwof) \typearrow (\cstwoe, \vptwo)\\
  \vpzero \eqdef \precisionjoin(\vpone, \vptwo)\\
  \vexpectedconstraintlength \eqdef \choice{\op = \MOD}{\listlen{\cstwoe}}{\listlen{\csonee} \times \listlen{\cstwoe}}\\
  {
  (\csonearg, \cstwoarg, \vp) \eqdef \left\{
  \begin{array}{ll}
    \textbf{if }&\vexpectedconstraintlength < \maxconstraintsize\text{ then}\\
    & (\csonee, \cstwoe, \vpzero)\\
    \textbf{else}&\\
    & (\csone, \cstwof, \PrecisionLost )
  \end{array}\right.
  }\\\\
  \commonsuffixline\\\\
  \constraintbinop(\op, \csonearg, \cstwoarg) \typearrow \csvanilla\\
  \refineconstraintfordiv(\vapprox, \op, \csvanilla) \typearrow \refinedcs \terminateas \CannotUnderapproximate,\CannotOverapproximate\\\\
  \reduceconstraints(\tenv, \refinedcs) \typearrow \annotatedcs
}{
  \annotateconstraintbinop(\vapprox, \tenv, \op, \csone, \cstwo) \typearrow \annotatedcs
}
\end{mathpar}

\begin{mathpar}
\inferrule[non\_exploding]{
  \binopfilterrhs(\vapprox, \tenv, \op, \cstwo) \typearrow \cstwof \terminateas \CannotUnderapproximate, \CannotOverapproximate\\\\
  \commonprefixline\\\\
  \binopisexploding(\op) \typearrow \False\\
  \vp \eqdef \PrecisionFull \\
  (\csonearg, \cstwoarg) \eqdef (\csone, \cstwof)\\\\
  \commonsuffixline\\\\
  \constraintbinop(\op, \csonearg, \cstwoarg) \typearrow \csvanilla\\
  \refineconstraintfordiv(\vapprox, \op, \csvanilla) \typearrow \refinedcs \terminateas \CannotUnderapproximate, \CannotOverapproximate\\\\
  \reduceconstraints(\tenv, \refinedcs) \typearrow \annotatedcs
}{
  \annotateconstraintbinop(\vapprox, \tenv, \op, \csone, \cstwo) \typearrow \annotatedcs
}
\end{mathpar}
\CodeSubsection{\AnnotateConstraintBinopBegin}{\AnnotateConstraintBinopEnd}{../StaticOperations.ml}

\TypingRuleDef{BinopFilterRhs}
\hypertarget{def-binopfilterrhs}{}
The function
\[
\begin{array}{r}
\binopfilterrhs(
    \overname{\{\Over,\Under\}}{\vapprox} \aslsep
    \overname{\staticenvs}{\tenv} \aslsep
    \overname{\binop}{\op} \aslsep
    \overname{\KleeneStar{\intconstraint}}{\cs})
\aslto \\
\overname{\KleenePlus{\intconstraint}}{\newcs}
\cup \{\CannotUnderapproximate, \CannotOverapproximate\}
\end{array}
\]
filters the list of constraints $\cs$ by removing values that will definitely result in a dynamic
error if found on the right-hand-side of a binary operation expression with the operator $\op$
in any environment consisting of the \staticenvironmentterm{} $\tenv$.
The result is the filtered list of constraints $\newcs$.
%
If the list is empty, the result is either $\CannotUnderapproximate$ or $\CannotOverapproximate$,
based on $\vapprox$ (this function is invoked in the context of approximating lists of constraints).

\ExampleDef{Filtering Right-hand-side Constraints}
An example of filtering constraints appears in \listingref{apply-binop-constraints},
where the constraints inferred for \verb|a_div| filter out \verb|-5..0|
from the constraint \verb|-5..3|, thus avoiding including constraints
\verb|A DIV -5| and \verb|A DIV 0|.

\ProseParagraph
\OneApplies
\begin{itemize}
  \item \AllApplyCase{greater\_or\_equal}
  \begin{itemize}
    \item $\op$ is one of $\SHL$, $\SHR$, and $\POW$;
    \item define $\vf$ as the specialization of $\refineconstraintbysign$ for the predicate
          $\lambda x.\ x \geq 0$, which is $\True$ if and only if the tested number is greater or equal to $0$;
    \item refining the list of constraints $\cs$ with $\vf$ and $\vapprox$ via $\refineconstraints$ yields \\
          $\newcs$\ProseTerminateAs{\CannotUnderapproximate, \CannotOverapproximate};
    \item checking whether $\newcs$ is empty yields $\True$\ProseTerminateAs{\BadOperands}.
  \end{itemize}

  \item \AllApplyCase{greater\_than}
  \begin{itemize}
    \item $\op$ is one of $\MOD$, $\DIV$, and $\DIVRM$;
    \item define $\vf$ as the specialization of $\refineconstraintbysign$ for the predicate
          $\lambda x.\ x > 0$, which is $\True$ if and only if the tested number is greater than $0$;
    \item refining the list of constraints $\cs$ with $\vf$ and $\vapprox$ via $\refineconstraints$ yields \\
          $\newcs$\ProseTerminateAs{\CannotUnderapproximate, \CannotOverapproximate};
    \item checking whether $\newcs$ is empty yields $\True$\ProseTerminateAs{\BadOperands}.
  \end{itemize}

  \item \AllApplyCase{no\_filtering}
  \begin{itemize}
    \item $\op$ is one of $\SUB$, $\MUL$, and $\ADD$;
    \item $\newcs$ is $\cs$.
  \end{itemize}
\end{itemize}

\FormallyParagraph
\begin{mathpar}
\inferrule[greater\_or\_equal]{
  \op \in \{\SHL, \SHR, \POW\}\\\\
  \refineconstraintbysign(\tenv, \lambda x.\ x \geq 0) \typearrow \vf\\
  \refineconstraints(\vapprox, \cs, \vf) \typearrow \newcs \terminateas \CannotUnderapproximate, \CannotOverapproximate\\\\
  \checktrans{\newcs \neq \emptylist}{\BadOperands} \typearrow \True\OrTypeError
}{
  \binopfilterrhs(\vapprox, \tenv, \op, \cs) \typearrow \newcs
}
\end{mathpar}

\begin{mathpar}
\inferrule[greater\_than]{
  \op \in \{\MOD, \DIV, \DIVRM\}\\\\
  \vf \eqdef \refineconstraintbysign(\tenv, \lambda x.\ x > 0)\\
  \refineconstraints(\vapprox, \cs, \vf) \typearrow \newcs \terminateas \CannotUnderapproximate, \CannotOverapproximate\\\\
  \checktrans{\newcs \neq \emptylist}{\BadOperands} \typearrow \True\OrTypeError
}{
  \binopfilterrhs(\vapprox, \tenv, \op, \cs) \typearrow \newcs
}
\end{mathpar}

\begin{mathpar}
\inferrule[no\_filtering]{
  \op \in \{\SUB, \MUL, \ADD\}
}{
  \binopfilterrhs(\vapprox, \op, \cs) \typearrow \overname{\cs}{\newcs}
}
\end{mathpar}

\TypingRuleDef{RefineConstraintBySign}
\hypertarget{def-refineconstraintbysign}{}
The function
\[
\refineconstraintbysign(\overname{\staticenvs}{\tenv} \aslsep \overname{\Z\rightarrow \Bool}{\vp} \aslsep \overname{\intconstraint}{\vc})
\aslto \overname{\langle\intconstraint\rangle}{\vcopt}
\]
takes a predicate $\vp$ that returns $\True$ based on the sign of its input.
The function conservatively refines the constraint $\vc$ in $\tenv$ by applying symbolic reasoning to yield a new constraint
(inside an optional)
that represents the values that satisfy the $\vc$ and for which $\vp$ holds.
In this context, conservatively means that the new constraint may represent a superset of the values that a more precise
reasoning may yield.
If the set of those values is empty the result is $\None$.

\ExampleDef{Refining Constraints with a Sign Predicate}
The specification in \listingref{RefineConstraintBySign}
shows examples of how the constraints for \verb|y|
are refined for the \DIV{} operator via the sign predicate
that tests whether an integer is positive: $\lambda x.\ x > 0$.

\ASLListing{Refining Constraints with a Sign Predicate}{RefineConstraintBySign}{\typingtests/TypingRule.RefineConstraintBySign.asl}

\ProseParagraph
\OneApplies
\begin{itemize}
  \item \AllApplyCase{exact\_reduces\_to\_z}
  \begin{itemize}
    \item $\vc$ is an exact constraint for the expression $\ve$, that is, $\ConstraintExact(\ve)$;
    \item applying $\reducetozopt$ to $\ve$ in $\tenv$, in order to symbolically simplify $\ve$ to an integer,
          yields $\langle\vz\rangle$;
    \item $\vcopt$ is $\langle\vc\rangle$ if $\vp$ holds for $\vz$ and $\None$ otherwise.
  \end{itemize}

  \item \AllApplyCase{exact\_does\_not\_reduce\_to\_z}
  \begin{itemize}
    \item $\vc$ is an exact constraint for the expression $\ve$, that is, $\ConstraintExact(\ve)$;
    \item applying $\reducetozopt$ to $\ve$ in $\tenv$, in order to symbolically simplify $\ve$ to an integer,
          yields $\None$;
    \item $\vcopt$ is $\langle\vc\rangle$.
  \end{itemize}

  \item \AllApplyCase{range\_both\_reduce\_to\_z}
  \begin{itemize}
    \item $\vc$ is a range constraint for the expressions $\veone$ and $\vetwo$, that is, \\
          $\ConstraintRange(\veone, \vetwo)$;
    \item applying $\reducetozopt$ to $\veone$ in $\tenv$, in order to symbolically simplify $\veone$ to an integer,
          yields $\langle\vzone\rangle$;
    \item applying $\reducetozopt$ to $\vetwo$ in $\tenv$, in order to symbolically simplify $\vetwo$ to an integer,
          yields $\langle\vztwo\rangle$;
    \item \OneApplies{} (defining $\vcopt$)
    \begin{itemize}
      \item if $\vp$ is $\True$ for both $\vzone$ and $\vztwo$, define $\vcopt$ as $\langle\vc\rangle$;
      \item if $\vp$ is $\False$ for $\vzone$ and $\True$ for $\vztwo$, define $\vcopt$ as the optional range constraint
            where the bottom expression is the literal expression for $0$ if $\vp$ holds for $0$ and the literal expression for $1$ otherwise,
            and the top expression is $\vetwo$;
      \item if $\vp$ is $\True$ for $\vzone$ and $\False$ for $\vztwo$, define $\vcopt$ as the optional range constraint
            where the bottom expression is $\veone$ and the top expression is the literal expression for $0$ if $\vp$ holds for $0$
            and the literal expression for $-1$ otherwise;
      \item if $\vp$ is $\False$ for both $\vzone$ and $\vztwo$, define $\vcopt$ as $\None$.
    \end{itemize}
  \end{itemize}

  \item \AllApplyCase{only\_e1\_reduces\_to\_z}
  \begin{itemize}
    \item $\vc$ is a range constraint for the expressions $\veone$ and $\vetwo$, that is, \\
          $\ConstraintRange(\veone, \vetwo)$;
    \item applying $\reducetozopt$ to $\veone$ in $\tenv$, in order to symbolically simplify $\veone$ to an integer,
          yields $\langle\vzone\rangle$;
    \item applying $\reducetozopt$ to $\vetwo$ in $\tenv$, in order to symbolically simplify $\vetwo$ to an integer,
          yields $\None$;
    \item \OneApplies{} (defining $\vcopt$):
    \begin{itemize}
      \item if $\vp$ is $\True$ for $\vzone$, define $\vcopt$ as $\langle\vc\rangle$;
      \item if $\vp$ is $\False$ for $\vzone$, define $\vcopt$ as the optional range constraint with the bottom expression
            as the literal expression for $0$ if $\vp$ holds for $0$ and the literal expression for $1$ otherwise,
            and the top expression $\vetwo$.
    \end{itemize}
  \end{itemize}

  \item \AllApplyCase{only\_e2\_reduces\_to\_z}
  \begin{itemize}
    \item $\vc$ is a range constraint for the expressions $\veone$ and $\vetwo$, that is, \\
          $\ConstraintRange(\veone, \vetwo)$;
    \item applying $\reducetozopt$ to $\veone$ in $\tenv$, in order to symbolically simplify $\veone$ to an integer,
          yields $\None$;
    \item applying $\reducetozopt$ to $\vetwo$ in $\tenv$, in order to symbolically simplify $\vetwo$ to an integer,
          yields $\langle\vztwo\rangle$;
    \item One of the following applies (defining $\vcopt$):
    \begin{itemize}
      \item if $\vp$ is $\True$ for $\vztwo$, define $\vcopt$ as $\langle\vc\rangle$;
      \item if $\vp$ is $\False$ for $\vztwo$, define $\vcopt$ as the optional range constraint with the bottom expression
            $\veone$ and the top expression the literal expression for $0$ if $\vp$ holds for $0$ and the literal expression for $-1$ otherwise.
    \end{itemize}
  \end{itemize}

  \item \AllApplyCase{none\_reduce\_to\_z}
  \begin{itemize}
    \item $\vc$ is a range constraint for the expressions $\veone$ and $\vetwo$, that is, \\
          $\ConstraintRange(\veone, \vetwo)$;
    \item applying $\reducetozopt$ to $\veone$ in $\tenv$, in order to symbolically simplify $\veone$ to an integer,
          yields $\None$;
    \item applying $\reducetozopt$ to $\vetwo$ in $\tenv$, in order to symbolically simplify $\vetwo$ to an integer,
          yields $\None$;
    \item \Proseeqdef{$\vcopt$}{$\vc$}.
  \end{itemize}
\end{itemize}

\FormallyParagraph
\begin{mathpar}
\inferrule[exact\_reduces\_to\_z]{
  \reducetozopt(\tenv, \ve) \typearrow \langle\vz\rangle\\
  \vcopt \eqdef \choice{\vp(\vz)}{\langle\vc\rangle}{\None}
}{
  \refineconstraintbysign(\tenv, \vp, \overname{\ConstraintExact(\ve)}{\vc}) \typearrow \vcopt
}
\end{mathpar}

\begin{mathpar}
\inferrule[exact\_does\_not\_reduce\_to\_z]{
  \reducetozopt(\tenv, \ve) \typearrow \None
}{
  \refineconstraintbysign(\tenv, \vp, \overname{\ConstraintExact(\ve)}{\vc}) \typearrow \overname{\langle\vc\rangle}{\vcopt}
}
\end{mathpar}

\begin{mathpar}
\inferrule[range\_both\_reduce\_to\_z]{
  \reducetozopt(\tenv, \veone) \typearrow \langle\vzone\rangle\\
  \reducetozopt(\tenv, \vetwo) \typearrow \langle\vztwo\rangle\\
  {
    \begin{array}{c}
  \vcopt \eqdef \\ \wrappedline\ \begin{cases}
    \langle\vc\rangle& \text{if }\vp(\vzone) \land \vp(\vztwo)\\
    \langle\ConstraintRange(\choice{\vp(0)}{\ELInt{0}}{\ELInt{1}}, \vetwo)\rangle& \text{if }\neg\vp(\vzone) \land \vp(\vztwo)\\
    \langle\ConstraintRange(\veone, \choice{\vp(0)}{\ELInt{0}}{\ELInt{-1}})\rangle& \text{if }\vp(\vzone) \land \neg\vp(\vztwo)\\
    \None& \text{if }\neg\vp(\vzone) \land \neg\vp(\vztwo)\\
  \end{cases}
\end{array}
  }
}{
  \refineconstraintbysign(\tenv, \vp, \overname{\ConstraintRange(\veone, \vetwo)}{\vc}) \typearrow \vcopt
}
\end{mathpar}

\begin{mathpar}
\inferrule[only\_e1\_reduces\_to\_z]{
  \vc = \ConstraintRange(\veone, \vetwo)\\
  \reducetozopt(\tenv, \veone) \typearrow \langle\vzone\rangle\\
  \reducetozopt(\tenv, \vetwo) \typearrow \None\\
  {
    \begin{array}{c}
  \vcopt \eqdef \\ \wrappedline\ \begin{cases}
    \langle\vc\rangle& \text{if }\vp(\vzone)\\
    \langle\ConstraintRange(\choice{\vp(0)}{\ELInt{0}}{\ELInt{1}}, \vetwo)\rangle& \text{else}\\
  \end{cases}
\end{array}
  }
}{
  \refineconstraintbysign(\tenv, \vp, \vc) \typearrow \vcopt
}
\end{mathpar}

\begin{mathpar}
\inferrule[only\_e2\_reduces\_to\_z]{
  \vc = \ConstraintRange(\veone, \vetwo)\\
  \reducetozopt(\tenv, \veone) \typearrow \None\\
  \reducetozopt(\tenv, \vetwo) \typearrow \langle\vztwo\rangle\\
  {
    \begin{array}{c}
  \vcopt \eqdef \\ \wrappedline\ \begin{cases}
    \langle\vc\rangle& \text{if }\vp(\vztwo)\\
    \langle\ConstraintRange(\veone, \choice{\vp(0)}{\ELInt{0}}{\ELInt{-1}})\rangle& \text{else}\\
  \end{cases}
\end{array}
  }
}{
  \refineconstraintbysign(\tenv, \vp, \vc) \typearrow \vcopt
}
\end{mathpar}

\begin{mathpar}
\inferrule[none\_reduce\_to\_z]{
  \vc = \ConstraintRange(\veone, \vetwo)\\
  \reducetozopt(\tenv, \veone) \typearrow \None\\
  \reducetozopt(\tenv, \vetwo) \typearrow \None
}{
  \refineconstraintbysign(\tenv, \vp, \vc) \typearrow \overname{\vc}{\vcopt}
}
\end{mathpar}

\TypingRuleDef{ReduceToZOpt}
\hypertarget{def-reducetozopt}{}
The function
\[
\reducetozopt(\overname{\staticenvs}{\tenv} \aslsep \overname{\expr}{\ve})
\aslto \overname{\langle\Z\rangle}{\vzopt}
\]
returns an integer inside an optional if $\ve$ can be symbolically simplified into an integer in $\tenv$
and $\None$ otherwise.
The expression $\ve$ is assumed not to yield a \typingerrorterm{}
when applying $\normalize$ to it.

\ExampleDef{Reducing Expressions to Optional Integers}
The specification in \listingref{ReduceToZOpt} shows two examples of expressions
where one can be reduced into an integer constant, which is then returned in an optional,
and one that cannot be reduced into an integer constant, which yields $\None$.
\ASLListing{Reducing expressions to optional integers}{ReduceToZOpt}{\typingtests/TypingRule.ReduceToZOpt.asl}

\ProseParagraph
\OneApplies
\begin{itemize}
  \item \AllApplyCase{normalizes\_to\_z}
  \begin{itemize}
    \item symbolically simplifying $\ve$ in $\tenv$ via $\normalize$ yields a literal expression for the integer $\vz$;
    \item define $\vzopt$ as $\langle\vz\rangle$.
  \end{itemize}

  \item \AllApplyCase{does\_not\_normalize\_to\_z}
  \begin{itemize}
    \item symbolically simplifying $\ve$ in $\tenv$ via $\normalize$ yields an expression that is not an integer literal;
    \item define $\vzopt$ as $\None$.
  \end{itemize}
\end{itemize}

\FormallyParagraph
\begin{mathpar}
\inferrule[normalizes\_to\_z]{
  \normalize(\tenv, \ve) \typearrow \ELInt{\vz}
}{
  \reducetozopt(\tenv, \ve) \typearrow \overname{\langle\vz\rangle}{\vzopt}
}
\end{mathpar}

\begin{mathpar}
\inferrule[does\_not\_normalize\_to\_z]{
  \normalize(\tenv, \ve) \typearrow \vep\\
  \forall \vz\in\Z.\ \vep \neq \ELInt{\vz}
}{
  \reducetozopt(\tenv, \ve) \typearrow \overname{\None}{\vzopt}
}
\end{mathpar}

\TypingRuleDef{RefineConstraints}
\hypertarget{def-refineconstraints}{}
The function
\[
\refineconstraints\left(
  \begin{array}{c}
  \overname{\{\Over,\Under\}}{\vapprox} \aslsep\\
  \overname{\staticenvs}{\tenv} \aslsep\\
  \overname{\intconstraint\rightarrow\langle\intconstraint\rangle}{\vf} \aslsep\\
  \overname{\KleeneStar{\intconstraint}}{\cs}
  \end{array}
  \right)
\aslto
\begin{array}{ll}
\overname{\KleeneStar{\intconstraint}}{\newcs} & \cup\\
\{\CannotUnderapproximate, \CannotOverapproximate\}
\end{array}
\]
refines a list of constraints $\cs$ by applying the refinement function $\vf$ to each constraint and retaining the constraints
that do not refine to $\None$. The resulting list of constraints is given in $\newcs$.
%
If the list is empty, the result is either $\CannotUnderapproximate$ or $\CannotOverapproximate$,
based on $\vapprox$ (this function is invoked in the context of approximating lists of constraints).

\ExampleDef{Filtering Optional Constraints}
The specification in \listingref{RefineConstraintBySign}
shows how certain constraints are filtered out.
For example, $\AbbrevConstraintExact{\AbbrevEBinop{\DIV}{\AbbrevEVar{\vA}}{\ELInt{0}}}$.
Similarly, since both expressions in the constraint \verb|-4..-3| fail the sign predicate
used by $\refineconstraintbysign$, the resulting $\None$ is also filtered out.

\ProseParagraph
\OneApplies
\begin{itemize}
  \item \AllApplyCase{empty}
  \begin{itemize}
    \item $\cs$ is the empty list;
    \item $\newcs$ is the empty list.
  \end{itemize}

  \item \AllApply
  \begin{itemize}
    \item $\cs$ is the list with $\vc$ as its \head\ and $\csone$ as its \tail;
    \item applying $\vf$ to $\vc$ yields $\None$;
    \item applying $\refineconstraints$ to $\vapprox$, $\vf$ and $\csone$ yields $\csonep$;
    \item \OneApplies
    \begin{itemize}
      \item \AllApplyCase{non\_empty\_none\_precise}
      \begin{itemize}
        \item $\csonep$ is not the empty list;
        \item $\newcs$ is $\csonep$.
      \end{itemize}

      \item \AllApplyCase{non\_empty\_none\_approx}
      \begin{itemize}
        \item $\csonep$ is not the empty list;
        \item the result is $\CannotOverapproximate$ if $\vapprox$ is $\Over$ and $\CannotUnderapproximate$, otherwise.
      \end{itemize}
    \end{itemize}
  \end{itemize}

  \item \AllApplyCase{non\_empty\_some}
  \begin{itemize}
    \item $\cs$ is the list with $\vc$ as its \head\ and $\csone$ as its \tail;
    \item applying $\vf$ to $\vc$ yields $\langle\vcp\rangle$;
    \item applying $\refineconstraints$ to $\vapprox$, $\vf$ and $\csone$ yields $\csonep$;
    \item $\newcs$ is the list with $\vcp$ as its \head\ and $\csonep$ as its \tail.
  \end{itemize}
\end{itemize}

\FormallyParagraph
\begin{mathpar}
\inferrule[empty]{}{
  \refineconstraints(\vapprox, \tenv, \vf, \overname{\emptylist}{\cs}) \typearrow \overname{\emptylist}{\newcs}
}
\end{mathpar}

\begin{mathpar}
\inferrule[non\_empty\_none\_precise]{
  \vf(\vc) \typearrow \None\\
  \refineconstraints(\vapprox, \vf, \csone) \typearrow \csonep\\
  \csonep \neq \emptylist
}{
  \refineconstraints(\vapprox, \vf, \overname{[\vc]\concat \csone}{\cs}) \typearrow \overname{\csonep}{\newcs}
}
\end{mathpar}

\begin{mathpar}
\inferrule[non\_empty\_none\_approx]{
  \vf(\vc) \typearrow \None\\
  \refineconstraints(\vapprox, \vf, \csone) \typearrow \csonep\\
  \csonep = \emptylist
}{
  \refineconstraints(\vapprox, \vf, \overname{[\vc]\concat \csone}{\cs}) \typearrow \choice{\vapprox=\Over}{\CannotOverapproximate}{\CannotUnderapproximate}
}
\end{mathpar}

\begin{mathpar}
\inferrule[non\_empty\_some]{
  \vf(\vc) \typearrow \langle\vcp\rangle\\
  \refineconstraints(\vapprox, \vf, \csone) \typearrow \csonep\\
}{
  \refineconstraints(\vapprox, \vf, \overname{[\vc]\concat \csone}{\cs}) \typearrow \overname{[\vcp] \concat \csonep}{\newcs}
}
\end{mathpar}

\TypingRuleDef{RefineConstraintForDiv}
\hypertarget{def-refineconstraintfordiv}{}
The function
\[
\refineconstraintfordiv\left(
  \begin{array}{c}
  \overname{\{\Over,\Under\}}{\vapprox} \aslsep\\
  \overname{\binop}{\op} \aslsep\\
  \overname{\KleeneStar{\intconstraint}}{\cs}
  \end{array}
  \right) \aslto
  \begin{array}{ll}
  \overname{\KleeneStar{\intconstraint}}{\vres} & \cup\\
  \{\CannotUnderapproximate, \CannotOverapproximate\} &
  \end{array}
\]
filters the list of constraints $\cs$ for $\op$,
removing constraints that represents a division operation that will definitely fail
when $\op$ is the division operation.
%
If the list is empty, the result is either $\CannotUnderapproximate$ or $\CannotOverapproximate$,
based on $\vapprox$ (this function is invoked in the context of approximating lists of constraints).

See \ExampleRef{Filtering and Reducing Division-based Constraints}.

\ProseParagraph
\OneApplies
\begin{itemize}
  \item \AllApply
  \begin{itemize}
    \item $\op$ is $\DIV$;
    \item applying $\filterreduceconstraintdiv$ to each constraint $\cs[\vi]$, for each $\vi$ in $\listrange(\cs)$,
          yields the optional constraint $\vcopt_\vi$\ProseOrTypeError;
    \item define $\vres$ as the list made of constraints $\vcp_\vi$, for each $\vi$ in $\listrange(\cs)$
          such that $\vcopt_\vi = \langle\vcp_\vi\rangle$;
    \item \OneApplies
    \begin{itemize}
      \item \AllApplyCase{div\_non\_empty}
      \begin{itemize}
        \item $\vres$ is not the empty list;
      \end{itemize}

      \item \AllApplyCase{div\_empty}
      \begin{itemize}
        \item $\vres$ is the empty list;
        \item the result is $\CannotOverapproximate$ if $\vapprox$ is $\Over$ and $\CannotUnderapproximate$, otherwise.
      \end{itemize}
    \end{itemize}
  \end{itemize}

  \item \AllApplyCase{non\_div}
  \begin{itemize}
    \item $\op$ is not $\DIV$;
    \item define $\vres$ as $\cs$.
  \end{itemize}
\end{itemize}

\FormallyParagraph
\begin{mathpar}
\inferrule[div\_non\_empty]{
  \op = \DIV\\
  \vi\in\listrange(\cs): \filterreduceconstraintdiv(\cs[\vi]) \typearrow \vcopt_\vi \OrTypeError\\\\
  \vres \eqdef [\vi\in\listrange(\cs): \choice{\vcopt_\vi = \langle\vcp_\vi\rangle}{\vcp}{\epsilon}]\\
  \vres \neq \emptylist
}{
  \refineconstraintfordiv(\vapprox, \op, \cs) \typearrow \vres
}
\end{mathpar}

\begin{mathpar}
\inferrule[div\_empty]{
  \op = \DIV\\
  \vi\in\listrange(\cs): \filterreduceconstraintdiv(\cs[\vi]) \typearrow \vcopt_\vi \OrTypeError\\\\
  \vres \eqdef [\vi\in\listrange(\cs): \choice{\vcopt_\vi = \langle\vcp_\vi\rangle}{\vcp}{\epsilon}]\\
  \vres = \emptylist\\
}{
  \refineconstraintfordiv(\vapprox, \op, \cs) \typearrow \choice{\vapprox=\Over}{\CannotOverapproximate}{\CannotUnderapproximate}
}
\end{mathpar}

\begin{mathpar}
\inferrule[non\_div]{
  \op \neq \DIV
}{
  \refineconstraintfordiv(\vapprox, \op, \cs) \typearrow \overname{\cs}{\vres}
}
\end{mathpar}
\CodeSubsection{\RefineConstraintForDIVBegin}{\RefineConstraintForDIVEnd}{../types.ml}

\TypingRuleDef{FilterReduceConstraintDiv}
\hypertarget{def-filterreduceconstraintdiv}{}
The function
\[
\filterreduceconstraintdiv(
  \overname{\intconstraint}{\vc}) \aslto
  \overname{\langle\intconstraint\rangle}{\vcopt}
\]
returns $\None$ if $\vc$ is an exact constraint for a \binopexpressionterm{} for dividing two integer literals
where the denominator does not divide the numerator, and an optional containing $\vc$ otherwise.
The result is returned in $\vcopt$.
This is used to conservatively test whether $\vc$ would always fail dynamically.

\ExampleDef{Filtering and Reducing Division-based Constraints}
The specification in \listingref{FilterReduceConstraintDiv}
shows how different constraints formed by dividing the constraints of \verb|x| by \verb|2|
in order to annotate \verb|z|.
The resulting constraints are listed in the type annotation for \verb|z_typed|.
Specifically, notice the following:
\begin{itemize}
  \item the constraint for \verb|3| divided by \verb|2| is filtered away,
        since the result is not an integer,
  \item the constraint for \verb|7..10| divided by \verb|2|
        effectively rounds up $7/2$ into \verb|3|,
  \item the constraint for \verb|14..17| divided by \verb|2| effectively rounds down
        $17/2$ into \verb|16|, and
  \item the constraint for \verb|16..15| divided by \verb|2| is filtered away,
        since it is not a legal \rangeconstraintterm.
\end{itemize}
\ASLListing{Filtering and Reducing Division-based Constraints}{FilterReduceConstraintDiv}{\typingtests/TypingRule.FilterReduceConstraintDiv.asl}

\ProseParagraph
\OneApplies
\begin{itemize}
  \item \AllApplyCase{exact}
  \begin{itemize}
    \item $\vc$ is an exact constraint for the expression $\ve$, that is, $\ConstraintExact(\ve)$;
    \item applying $\getliteraldivopt$ to $\ve$ yields $\langle(\vzone, \vztwo)\rangle$\ProseTerminateAs{\None};
    \item define $\vcopt$ as follows:
    \begin{itemize}
      \item $\None$, if $\vztwo$ is positive and $\vztwo$ does not divide $\vzone$;
      \item $\langle\vc\rangle$, otherwise.
    \end{itemize}
  \end{itemize}

  \item \AllApplyCase{range}
  \begin{itemize}
    \item $\vc$ is a range constraint for $\veone$ and $\vetwo$, that is, $\ConstraintRange(\veone, \vetwo)$;
    \item applying $\getliteraldivopt$ to $\veone$ yields $\veoneopt$;
    \item define $\vzoneopt$ as follows:
    \begin{itemize}
      \item $\vzone$ divided by $\vztwo$ and rounded up, if $\veoneopt$ is $(\vzone, \vztwo)$ and $\vztwo$ is positive;
      \item $\None$, otherwise.
    \end{itemize}
    \item applying $\getliteraldivopt$ to $\vetwo$ yields $\vetwoopt$;
    \item define $\vztwoopt$ as follows:
    \begin{itemize}
      \item $\vzthree$ divided by $\vzfour$ and rounded down, if $\vetwoopt$ is $(\vzthree, \vzfour)$ and $\vzfour$ is positive;
      \item $\None$, otherwise.
    \end{itemize}
    \item define $\vcopt$ as follows:
    \begin{itemize}
      \item the exact constraint for the literal integer $\vzfive$, if $\vzoneopt$ is $\langle\vzfive\rangle$ and $\vztwoopt$ is $\langle\vzsix\rangle$ and $\vzfive$ is equal to $\vzsix$;
      \item the range constraint for the literal integer $\vzfive$ and $\vzsix$, if $\vzoneopt$ is $\langle\vzfive\rangle$ and $\vztwoopt$ is $\langle\vzsix\rangle$ and $\vzfive$ is less than $\vzsix$;
      \item $\None$, if $\vzoneopt$ is $\langle\vzfive\rangle$ and $\vztwoopt$ is $\langle\vzsix\rangle$ and $\vzfive$ is greater than $\vzsix$;
      \item the range constraint for the literal integer $\vzfive$ and $\vetwo$, if $\vzoneopt$ is $\langle\vzfive\rangle$ and $\vztwoopt$ is $\None$;
      \item the range constraint for $\veone$ and the literal integer $\vzsix$, if $\vzoneopt$ is $\None$ and $\vztwoopt$ is $\langle\vzsix\rangle$;
      \item $\vc$ if $\vzoneopt$ is $\None$ and $\vztwoopt$ is $\None$.
    \end{itemize}
  \end{itemize}
\end{itemize}

\FormallyParagraph
\begin{mathpar}
\inferrule[exact]{
  \getliteraldivopt(\ve) \typearrow \langle(\vzone, \vztwo)\rangle \terminateas \None\\\\
  {
    \vcopt \eqdef
    \begin{cases}
      \None & \text{if }\vztwo > 0 \land \frac{\vzone}{\vztwo} \not\in \Z\\
      \langle\vc\rangle & \text{else}
    \end{cases}
  }
}{
  \filterreduceconstraintdiv(\tenv, \overname{\ConstraintExact(\ve)}{\vc}) \typearrow \vcopt
}
\end{mathpar}

\begin{mathpar}
\inferrule[range]{
  \getliteraldivopt(\veone) \typearrow \veoneopt\\
  {
    \vzoneopt \eqdef
    \begin{cases}
      \left\lceil\frac{\vzone}{\vztwo}\right\rceil & \text{if }\veoneopt = \langle(\vzone, \vztwo)\rangle \land \vztwo > 0\\
      \None & \text{else}
    \end{cases}
  }\\
  \getliteraldivopt(\vetwo) \typearrow \vetwoopt\\
  {
    \vztwoopt \eqdef
    \begin{cases}
      \left\lfloor\frac{\vzthree}{\vzfour}\right\rfloor & \text{if }\vetwoopt = \langle(\vzthree, \vzfour)\rangle \land \vzfour > 0\\
      \None & \text{else}
    \end{cases}
  }\\
  {
    \vcopt \eqdef
    \begin{cases}
     \langle\AbbrevConstraintExact{\ELInt{\vzfive}}\rangle & \text{if }\vzoneopt = \langle\vzfive\rangle \land \vztwoopt = \langle\vzsix\rangle \land \vzfive=\vzsix\\
     \langle\AbbrevConstraintRange{\ELInt{\vzfive}}{\ELInt{\vzsix}}\rangle & \text{if }\vzoneopt = \langle\vzfive\rangle \land \vztwoopt = \langle\vzsix\rangle \land \vzfive<\vzsix\\
     \None & \text{if }\vzoneopt = \langle\vzfive\rangle \land \vztwoopt = \langle\vzsix\rangle \land \vzfive>\vzsix\\
     \langle\AbbrevConstraintRange{\ELInt{\vzfive}}{\vetwo}\rangle & \text{if }\vzoneopt = \langle\vzfive\rangle \land \vztwoopt = \None\\
     \langle\AbbrevConstraintRange{\veone}{\ELInt{\vzsix}}\rangle & \text{if }\vzoneopt = \None \land \vztwoopt = \langle\vzsix\rangle\\
     \langle\AbbrevConstraintRange{\veone}{\vetwo}\rangle & \text{if }\vzoneopt = \None \land \vztwoopt = \None\\
    \end{cases}
  }
}{
  \filterreduceconstraintdiv(\tenv, \overname{\ConstraintRange(\veone, \vetwo)}{\vc}) \typearrow \overname{\langle\vc\rangle}{\vcopt}
}
\end{mathpar}

\TypingRuleDef{GetLiteralDivOpt}
\hypertarget{def-getliteraldivopt}{}
The function
\[
\getliteraldivopt(\overname{\expr}{\ve}) \aslto \overname{\langle\Z\cartimes\Z\rangle}{\rangeopt}
\]
matches the expression $\ve$ to a binary operation expression over the division operation and two literal integer expressions.
If $\ve$ matches this pattern, the result $\rangeopt$ is an optional containing the pair of integers appearing in the operand
expressions. Otherwise, the result is $\None$.

\ExampleDef{Obtaining Division Expressions Optionally Resulting in Literals}
In \listingref{FilterReduceConstraintDiv},
the following are examples of $\getliteraldivopt$ applications that are used
for generating the constraints for \verb|z|:
\[
\begin{array}{rcl}
\getliteraldivopt(\AbbrevEBinop{\DIV}{\ELInt{14}}{\ELInt{2}}) &\typearrow& \some{(14, 2)}\\
\getliteraldivopt(\AbbrevEBinop{\DIV}{\ELInt{7}}{\ELInt{2}}) &\typearrow& \some{(7, 2)}\\
\getliteraldivopt(\AbbrevEBinop{\DIV}{\AbbrevEVar{\vA}}{\ELInt{2}}) &\typearrow& \None\\
\end{array}
\]

\ProseParagraph
The value $\rangeopt$ is $\langle(\vzone, \vztwo)\rangle$ if $\ve$ is a binary operation expression over the division operation
and two literal integer expressions for the integers $\vzone$ and $\vztwo$ and $\None$ otherwise.

\FormallyParagraph
\begin{mathpar}
\inferrule{
  \rangeopt \eqdef \choice{\ve = \AbbrevEBinop{\DIV}{\ELInt{\vzone}}{\ELInt{\vztwo}}}{\langle(\vzone, \vztwo)\rangle}{\None}
}{
  \getliteraldivopt(\ve) \typearrow \rangeopt
}
\end{mathpar}

\TypingRuleDef{ExplodeIntervals}
\hypertarget{def-explodeintervals}{}
The function
\[
\explodeintervals(\overname{\staticenvs}{\tenv} \aslsep \overname{\KleeneStar{\intconstraint}}{\cs})
\aslto \left( \overname{\KleeneStar{\intconstraint}}{\newcs} \aslsep \overname{\precisionlossindicator}{\vp} \right)
\]
applies $\explodedinterval$ to each constraint of $\cs$ in $\tenv$, and returns
a pair consisting of the list of exploded constraints in $\newcs$ and a
\Proseprecisionlossindicator{} $\vp$.

\ExampleDef{Exploding Intervals}
In \listingref{ExplodeIntervals}, annotating the constraints for \verb|y|
involves exploding each of the constraints for \verb|{C..0, 0..A}|,
which does not incur a precision loss.
On the other hand, annotating the constraints for \verb|z|
involves exploding each of the constraints for \verb|C..0, 0..B|
where exploding the first constraint \verb|C..0| does not incur precision loss,
but exploding the second constraint \verb|0..B| does incur precision loss,
and so overall the result is that there is precision loss.
\ASLListing{Exploding intervals}{ExplodeIntervals}{\typingtests/TypingRule.ExplodeIntervals.bad.asl}

\ProseParagraph
\OneApplies
\begin{itemize}
  \item \AllApplyCase{empty}
  \begin{itemize}
    \item $\cs$ is the empty list;
    \item $\newcs$ is the empty list;
    \item $\vp$ is $\PrecisionFull$.
  \end{itemize}

  \item \AllApplyCase{non\_empty}
  \begin{itemize}
    \item $\cs$ is the list with $\vc$ as its \head\ and $\csone$ as its \tail;
    \item applying $\explodeconstraint$ to $\vc$ in $\tenv$ yields $\vcp$ (a list of constraints);
    \item applying $\explodeintervals$ to $\csone$ in $\tenv$ yields $\csonep$;
    \item $\newcs$ is the concatenation of $\vcp$ and $\csonep$.
  \end{itemize}
\end{itemize}

\FormallyParagraph
\begin{mathpar}
\inferrule[empty]{}{
  \explodeintervals(\tenv, \overname{\emptylist}{\cs}) \typearrow \left( \overname{\emptylist}{\newcs} \aslsep \overname{\PrecisionFull}{\vp} \right)
}
\end{mathpar}

\begin{mathpar}
\inferrule[non\_empty]{
  \explodeconstraint(\tenv, \vc) \typearrow (\vcp, \vpone)\\
  \explodeintervals(\tenv, \csone) \typearrow (\csonep, \vptwo)\\
  \vp \eqdef \precisionjoin(\vpone, \vptwo)\\
  \newcs \eqdef \vcp \concat \csonep\\
}{
  \explodeintervals(\tenv, \overname{[\vc] \concat \csone}{\cs}) \typearrow (\newcs, \vp)
}
\end{mathpar}

\TypingRuleDef{ExplodeConstraint}
\hypertarget{def-explodeconstraint}{}
% Transliteration note: the implementation uses folding where explode_constraint
% is a folder. To simplify this function, we remove the input precision lost flag
% and join all precision loss flags in explode_intervals.
The function
\[
\begin{array}{r}
\explodeconstraint(
  \overname{\staticenvs}{\tenv} \aslsep
  \overname{\intconstraint}{\vc})
\aslto \\
(\overname{\KleeneStar{\intconstraint}}{\vcs}, \overname{\precisionlossindicator}{\newprec})
\end{array}
\]
given the \staticenvironmentterm{} $\tenv$ and the constraint $\vc$,
expands $\vc$ into the equivalent list of exact constraints if
$\vc$ matches an ascending range constraint that is not too large,
and the singleton list for $\vc$ otherwise.
The resulting list of constraints and the \Proseprecisionlossindicator{}
are in $\vcs$ and $\newprec$, respectively.

See \ExampleRef{Intervals too Large to Enumerate}.

\ProseParagraph
\OneApplies
\begin{itemize}
  \item \AllApplyCase{exact}
  \begin{itemize}
    \item $\vc$ is an exact constraint;
    \item $\vcs$ is the singleton list for $\vc$;
    \item \Proseeqdef{$\newprec$}{$\PrecisionFull$}.
  \end{itemize}

  \item \AllApplyCase{range\_reduced}
  \begin{itemize}
    \item $\vc$ is a range constraint for the expressions $\va$ and $\vb$;
    \item applying $\reducetozopt$ to $\va$ in $\tenv$ yields $\langle\vza\rangle$;
    \item applying $\reducetozopt$ to $\vb$ in $\tenv$ yields $\langle\vzb\rangle$;
    \item define $\explodedinterval$ as the list of exact constraints for each integer literal in the range starting
          at $\vza$ and ending at $\vzb$, inclusively, which is empty if $\vzb < \vza$;
    \item applying $\intervaltoolarge$ to $\vza$ and $\vzb$ yields $\vbtoolarge$;
    \item define $(\vcs, \newprec)$ as
          the singleton list for $\vc$ and $\PrecisionLost$ if \\
          $\vbtoolarge$ is $\True$
          and $\explodedinterval$ and $\PrecisionFull$ otherwise.
  \end{itemize}

  \item \AllApplyCase{range\_not\_reduced}
  \begin{itemize}
    \item $\vc$ is a range constraint for the expressions $\va$ and $\vb$;
    \item applying $\reducetozopt$ to $\va$ in $\tenv$ yields $\vzaopt$;
    \item applying $\reducetozopt$ to $\vb$ in $\tenv$ yields $\vzbopt$;
    \item at least one of $\vzaopt$ and $\vzbopt$ is $\None$;
    \item $\vcs$ is the singleton list for $\vc$;
    \item \Proseeqdef{$\newprec$}{$\PrecisionFull$}.
  \end{itemize}
\end{itemize}

\FormallyParagraph
\begin{mathpar}
\inferrule[exact]{
  \astlabel(\vc) = \ConstraintExact
}{
  \explodeconstraint(\tenv, \vc) \typearrow
  (\overname{[\vc]}{\vcs}, \overname{\PrecisionFull}{\newprec})
}
\end{mathpar}

\begin{mathpar}
\inferrule[range\_reduced]{
  \vc = \ConstraintRange(\va, \vb)\\
  \reducetozopt(\tenv, \va) \typearrow \langle\vza\rangle\\
  \reducetozopt(\tenv, \vb) \typearrow \langle\vzb\rangle\\
  \explodedinterval \eqdef [\vz \in \vza..\vzb: \ConstraintExact(\ELInt{\vz})]\\
  \intervaltoolarge(\vza, \vzb) \typearrow \vbtoolarge\\
  {
  (\vcs, \newprec) \eqdef
  \begin{cases}
    ([\vc], \PrecisionLost) & \text{ if } \vbtoolarge\\
    ([\explodedinterval], \PrecisionFull) & \text{ else}
  \end{cases}
  }
}{
  \explodeconstraint(\tenv, \vc) \typearrow (\vcs, \newprec)
}
\end{mathpar}

\begin{mathpar}
\inferrule[range\_not\_reduced]{
  \vc = \ConstraintRange(\va, \vb)\\
  \reducetozopt(\tenv, \va) \typearrow \vzaopt\\
  \reducetozopt(\tenv, \vb) \typearrow \vzbopt\\
  \vzaopt = \None \lor \vzbopt = \None
}{
  \explodeconstraint(\tenv, \vc) \typearrow
  (\overname{[\vc]}{\vcs}, \overname{\PrecisionFull}{\newprec})
}
\end{mathpar}

\TypingRuleDef{IntervalTooLarge}
\hypertarget{def-intervaltoolarge}{}
The function
\[
\intervaltoolarge(\overname{\Z}{\vzone} \aslsep \overname{\Z}{\vztwo}) \aslto \overname{\Bool}{\vb}
\]
determines whether the set of numbers between $\vzone$ and $\vztwo$, inclusive, contains more than $\maxexplodedintervalsize$
integers, yielding the result in $\vb$.

\ExampleDef{Intervals too Large to Enumerate}
In \listingref{IntervalTooLarge}, the size of the interval used to annotate
\verb|z| is less than $\maxexplodedintervalsize$.
\ASLListing{An enumerated interval}{IntervalTooLarge}{\typingtests/TypingRule.IntervalTooLarge.asl}

In \listingref{IntervalTooLarge-bad}, the size of the interval used to annotate
\verb|y| is just below $\maxexplodedintervalsize$, but the size of the interval
used to annotate \verb|z| is equal to $\maxexplodedintervalsize$, which is too
large to be enumerated.
\ASLListing{An interval too large to enumerate}{IntervalTooLarge-bad}{\typingtests/TypingRule.IntervalTooLarge.bad.asl}

\ProseParagraph
The value $\vb$ is $\True$ if and only if the absolute value of $\vzone-\vztwo$ is greater than $\maxexplodedintervalsize$.

\FormallyParagraph
\begin{mathpar}
\inferrule{}{
  \intervaltoolarge(\vzone, \vztwo) \typearrow \overname{\vztwo-\vzone > \maxexplodedintervalsize}{\vb}
}
\end{mathpar}

\TypingRuleDef{BinopIsExploding}
\hypertarget{def-binopisexploding}{}
The function
\[
\binopisexploding(\overname{\binop}{\op}) \aslto \overname{\Bool}{\vb}
\]
determines whether the binary operation $\op$ should lead to applying $\explodeintervals$
when the $\op$ is applied to a pair of constraint lists.
It is assumed that $\op$ is one of $\MUL$, $\SHL$, $\POW$, $\ADD$, $\DIV$, $\SUB$, $\MOD$, $\SHR$,
and $\DIVRM$.

See \ExampleRef{Annotating Constraints for Binary Operations}.

\ProseParagraph
The value $\vb$ is $\True$ if and only if $\op$ is one of $\MUL$, $\SHL$, and $\POW$.

\FormallyParagraph
\begin{mathpar}
\inferrule{}{
  \binopisexploding(\op) \typearrow \overname{\op \in \{\MUL, \SHL, \POW, \DIV, \DIVRM, \MOD, \SHR\}}{\vb}
}
\end{mathpar}

\TypingRuleDef{BitFieldsIncluded}
\hypertarget{def-bitfieldsincluded}{}
The predicate
\[
  \bitfieldsincluded(\overname{\staticenvs}{\tenv}, \overname{\KleeneStar{\bitfield}}{\bfsone} \aslsep \overname{\KleeneStar{\bitfield}}{\bfstwo})
  \aslto \overname{\Bool}{\vb} \cup \overname{\typeerror}{\TypeErrorConfig}
\]
tests whether the set of bit fields $\bfsone$ is included in the set of bit fields $\bfstwo$ in environment $\tenv$,
returning a \typingerrorterm{}, if one is detected.

\ExampleDef{Checking Whether all Bitfields are Included}
In \listingref{MemBfs},
all bitfields of the type \\
\verb|bits(8) {[0] flag, [7:1] data { [2:0] low }}|
are included in the type \\
\verb|FlaggedPacket|.

In \listingref{MemBfs-bad}, not all bitfields of the type \verb|bits(8) {[0] flag, [1] lsb}|
are included in the type \verb|FlaggedPacket|. Specifically, the bitfield \verb|lsb| is
not included in \verb|FlaggedPacket|.

\ProseParagraph
\AllApply
\begin{itemize}
  \item checking whether each field $\vbf$ in $\bfsone$ exists in $\bfstwo$ via $\membfs$ yields $\vb_\vbf$\ProseOrTypeError;
  \item the result --- $\vb$ --- is the conjunction of $\vb_\vbf$ for all bitfields $\vbf$ in $\bfsone$.
\end{itemize}

\FormallyParagraph
\begin{mathpar}
\inferrule{
  \vbf \in \bfsone: \membfs(\bfstwo, \vbf) \typearrow \vb_\vbf \OrTypeError\\\\
  \vbf \eqdef \bigwedge_{\bf \in \bfsone} \vb_\vbf
}{
  \bitfieldsincluded(\tenv, \bfsone, \bfstwo) \typearrow \vb
}
\end{mathpar}

\TypingRuleDef{MemBfs}
\hypertarget{def-membfs}{}
The function
\[
  \membfs(\overname{\staticenvs}{\tenv} \aslsep \overname{\KleenePlus{\bitfield}}{\bfstwo} \aslsep \overname{\bitfield}{\vbfone})
  \aslto \overname{\Bool}{\vb}
\]
checks whether the bitfield $\vbf$ exists in $\bfstwo$ in the context of $\tenv$, returning the result in $\vb$.

\ExampleDef{Checking Bitfield Membership}
The specification in \listingref{MemBfs} is well-typed.
Specifically, the bitfield \verb|flag| in the type
\verb|bits(8) {[0] flag, [7:1] data { [2:0] low }}|
is a member of the set of bitfields in the type \verb|FlaggedPacket|.
\ASLListing{No missing bitfields}{MemBfs}{\typingtests/TypingRule.MemBfs.asl}

In \listingref{MemBfs-bad}, the bitfield \verb|lsb| is missing from
the set of bitfields in the type \verb|FlaggedPacket|, which is why the specification
is ill-typed.
\ASLListing{A missing bitfield}{MemBfs-bad}{\typingtests/TypingRule.MemBfs.bad.asl}

\ProseParagraph
\OneApplies
\begin{itemize}
  \item \AllApplyCase{none}
  \begin{itemize}
    \item the name associated with the bitfield $\vbfone$ is $\name$;
    \item finding the bitfield associated with $\name$ in $\bfstwo$ yields $\None$;
    \item $\vb$ is $\False$.
  \end{itemize}

  \item \AllApplyCase{simple\_any}
  \begin{itemize}
    \item the name associated with the bitfield $\vbfone$ is $\name$;
    \item finding the bitfield associated with $\name$ in $\bfstwo$ yields $\vbftwo$;
    \item $\vbftwo$ is a simple bitfield;
    \item symbolically checking whether $\vbfone$ is equivalent to $\vbftwo$ in $\tenv$ yields $\vb$.
  \end{itemize}

  \item \AllApplyCase{nested\_simple}
  \begin{itemize}
    \item the name associated with the bitfield $\vbfone$ is $\name$;
    \item finding the bitfield associated with $\name$ in $\bfstwo$ yields $\vbftwo$;
    \item $\vbftwo$ is a nested bitfield with name $\nametwo$, slices $\slicestwo$, and bitfields $\bfstwop$;
    \item $\vbfone$ is a simple bitfield;
    \item symbolically checking whether $\vbfone$ is equivalent to $\vbftwo$ in $\tenv$ yields $\vb$.
  \end{itemize}

  \item \AllApplyCase{nested\_nested}
  \begin{itemize}
    \item the name associated with the bitfield $\vbfone$ is $\name$;
    \item finding the bitfield associated with $\name$ in $\bfstwo$ yields $\vbftwo$;
    \item $\vbftwo$ is a nested bitfield with name $\nametwo$, slices $\slicestwo$, and bitfields $\bfstwop$;
    \item $\vbfone$ is a nested bitfield with name $\nameone$, slices $\sliceone$, and $\bfsone$;
    \item $\vbone$ is true if and only if $\nameone$ is equal to $\nametwo$;
    \item symbolically equating the slices $\slicesone$ and $\slicestwo$ in $\tenv$ yields $\vbtwo$;
    \item checking $\bfsone$ is included in $\bfstwop$ in the context of $\tenv$ yields $\vbthree$;
    \item $\vb$ is defined as the conjunction of $\vbone$, $\vbtwo$, and $\vbthree$.
  \end{itemize}

  \item \AllApplyCase{nested\_typed}
  \begin{itemize}
    \item the name associated with the bitfield $\vbfone$ is $\name$;
    \item finding the bitfield associated with $\name$ in $\bfstwo$ yields $\vbftwo$;
    \item $\vbftwo$ is a nested bitfield with name $\nametwo$, slices $\slicestwo$, and bitfields $\bfstwop$;
    \item $\vbfone$ is a typed bitfield;
    \item $\vb$ is $\False$.
  \end{itemize}

  \item \AllApplyCase{typed\_simple}
  \begin{itemize}
    \item the name associated with the bitfield $\vbfone$ is $\name$;
    \item finding the bitfield associated with $\name$ in $\bfstwo$ yields $\vbftwo$;
    \item $\vbftwo$ is a typed bitfield with name $\nametwo$, slices $\slicestwo$, and type $\ttytwo$;
    \item $\vbfone$ is a simple bitfield;
    \item symbolically checking whether $\vbfone$ is equivalent to $\vbftwo$ in $\tenv$ yields $\vb$.
  \end{itemize}

  \item \AllApplyCase{typed\_nested}
  \begin{itemize}
    \item the name associated with the bitfield $\vbfone$ is $\name$;
    \item finding the bitfield associated with $\name$ in $\bfstwo$ yields $\vbftwo$;
    \item $\vbftwo$ is a typed bitfield with name $\nametwo$, slices $\slicestwo$, and type $\ttytwo$;
    \item $\vbfone$ is a nested bitfield;
    \item $\vb$ is $\False$.
  \end{itemize}

  \item \AllApplyCase{typed\_typed}
  \begin{itemize}
    \item the name associated with the bitfield $\vbfone$ is $\name$;
    \item finding the bitfield associated with $\name$ in $\bfstwo$ yields $\vbftwo$;
    \item $\vbftwo$ is a typed bitfield with name $\nametwo$, slices $\slicestwo$, and type $\ttytwo$;
    \item $\vbfone$ is a typed bitfield with name $\nameone$, slices $\slicesone$, and type $\ttyone$;
    \item $\vbone$ is true if and only if $\nameone$ is equal to $\nametwo$;
    \item symbolically equating the slices $\slicesone$ and $\slicestwo$ in $\tenv$ yields $\vbtwo$;
    \item checking whether $\ttyone$ subtypes $\ttytwo$ in $\tenv$ yields $\vbthree$;
    \item $\vb$ is defined as the conjunction of $\vbone$, $\vbtwo$, and $\vbthree$.
  \end{itemize}
\end{itemize}

\FormallyParagraph
\begin{mathpar}
\inferrule[none]{
  \bitfieldgetname(\vbfone) \typearrow \name\\
  \findbitfieldopt(\name, \bfstwo) \typearrow \None
}{
  \membfs(\tenv, \bfstwo, \vbfone) \typearrow \False
}
\and
\inferrule[simple\_any]{
  \bitfieldgetname(\vbf) \typearrow \name\\
  \findbitfieldopt(\name, \bfstwo) \typearrow \langle \vbftwo \rangle\\
  \astlabel(\vbftwo) = \BitFieldSimple\\
  \bitfieldsequal(\tenv, \vbfone, \vbftwo) \typearrow \vb
}{
  \membfs(\tenv, \bfstwo, \vbfone) \typearrow \vb
}
\end{mathpar}

\begin{mathpar}
\inferrule[nested\_simple]{
  \bitfieldgetname(\vbf) \typearrow \name\\
  \findbitfieldopt(\name, \bfstwo) \typearrow \langle \vbftwo \rangle\\
  \vbftwo = \BitFieldNested(\nametwo, \slicestwo, \bfstwop)\\
  \vbfone = \BitFieldSimple(\Ignore)\\
  \bitfieldsequal(\tenv, \vbfone, \vbftwo) \typearrow \vb
}{
  \membfs(\tenv, \bfstwo, \vbfone) \typearrow \overname{\False}{\vb}
}
\and
\inferrule[nested\_nested]{
  \bitfieldgetname(\vbf) \typearrow \name\\
  \findbitfieldopt(\name, \bfstwo) \typearrow \langle \vbftwo \rangle\\
  \vbftwo = \BitFieldNested(\nametwo, \slicestwo, \bfstwop)\\
  \vbfone = \BitFieldNested(\nameone, \slicesone, \bfsone)\\
  \vbone \eqdef \nameone = \nametwo\\
  \slicesequal(\tenv, \slicesone, \slicestwo) \typearrow \vbtwo\\
  \bitfieldsincluded(\tenv, \bfsone, \bfstwop) \typearrow \vbthree\\
  \vb \eqdef \vbone \land \vbtwo \land \vbthree
}{
  \membfs(\tenv, \bfstwo, \vbfone) \typearrow \vb
}
\and
\inferrule[nested\_typed]{
  \bitfieldgetname(\vbf) \typearrow \name\\
  \findbitfieldopt(\name, \bfstwo) \typearrow \langle \vbftwo \rangle\\
  \vbftwo = \BitFieldNested(\nametwo, \slicestwo, \bfstwop)\\
  \astlabel(\vbfone) = \BitFieldType
}{
  \membfs(\tenv, \bfstwo, \vbfone) \typearrow \overname{\False}{\vb}
}
\end{mathpar}

\begin{mathpar}
\inferrule[typed\_simple]{
  \bitfieldgetname(\vbf) \typearrow \name\\
  \findbitfieldopt(\name, \bfstwo) \typearrow \langle \vbftwo \rangle\\
  \vbftwo = \BitFieldType(\nametwo, \slicestwo, \ttytwo)\\
  \vbfone = \BitFieldSimple(\Ignore)\\
  \bitfieldsequal(\tenv, \vbfone, \vbftwo) \typearrow \vb
}{
  \membfs(\tenv, \bfstwo, \vbfone) \typearrow \vb
}
\and
\inferrule[typed\_nested]{
  \bitfieldgetname(\vbf) \typearrow \name\\
  \findbitfieldopt(\name, \bfstwo) \typearrow \langle \vbftwo \rangle\\
  \vbftwo = \BitFieldType(\nametwo, \slicestwo, \ttytwo)\\
  \astlabel(\vbfone) = \BitFieldNested
}{
  \membfs(\tenv, \bfstwo, \vbfone) \typearrow \overname{\False}{\vb}
}
\and
\inferrule[typed\_typed]{
  \bitfieldgetname(\vbf) \typearrow \name\\
  \findbitfieldopt(\name, \bfstwo) \typearrow \langle \vbftwo \rangle\\
  \vbftwo = \BitFieldType(\nametwo, \slicestwo, \ttytwo)\\
  \vbfone = \BitFieldType(\nameone, \slicesone, \ttyone)\\
  \vbone \eqdef \nameone = \nametwo\\
  \slicesequal(\tenv, \slicesone, \slicestwo) \typearrow \vbtwo\\
  \subtypesat(\tenv, \ttyone, \ttytwo) \typearrow \vbthree \OrTypeError\\\\
  \vb \eqdef \vbone \land \vbtwo \land \vbthree
}{
  \membfs(\tenv, \bfstwo, \vbfone) \typearrow \vb
}
\end{mathpar}

\hypertarget{def-checkstructurelabel}{}
\TypingRuleDef{CheckStructure}
The function
\[
  \checkstructurelabel(\overname{\staticenvs}{\tenv} \aslsep \overname{\ty}{\vt} \aslsep \overname{\ASTLabels}{\vl}) \aslto
  \{\True\} \cup \typeerror
\]
returns $\True$ is $\vt$ is has the \structure\ a of type corresponding to the AST label $\vl$ and a \typingerrorterm{} otherwise.

See \ExampleRef{The Structure of a Type}.

\ProseParagraph
\OneApplies
\begin{itemize}
  \item \AllApplyCase{okay}
  \begin{itemize}
    \item determining the \structure\ of $\vt$ yields $\vtp$\ProseOrTypeError;
    \item $\vtp$ has the label $\vl$;
    \item the result is $\True$;
  \end{itemize}

  \item \AllApplyCase{error}
  \begin{itemize}
    \item determining the \structure\ of $\vt$ yields $\vtp$\ProseOrTypeError;
    \item $\vtp$ does not have the label $\vl$;
    \item the result is a \typingerrorterm{} indicating that $\vt$ was expected to have the \structure\ of a type with the AST label $\vl$.
  \end{itemize}
\end{itemize}

\FormallyParagraph
\begin{mathpar}
\inferrule[okay]{
  \tstruct(\vt) \typearrow \vtp \OrTypeError\\\\
  \astlabel(\vtp) = \vl
}
{
  \checkstructurelabel(\tenv, \vt, \vl) \typearrow \True
}
\and
\inferrule[error]{
  \tstruct(\vt) \typearrow \vtp\\
  \astlabel(\vtp) \neq \vl
}
{
  \checkstructurelabel(\tenv, \vt, \vl) \typearrow \TypeErrorVal{\UnexpectedType}
}
\end{mathpar}

\TypingRuleDef{ToWellConstrained}
\hypertarget{def-towellconstrained}{}
The function
\[
  \towellconstrained(\overname{\ty}{\vt}) \aslto \overname{\ty}{\vtp}
\]
returns the \wellconstrainedversion\ of a type $\vt$ --- $\vtp$,
which converts \parameterizedintegertypes{} to \wellconstrainedintegertypes{},
and leaves all other types as are.

\ExampleDef{Converting Parameterized Integer Types to Well-constrainted Integer Types}
The following table shows examples of applying $\towellconstrained$ to various types:
\[
\begin{array}{rl}
\textbf{input type}           & \textbf{output type}\\
\hline
\TInt(\Parameterized(\vx))    & \TInt(\WellConstrained(\ConstraintExact(\EVar(\vx))))\\
\TInt(\unconstrainedinteger)  & \TInt(\unconstrainedinteger)\\
\TReal                        & \TReal\\
\end{array}
\]

\ProseParagraph
\OneApplies
\begin{itemize}
  \item \AllApplyCase{t\_int\_parameterized}
  \begin{itemize}
    \item $\vt$ is a \parameterizedintegertype\ for the variable $\vv$;
    \item $\vtp$ is the well-constrained integer constrained by the variable expression for $\vv$,
    that is, $\TInt(\WellConstrained(\ConstraintExact(\EVar(\vv))))$.
  \end{itemize}

  \item \AllApplyCase{t\_int\_other, other}
  \begin{itemize}
    \item $\vt$ is not a \parameterizedintegertype\ for the variable $\vv$;
    \item $\vtp$ is $\vt$.
  \end{itemize}
\end{itemize}

\FormallyParagraph
\begin{mathpar}
\inferrule[t\_int\_parameterized]{}
{
  \towellconstrained(\TInt(\Parameterized(\vv))) \typearrow\\ \TInt(\WellConstrained(\ConstraintExact(\EVar(\vv))))
}
\and
\inferrule[t\_int\_other]{
  \astlabel(\vi) \neq \Parameterized
}{
  \towellconstrained(\TInt(\vi)) \typearrow \vt
}
\and
\inferrule[other]{
  \astlabel(\vt) \neq \TInt
}{
  \towellconstrained(\vt) \typearrow \vt
}
\end{mathpar}

\TypingRuleDef{GetWellConstrainedStructure}
\hypertarget{def-getwellconstrainedstructure}{}
The function
\[
  \getwellconstrainedstructure(\overname{\staticenvs}{\tenv} \aslsep \overname{\ty}{\vt})
  \aslto \overname{\ty}{\vtp} \cup \overname{\typeerror}{\TypeErrorConfig}
\]
returns the \wellconstrainedstructure\ of a type $\vt$ in the \staticenvironmentterm{} $\tenv$ --- $\vtp$, which is defined as follows.
\ProseOtherwiseTypeError

\ExampleDef{Obtaining the Well-constrained Structure of a Type}
In \listingref{GetWellConstrainedStructure},
annotating the expression \verb|-N| involves obtaining the well-constrained structure of the
type of \verb|N|, which yields \\
$\TInt(\WellConstrained([\ConstraintExact(\EVar(\vN))]))$.
Similarly, the for expression \verb|-x|,
obtaining the well-constrained structure of the type for $\vx$ yields \\
$\unconstrainedinteger$.
\ASLListing{Obtaining the well constrained structure of a type}{GetWellConstrainedStructure}{\typingtests/TypingRule.GetWellConstrainedStructure.asl}

\ProseParagraph
\AllApply
\begin{itemize}
  \item the \structure\ of $\vt$ in $\tenv$ is $\vtone$\ProseOrTypeError;
  \item the well-constrained version of $\vtone$ is $\vtp$.
\end{itemize}

\FormallyParagraph
\begin{mathpar}
\inferrule{
  \tstruct(\tenv, \vt) \typearrow \vtone \OrTypeError\\\\
  \towellconstrained(\vtone) \typearrow \vtp
}{
  \getwellconstrainedstructure(\tenv, \vt) \typearrow \vtp
}
\end{mathpar}

\TypingRuleDef{GetBitvectorWidth}
\hypertarget{def-getbitvectorwidth}{}
The function
\[
  \getbitvectorwidth(\overname{\staticenvs}{\tenv} \aslsep \overname{\ty}{\vt}) \aslto
  \overname{\expr}{\ve} \cup \overname{\typeerror}{\TypeErrorConfig}
\]
returns the expression $\ve$, which represents the width of the bitvector type $\vt$
in the \staticenvironmentterm{} $\tenv$.
\ProseOtherwiseTypeError

See \ExampleRef{Obtaining the Integer Corresponding to a Bitvector Width}.

\ProseParagraph
\OneApplies
\begin{itemize}
  \item \AllApplyCase{okay}
  \begin{itemize}
    \item obtaining the \structure\ of $\vt$ in $\tenv$ yields a bitvector type with width expression $\ve$,
          that is, $\TBits(\ve, \Ignore)$\ProseOrTypeError;
    \item the result is $\ve$.
  \end{itemize}

  \item \AllApplyCase{error}
  \begin{itemize}
    \item obtaining the \structure\ of $\vt$ in $\tenv$ yields a type that is not a bitvector type;
    \item the result is a \typingerrorterm{} indicating that a bitvector type was expected.
  \end{itemize}
\end{itemize}

\FormallyParagraph
\begin{mathpar}
\inferrule[okay]{
  \tstruct(\tenv, \vt) \typearrow \TBits(\ve, \Ignore) \OrTypeError
}{
  \getbitvectorwidth(\tenv, \vt) \typearrow \ve
}
\and
\inferrule[error]{
  \tstruct(\tenv, \vt) \typearrow \vtp\\
  \astlabel(\vtp) \neq \TBits
}{
  \getbitvectorwidth(\tenv, \vt) \typearrow \TypeErrorVal{\UnexpectedType}
}
\end{mathpar}
\CodeSubsection{\GetBitvectorWidthBegin}{\GetBitvectorWidthEnd}{../Typing.ml}

\TypingRuleDef{GetBitvectorConstWidth}
\hypertarget{def-getbitvectorconstwidth}{}
The function
\[
  \getbitvectorconstwidth(\overname{\staticenvs}{\tenv} \aslsep \overname{\ty}{\vt}) \aslto
  \overname{\N}{\vw} \cup \overname{\typeerror}{\TypeErrorConfig}
\]
returns the natural number $\vw$, which represents the width of the bitvector type $\vt$
in the \staticenvironmentterm{} $\tenv$.
\ProseOtherwiseTypeError

\ExampleDef{Obtaining the Integer Corresponding to a Bitvector Width}
In \listingref{lesetfields}, annotating the \assignableexpression{} \verb|x.[status, time, data]|
requires obtaining the widths of the fields \verb|status|, \verb|time|, and \verb|data|
whose corresponding types are \verb|bit|, \verb|bits(16)|, and \verb|bits(8)|,
which yield $1$, $16$, and $8$, respectively.

\ProseParagraph
\AllApply
\begin{itemize}
  \item applying $\getbitvectorwidth$ to $\vt$ in $\tenv$ yields $\ewidth$\ProseOrTypeError;
  \item \Prosestaticeval{$\tenv$}{$\ewidth$}{integer for $\vw$}\ProseOrTypeError.
\end{itemize}

\FormallyParagraph
\begin{mathpar}
\inferrule{
  \getbitvectorwidth(\tenv, \vt) \typearrow \ewidth \OrTypeError\\\\
  \staticeval(\tenv, \ewidth) \typearrow \LInt(\vw) \OrTypeError
}{
  \getbitvectorconstwidth(\tenv, \vt) \typearrow \vw
}
\end{mathpar}
\CodeSubsection{\GetBitvectorConstWidthBegin}{\GetBitvectorConstWidthEnd}{../Typing.ml}

\TypingRuleDef{CheckBitsEqualWidth}
\hypertarget{def-checkbitsequalwidth}{}
The function
\[
  \checkbitsequalwidth(
    \overname{\staticenvs}{\tenv} \aslsep
    \overname{\ty}{\vtone} \aslsep
    \overname{\ty}{\vttwo}) \aslto
  \{\True\} \cup \typeerror
\]
tests whether the types $\vtone$ and $\vttwo$ are bitvector types of the same width.
If the answer is positive, the result is $\True$. \ProseOtherwiseTypeError

\ExampleDef{Ensuring That Two Bitvectors Have Equal Width}
In \listingref{CheckBitsEqualWidth}, annotating the bitfield \verb|info|
requires checking that its declared width given by \verb|bits(4)| is equal to
the width defined by the corresponding slice \verb|[6:3]|.

In addition, annotating the expression \verb|x AND y| requires checking that
the widths of \verb|x| and \verb|y| are equal.
\ASLListing{Ensuring that two bitvectors Have equal width}{CheckBitsEqualWidth}{\typingtests/TypingRule.CheckBitsEqualWidth.asl}

The specification in \listingref{CheckBitsEqualWidth-bad}
is ill-typed, since \verb|M| and \verb|N| are not necessarily equal, even though they have the same type.
\ASLListing{Two bitvectors of possibly different widths}{CheckBitsEqualWidth-bad}{\typingtests/TypingRule.CheckBitsEqualWidth.bad.asl}

The specification in \listingref{CheckBitsEqualWidth-bad2} is ill-typed for a similar reason.
\ASLListing{Two bitvectors of possibly different widths}{CheckBitsEqualWidth-bad2}{\typingtests/TypingRule.CheckBitsEqualWidth.bad2.asl}

\ProseParagraph
\AllApply
\begin{itemize}
  \item obtaining the width of $\vtone$ in $\tenv$ (via $\getbitvectorwidth$) yields the expression $\vn$\ProseOrTypeError;
  \item obtaining the width of $\vttwo$ in $\tenv$ (via $\getbitvectorwidth$) yields the expression $\vm$\ProseOrTypeError;
  \item symbolically checking whether the bitwidth expressions $\vn$ and $\vm$ are equal (via $\bitwidthequal$) yields $\vb$;
  \item checking whether $\vb$ is $\True$ yields $\True$\ProseTerminateAs{\UnexpectedType};
  \item the result is $\True$.
\end{itemize}

\FormallyParagraph
\begin{mathpar}
\inferrule{
  \getbitvectorwidth(\tenv, \vtone) \typearrow \vn \OrTypeError\\\\
  \getbitvectorwidth(\tenv, \vttwo) \typearrow \vm \OrTypeError\\\\
  \bitwidthequal(\tenv, \vn, \vm) \typearrow \vb\\
  \checktrans{\vb}{\UnexpectedType} \checktransarrow \True \OrTypeError
}{
  \checkbitsequalwidth(\tenv, \vtone, \vttwo) \typearrow \True
}
\end{mathpar}

\TypingRuleDef{PrecisionJoin}
\hypertarget{def-precisionjoin}{}
The function
\[
    \precisionjoin(
      \overname{\vpone}{\precisionlossindicator} \aslsep
      \overname{\vpone}{\precisionlossindicator}
    )
    \aslto
    \overname{\vp}{\precisionlossindicator}
\]
returns the \Proseprecisionlossindicator{} $\vp$, denoting whether $\vpone$ or
$\vptwo$ denote a precision loss.

\ExampleDef{Precision join}
In \listingref{precisionjoin}, the statement \verb|var b = (a * a) + 2;| is
forbidden because it tries to declare a type with a precision loss (see
\TypingRuleRef{LDVar}).
The expression \verb|a * a| has a type that results in a precision loss (see
\TypingRuleRef{AnnotateConstraintBinop}).
The typing rule \TypingRuleRef{PrecisionJoin} is called by
\TypingRuleRef{ApplyBinopTypes} to compute the precision of the type of the
expression \verb|(a * a) + 2|. Because the type of \verb|(a * a)| denotes a
precision lost, the type of \verb|(a * a) + 2| also denotes a precision lost.
\ASLListing{Precision join}{precisionjoin}{\typingtests/TypingRule.PrecisionJoin.asl}

\ProseParagraph
\OneApplies
\begin{itemize}
  \item \AllApplyCase{Loss}
    \begin{itemize}
      \item $\vpone$ is $\PrecisionLost$ or $\vptwo$ is $\PrecisionLost$;
      \item $\vp$ is $\PrecisionLost$;
    \end{itemize}
  \item \AllApplyCase{Full}
    \begin{itemize}
      \item $\vpone$ is $\PrecisionFull$ and $\vptwo$ is $\PrecisionFull$;
      \item $\vp$ is $\PrecisionFull$;
    \end{itemize}
\end{itemize}

\FormallyParagraph
\begin{mathpar}
  \inferrule[Loss]{
    \vpone = \PrecisionLost \lor
    \vptwo = \PrecisionLost
  }{
    \vp = \PrecisionLost
  }
  \and
  \inferrule[Full]{
    \vpone = \PrecisionFull \\
    \vptwo = \PrecisionFull
  }{
    \vp = \PrecisionFull
  }
\end{mathpar}

\section{Base Values\label{sec:BaseValues}}
\hypertarget{def-basevalueterm}{}
Each type, with the exceptions stated below, have a \basevalueterm,
which is used to initialize storage elements (either local of global),
if an initializer is not supplied.

\RequirementDef{NoBaseValue}
The following types do not have a \basevalueterm{}:
\begin{itemize}
    \item \parameterizedintegertypesterm{};
    \item a \wellconstrainedintegertypeterm{} whose list of constraints
        represents the empty set;
    \item a \bitvectortypeterm{} whose length is negative.
\end{itemize}

\identi{WVQZ}
Subprogram parameters can be parameterized integers, and since they will be initialized by their
invocation, there is no need to have a \basevalueterm{} for them.

\ExampleDef{Base Values}
\listingref{base-values} shows a specification with examples of well-typed \basevalueterm{}
for various types, followed by the output to the console.
\ASLListing{Well-typed Base Values}{base-values}{\typingtests/TypingRule.BaseValue.asl}
% CONSOLE_BEGIN aslref \typingtests/TypingRule.BaseValue.asl
\begin{Verbatim}[fontsize=\footnotesize, frame=single]
global_base = 0, unconstrained_integer_base = 0, constrained_integer_base = -3
bool_base = FALSE, real_base = 0, string_base = , enumeration_base = RED
bits_base = 0x00
tuple_base = (0, -3, RED)
record_base      = {data=0x00, time=0, flag=FALSE}
record_base_init = {data=0x00, time=0, flag=FALSE}
exception_base = {msg=}
integer_array_base = [[0, 0, 0, 0]]
enumeration_array_base = [[RED=0, GREEN=0, BLUE=0]]
\end{Verbatim}
% CONSOLE_END

\listingref{base-values-parameterised} shows a specification that relies on the base value of a \bitvectortypeterm{} whose width is a \parameterizedintegertypeterm{}.
\ASLListing{Base Value for Parameterized Bitvector Width}{base-values-parameterised}{\typingtests/TypingRule.BaseValue.parameterized.asl}

\ExampleDef{Types Without Base Value}
\listingref{base-values-bad-negative-width} shows an ill-typed specification
where the width of a bitvector is negative.
\ASLListing{No Base Value for Bitvectors of Negative Width}{base-values-bad-negative-width}{\typingtests/TypingRule.BaseValue.bad_negative_width.asl}

\listingref{base-values-bad-empty-type} shows an ill-typed specification
where the constraint \verb|5..0| represents an empty set.
Therefore, the domain of values for the type \verb|integer{5..0}| is empty,
which negates the possibility of having a \basevalueterm.
\ASLListing{No Base Value for an Empty Integer Type}{base-values-bad-empty-type}{\typingtests/TypingRule.BaseValue.bad_empty.asl}

\hypertarget{def-basevalue}{}
The function
\[
\basevalue(\overname{\staticenvs}{\tenv} \aslsep \overname{\ty}{\vt}) \aslto
\overname{\expr}{\veinit} \cup \overname{\typeerror}{\TypeErrorConfig}
\]
returns the expression $\veinit$ which can be used to initialize a storage element
of type $\vt$ in the \staticenvironmentterm{} $\tenv$.
\ProseOtherwiseTypeError

\TypingRuleDef{BaseValue}
See \ExampleRef{Base Values} and \ExampleRef{Types Without Base Value}.

\ProseParagraph
\OneApplies
\begin{itemize}
    \item \AllApplyCase{t\_bool} \identr{CPCK}
    \begin{itemize}
        \item $\vt$ is the Boolean type;
        \item $\veinit$ is the literal expression for $\False$.
    \end{itemize}

    \item \AllApplyCase{t\_bits\_static} \identr{ZVPT}
    \begin{itemize}
        \item $\vt$ is the bitvector type with width expression $\ve$;
        \item applying $\reducetozopt$ to $\ve$ in $\tenv$ yields $\vzopt$;
        \item $\vzopt$ is not $\None$;
        \item view $\vzopt$ as the singleton integer $\length$;
        \item checking that $\length$ is greater or equal to $0$ yields $\True$\ProseTerminateAs{\NoBaseValue};
        \item $\veinit$ is the literal expression for a bitvector made of a sequence of $\length$ values of $0$.
    \end{itemize}

    \item \AllApplyCase{t\_bits\_non\_static}
    \begin{itemize}
        \item $\vt$ is the bitvector type with width expression $\ve$;
        \item applying $\reducetozopt$ to $\ve$ in $\tenv$ yields $\vzopt$;
        \item $\vzopt$ is $\None$;
        \item $\veinit$ is the literal expression for a slice of the integer literal for $0$, with start position $0$ and length $\ve$.
    \end{itemize}

    \item \AllApplyCase{t\_enum} \identr{LCCN}
    \begin{itemize}
        \item $\vt$ is the \enumerationtypeterm{} with a list of labels where $\name$ as its \head;
        \item $\name$ is bound to the literal $\vl$ by the $\constantvalues$ in the \globalstaticenvironmentterm{} of $\tenv$;
        \item $\veinit$ is the literal expression for $\vl$, that is, $\eliteral{\vl}$.
    \end{itemize}

    \item \AllApplyCase{t\_int\_unconstrained} \identr{NJDZ}
    \begin{itemize}
        \item $\vt$ is the \unconstrainedintegertypeterm;
        \item $\veinit$ is the literal expression for $0$, that is, $\ELiteral(\LInt(0))$.
    \end{itemize}

    \item \AllApplyCase{t\_int\_parameterized} \identr{QGGH}
    \begin{itemize}
        \item $\vt$ is the \parameterizedintegertypeterm;
        \item the result is a \typingerrorterm{} indicating the lack of a statically known base value.
    \end{itemize}

    \item \AllApplyCase{t\_int\_wellconstrained} \identr{CFTD}
    \begin{itemize}
        \item $\vt$ is the \wellconstrainedintegertypeterm\ with a list of constraints $\cs$;
        \item define $\vzminlist$ as the concatenation of lists obtained for each
              constraint $\cs[\vi]$ in $\tenv$, for each $\vi\in\listrange(\cs)$, via $\constraintabsmin$;
        \item checking whether $\vzminlist$ is empty yields $\True$\ProseOrTypeError{\NoBaseValue};
        \item determining the minimal absolute integer in $\vzminlist$ via $\listminabs$ yields $\vzmin$;
        \item $\veinit$ is the integer literal expression for $\vzmin$.
    \end{itemize}

    \item \AllApplyCase{t\_named}
    \begin{itemize}
        \item $\vt$ is the \namedtype\ for $\id$;
        \item obtaining the \underlyingtypeterm\ for $\id$ in $\tenv$ yields $\vtp$\ProseOrTypeError;
        \item applying $\basevalue$ to $\vtp$ in $\tenv$ yields $\veinit$\ProseOrTypeError.
    \end{itemize}

    \item \AllApplyCase{t\_real} \identr{GYCG}
    \begin{itemize}
        \item $\vt$ is the \realtypeterm{};
        \item $\veinit$ is the real literal expression for $0$.
    \end{itemize}

    \item \AllApplyCase{structured} \identr{MBRM} \identr{SVJB}
    \begin{itemize}
        \item $\vt$ is a \structuredtypeterm\ with list of fields $\fields$;
        \item applying $\basevalue$ to $\vtefield$ in $\tenv$ for each $(\name, \vtefield)$ in $\fields$
              yields $\ve_\name$\ProseOrTypeError;
        \item $\veinit$ is the record construction expression assigning each field $\name$
              where $(\name, \vtefield)$ is an element of $\fields$ to $\vtefield$, that is, \\
              $\ERecord((\name, \vtefield) \in \fields: (\name, \ve_\name))$.
    \end{itemize}

    \item \AllApplyCase{t\_string} \identr{WKCY}
    \begin{itemize}
        \item $\vt$ is the \stringtypeterm{};
        \item $\veinit$ is the string literal expression for the empty list of characters.
    \end{itemize}

    \item \AllApplyCase{t\_tuple} \identr{QWSQ}
    \begin{itemize}
        \item $\vt$ is the \tupletypeterm{} over the list of types $\vt_{1..k}$, that is, $\TTuple(\vt_{1..k})$;
        \item applying $\basevalue$ to each type $\vt_\vi$ in $\tenv$ for $\vi=1..k$; yields the list of expressions $\ve_{1..k}$;
        \item $\veinit$ is the tuple expression $\ETuple(\ve_{1..k})$.
    \end{itemize}

    \item \AllApplyCase{t\_array\_enum}
    \begin{itemize}
        \item $\vt$ is the enumerated array type over for the enumeration $\venum$ and labels $\vlabels$ and element type $\tty$,
              that is, $\TArray(\ArrayLengthEnum(\venum, \vlabels), \tty)$ ;
        \item applying $\basevalue$ to $\tty$ in $\tenv$ yields the expression $\vvalue$\ProseOrTypeError;
        \item $\veinit$ is the array construction expression for an enumerated array with labels $\vlabels$ and initial value $\vvalue$,
              that is, $\EEnumArray\{\EArrayLabels: \vlabels, \EArrayValue: \vvalue\}$.
    \end{itemize}

    \item \AllApplyCase{t\_array\_expr} \identr{WGVR}
    \begin{itemize}
        \item $\vt$ is the array type over an integer index expression $\vlength$ and element type $\tty$, that is,
              $\TArray(\ArrayLengthExpr(\vlength), \tty)$ ;
        \item applying $\basevalue$ to $\tty$ in $\tenv$ yields the expression $\vvalue$\ProseOrTypeError;
        \item $\veinit$ is the array construction expression with length expression $\vlength$ and value expression $\vvalue$,
              that is, $\EArray\{\EArrayLength: \length, \EArrayValue: \vvalue\}$.
    \end{itemize}
\end{itemize}

\FormallyParagraph
\begin{mathpar}
\inferrule[t\_bool]{}{
    \basevalue(\tenv, \overname{\TBool}{\vt}) \typearrow \overname{\ELiteral(\LBool(\False))}{\veinit}
}
\end{mathpar}

\begin{mathpar}
\inferrule[t\_bits\_static]{
    \reducetozopt(\tenv, \ve) \typearrow \vzopt\\
    \vzopt \neq \None\\\\
    \vzopt \eqname \some{\length}\\
    \checktrans{\length \geq 0}{\NoBaseValue} \checktransarrow \True\OrTypeError
}{
    \basevalue(\tenv, \overname{\TBits(\ve, \Ignore)}{\vt}) \typearrow \overname{\ELiteral(\LBitvector(i=1..\length: 0))}{\veinit}
}
\end{mathpar}

\begin{mathpar}
\inferrule[t\_bits\_non\_static]{
    \reducetozopt(\tenv, \ve) \typearrow \vzopt\\
    \vzopt = \None \\\\
    \veinit \eqdef \ESlice (\ELInt{0}, [\SliceLength(\ELInt{0}, \ve)])
}{
    \basevalue(\tenv, \overname{\TBits(\ve, \Ignore)}{\vt}) \typearrow \veinit
}
\end{mathpar}

\begin{mathpar}
\inferrule[t\_enum]{%
    \lookupconstant(\tenv, \name) \typearrow \vl
}{%
    \basevalue(\tenv, \overname{\TEnum(\name \concat \Ignore)}{\vt}) \typearrow \overname{\ELiteral(\vl)}{\veinit}
}
\end{mathpar}

\begin{mathpar}
\inferrule[t\_int\_unconstrained]{}{
    \basevalue(\tenv, \overname{\unconstrainedinteger}{\vt}) \typearrow \overname{\ELiteral(\LInt(0))}{\veinit}
}
\end{mathpar}

\begin{mathpar}
\inferrule[t\_int\_parameterized]{}{
    \basevalue(\tenv, \overname{\TInt(\Parameterized(\id))}{\vt}) \typearrow \TypeErrorVal{\NoBaseValue}
}
\end{mathpar}

\begin{mathpar}
\inferrule[t\_int\_wellconstrained]{
    \cs \eqname \vc_{1..k}\\
    \vzminlist \eqdef \constraintabsmin(\tenv, \vc_1) \concat \ldots \concat \constraintabsmin(\tenv, \vc_k)\\
    \checktrans{\vzminlist \neq \emptyset}{\NoBaseValue} \typearrow \True \OrTypeError\\\\
    \listminabs(\vzminlist) \typearrow \vzmin
}{
    \basevalue(\tenv, \overname{\TInt(\WellConstrained(\cs))}{\vt}) \typearrow \overname{\ELiteral(\LInt(\vzmin))}{\veinit}
}
\end{mathpar}

\begin{mathpar}
\inferrule[t\_named]{
    \makeanonymous(\tenv, \TNamed(\id)) \typearrow \vtp \OrTypeError\\\\
    \basevalue(\tenv, \vtp) \typearrow \veinit \OrTypeError
}{
    \basevalue(\tenv, \overname{\TNamed(\id)}{\vt}) \typearrow \veinit
}
\end{mathpar}

\begin{mathpar}
\inferrule[t\_real]{}{
    \basevalue(\tenv, \overname{\TReal}{\vt}) \typearrow \overname{\ELiteral(\LReal(0))}{\veinit}
}
\end{mathpar}

\begin{mathpar}
\inferrule[structured]{
    \isstructured(\vt) \typearrow \True\\
    \vt \eqname L(\fields)\\
    (\name, \vtefield) \in \fields: \basevalue(\tenv, \vtefield) \typearrow \ve_\name \OrTypeError
}{
    \basevalue(\tenv, \vt) \typearrow \overname{\ERecord((\name, \vtefield) \in \fields: (\name, \ve_\name))}{\veinit}
}
\end{mathpar}

\begin{mathpar}
\inferrule[t\_string]{}{
    \basevalue(\tenv, \overname{\TString}{\vt}) \typearrow \overname{\ELiteral(\LString(\emptylist))}{\veinit}
}
\end{mathpar}

\begin{mathpar}
\inferrule[t\_tuple]{
    \vi=1..k: \basevalue(\tenv, \vt_\vi) \typearrow \ve_\vi \OrTypeError
}{
    \basevalue(\tenv, \overname{\TTuple}{\vt_{1..k}}) \typearrow \overname{\ETuple(\ve_{1..k})}{\veinit}
}
\end{mathpar}

\begin{mathpar}
\inferrule[t\_array\_enum]{
    \basevalue(\tenv, \tty) \typearrow \vvalue \OrTypeError
}{
    {
        \begin{array}{r}
            \basevalue(\tenv, \overname{\TArray(\ArrayLengthEnum(\venum, \vlabels), \tty)}{\vt}) \typearrow \\
            \overname{\EEnumArray\{\EArrayLabels: \vlabels, \EArrayValue: \vvalue\}}{\veinit}
        \end{array}
    }
}
\end{mathpar}

\begin{mathpar}
\inferrule[t\_array\_expr]{
    \basevalue(\tenv, \tty) \typearrow \vvalue \OrTypeError
}{
    {
        \begin{array}{r}
            \basevalue(\tenv, \overname{\TArray(\ArrayLengthExpr(\length), \tty)}{\vt}) \typearrow\\
            \overname{\EArray\{\EArrayLength: \length, \EArrayValue: \vvalue\}}{\veinit}
        \end{array}
    }
}
\end{mathpar}

\TypingRuleDef{ConstraintAbsMin}
\hypertarget{def-constraintabsmin}{}
The function
\[
    \constraintabsmin(\overname{\staticenvs}{\tenv} \aslsep \overname{\intconstraint}{\vc}) \aslto
    \overname{\KleeneStar{\Z}}{\vzs}
    \cup \overname{\typeerror}{\TypeErrorVal{\NoBaseValue}}
\]
returns a single element list containing the integer closest to $0$ that satisfies the constraint $\vc$ in $\tenv$, if one exists,
and an empty list if the constraint represents an empty set.
Otherwise, the result is $\TypeErrorVal{\NoBaseValue}$.

\ExampleDef{Minimal Absolute Value in a Constraint List}
The minimal absolute value of \verb|{7, -2}| is \verb|-2|.\\
%
The minimal absolute value of \verb|{2, -2}| is \verb|2|.\\
%
The minimal absolute value of \verb|{-2..2, 5}| is \verb|0|.

\ProseParagraph
\OneApplies
\begin{itemize}
    \item \AllApplyCase{exact}
    \begin{itemize}
        \item $\vc$ is the constraint given by the expression $\ve$, that is, $\ConstraintExact(\ve)$;
        \item applying $\reducetozopt$ to $\ve$ in $\tenv$ yields the optional integer $\vzopt$;
        \item checking that $\vzopt$ is not $\None$ yields $\True$\ProseTerminateAs{\NoBaseValue};
        \item view $\vzopt$ as the singleton set for the integer $\vz$;
        \item define $\vzs$ as the single element list containing $\vz$.
    \end{itemize}

    \item \AllApplyCase{range}
    \begin{itemize}
        \item $\vc$ is the constraint given by the expression $\veone$ and $\vetwo$, that is, \\
                $\ConstraintRange(\veone, \vetwo)$;
        \item applying $\reducetozopt$ to $\veone$ in $\tenv$ yields the optional integer $\vzoptone$;
        \item checking that $\vzoptone$ is not $\None$ yields $\True$\ProseTerminateAs{\NoBaseValue};
        \item view $\vzoptone$ as the singleton set for $\vvone$;
        \item applying $\reducetozopt$ to $\vetwo$ in $\tenv$ yields the optional integer $\vzopttwo$;
        \item checking that $\vzopttwo$ is not $\None$ yields $\True$\ProseTerminateAs{\NoBaseValue};
        \item view $\vzopttwo$ as the singleton set for $\vvtwo$;
        \item define $\vzs$ as based on the following cases for $\vvone$ and $\vvtwo$:
        \begin{itemize}
            \item the empty list, if $\vvone$ is greater than $\vvtwo$ (since there are no integers satisfying the constraint);
            \item the single element list for $\vvtwo$, if $\vvone$ is less than $\vvtwo$ and both are negative;
            \item the single element list for $0$, if $\vvone$ is negative and $\vvtwo$ is non-negative;
            \item the single element list for $\vvone$, if $\vvone$ is non-negative and $\vvtwo$ is greater or equal to $\vvone$.
        \end{itemize}
    \end{itemize}
\end{itemize}

\FormallyParagraph
\begin{mathpar}
\inferrule[exact]{
    \reducetozopt(\tenv, \ve) \typearrow \vzopt\\
    \checktrans{\vzopt \neq \None}{\NoBaseValue} \checktransarrow \True \OrTypeError\\\\
    \vzopt \eqname \some{\vz}
}{
    \constraintabsmin(\overname{\ConstraintExact(\ve)}{\vc}) \typearrow \overname{[\vz]}{\vzs}
}
\end{mathpar}

\begin{mathpar}
\inferrule[range]{
    \reducetozopt(\tenv, \veone) \typearrow \vzoptone\\
    \checktrans{\vzoptone \neq \None}{\NoBaseValue} \checktransarrow \True \OrTypeError\\\\
    \vzoptone \eqname \some{\vvone}\\
    \reducetozopt(\tenv, \vetwo) \typearrow \vzopttwo\\
    \checktrans{\vzopttwo \neq \None}{\NoBaseValue} \checktransarrow \True \OrTypeError\\\\
    \vzopttwo \eqname \some{\vvtwo}\\
    \vzs \eqdef {
        \begin{cases}
           \emptylist & \vvone > \vvtwo\\
           [\vvtwo] & \vvone \leq \vvtwo < 0\\
           [0] & \vvone < 0 \leq \vvtwo\\
           [\vvone] & 0 \leq \vvone \leq \vvtwo\\
        \end{cases}
    }
}{
    \constraintabsmin(\overname{\ConstraintRange(\veone, \vetwo)}{\vc}) \typearrow \vzs
}
\end{mathpar}

\TypingRuleDef{ListMinAbs}
\hypertarget{def-listminabs}{}
The function
\[
\listminabs(\overname{\KleeneStar{\Z}}{\vl}) \aslto \overname{\Z}{\vz}
\]
returns $\vz$ --- the integer closest to $0$ among the list
of integers in the list $\vl$. The result is biased towards positive integers. That is,
if two integers $x$ and $y$ have the same absolute value and $x$ is positive and $y$ is negative
then $x$ is considered closer to $0$.

\ExampleDef{Minimal Absolute Value}
The minimal absolute value of $[9, -3]$ is $-3$,
and the minimal absolute value of $[2, -2]$ is $2$.

\ProseParagraph
\OneApplies
\begin{itemize}
    \item \AllApplyCase{one}
    \begin{itemize}
        \item $\vl$ is the single element list for $\vz$.
    \end{itemize}

    \item \AllApplyCase{more\_than\_one}
    \begin{itemize}
        \item $\vl$ is the list where $\vzone$ is its \head\ and $\vltwo$ is its \tail;
        \item $\vltwo$ is not the empty list;
        \item applying $\listminabs$ to $\vltwo$ yields $\vztwo$;
        \item define $\vz$ based on $\vzone$ and $\vztwo$ by the following cases:
        \begin{itemize}
            \item $\vzone$ if the absolute value of $\vzone$ is less than the absolute value of $\vztwo$;
            \item $\vztwo$ if the absolute value of $\vzone$ is greater than the absolute value of $\vztwo$;
            \item $\vzone$ if $\vzone$ is equal to $\vztwo$;
            \item the absolute value of $\vzone$ if the absolute value of $\vzone$ is equal to the absolute value of $\vztwo$
                    and $\vzone$ is not equal to $\vztwo$;
        \end{itemize}
    \end{itemize}
\end{itemize}

\FormallyParagraph
\begin{mathpar}
\inferrule[one]{}{
    \listminabs(\overname{[\vz]}{\vl}) \typearrow \vz
}
\end{mathpar}

\begin{mathpar}
\inferrule[more\_than\_one]{
    \vlone \neq \emptylist\\
    \listminabs(\vltwo) = \vztwo\\
    {
        \vz \eqdef \begin{cases}
            \vzone & |\vzone| < |\vztwo|\\
            \vztwo & |\vzone| > |\vztwo|\\
            \vzone & \vzone = \vztwo\\
            |\vzone| & |\vzone| = |\vztwo| \land \vzone \neq \vztwo
        \end{cases}
    }
}{
    \listminabs(\overname{[\vzone] \concat \vltwo}{\vl}) \typearrow \vz
}
\end{mathpar}

