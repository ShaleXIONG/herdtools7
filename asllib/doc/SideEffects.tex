\chapter{Side Effects\label{chap:SideEffects}}

This chapter defines a static \emph{side effect analysis}.
The analysis aims to answer which expressions are \pureterm{}, \readonlyterm{}, or \symbolicallyevaluableterm{}.
For pure and read-only, the analysis is \emph{sound}: it proves that a \underline{sufficient condition} holds.

Intuitively, a pure expression is one whose evaluation does not depend on dynamic values.
In other words, a pure expression can be evaluated during typechecking.
A read-only expression is one whose evaluation does not affect the evaluation of any further expressions.
In other words, a read-only expression can be evaluated multiple times (or not at all) with no effect on the overall specification.
A \symbolicallyevaluableterm{} expression is compatible with symbolic reduction and equivalence testing (see \chapref{SymbolicEquivalenceTesting}).

\ChapterOutline
\begin{itemize}
  \item \secref{Purity} defines the concept of expression \purities{};
  \item \secref{SideEffectDescriptors} defines \sideeffectdescriptorsetsterm, which the type system infers
        for each kind of ASL construct;
  \item \secref{SideEffectSets} defines functions used to construct, modify, and query sets
        of \sideeffectdescriptorsetsterm{}.
        These are used throughout the type system to infer \sideeffectdescriptorsetsterm{}
        and to ensure each construct meets its requirements in terms of side effects.
\end{itemize}

%%%%%%%%%%%%%%%%%%%%%%%%%%%%%%%%%%%%%%%%%%%%%%%%%%%%%%%%%%%%%%%%%%%%%%%%%%%%%%%%
\section{Purity\label{sec:Purity}}
%%%%%%%%%%%%%%%%%%%%%%%%%%%%%%%%%%%%%%%%%%%%%%%%%%%%%%%%%%%%%%%%%%%%%%%%%%%%%%%%

We now define configurations used to determine whether expressions are \pureterm{} and \readonlyterm{} by the side effect analysis, as explained below.
These are known as \purities{}.

\RenderType{TPurity}
\BackupOriginalType{
\[
\TPurity \triangleq \{ \SEPure,\; \SEReadonly,\; \SEImpure \}
\]
}
\begin{description}
    \item[\RenderConstant{SE_Pure}] describes evaluation of a \pureterm{} construct, one that can be evaluated at type-checking.

\item[\RenderConstant{SE_Readonly}] describes evaluation of a \readonlyterm{} construct, one that can read mutable global storage elements but not modify them.

\item[\RenderConstant{SE_Impure}] describes evaluation of a construct that is neither \pureterm{} nor \readonlyterm{}.
\end{description}

Formally, \purities{} are totally ordered via $\purityless$ as follows:
\hypertarget{def-purityless}{}
\[
\SEImpure \purityless \SEReadonly \purityless \SEPure \enspace.
\]
Additionally, we define the greater-than-or-equal ordering as follows:
\hypertarget{def-puritygeq}{}
\[
f \puritygeq f' \triangleq \lnot(f \purityless f') \enspace.
\]

%%%%%%%%%%%%%%%%%%%%%%%%%%%%%%%%%%%%%%%%%%%%%%%%%%%%%%%%%%%%%%%%%%%%%%%%%%%%%%%%
\section{Side Effect Descriptors\label{sec:SideEffectDescriptors}}
%%%%%%%%%%%%%%%%%%%%%%%%%%%%%%%%%%%%%%%%%%%%%%%%%%%%%%%%%%%%%%%%%%%%%%%%%%%%%%%%

\hypertarget{def-sideeffectdescriptorterm}{}
We now define \sideeffectdescriptorsterm,
which are configurations used to describe side effects, as explained below:
\RenderType{TSideEffect}
\BackupOriginalType{
\[
\TSideEffect \triangleq \left\lbrace
\begin{array}{ll}
    \LocalEffect(\overname{\TPurity}{\vp}) & \cup \\
    \GlobalEffect(\overname{\TPurity}{\vp}) & \cup \\
    \Immutability(\overname{\Bool}{\vimmutable}) &
\end{array} \right.
\]
}

\hypertarget{def-term-LocalEffect}{}
\begin{description}
    \item[$\LocalEffect$] a \LocalEffectTerm{} describes an evaluation of a construct whose minimum \purity{} for local reads/writes is $\vp$.
    \hypertarget{def-term-GlobalEffect}{}
    \item[$\GlobalEffect$] a \GlobalEffectTerm{} describes an evaluation of a construct whose minimum \purity{} for global reads/writes is $\vp$.
    \hypertarget{def-term-Immutability}{}
    \item[$\Immutability$] an \ImmutabilityTerm{} describes an evaluation of a construct that accesses storage elements whose immutability is given by $\vb$ (that is, if $\vb$ is $\True$ then accesses are only to \texttt{constant}, \texttt{config}, or \texttt{let} elements).
\end{description}

We now define a few helper functions over \sideeffectdescriptorsterm.

\TypingRuleDef{SideEffectIsPure}
\hypertarget{def-sideeffectispure}{}
\[
    \sideeffectispure(\overname{\TSideEffect}{\vs}) \aslto \overname{\Bool}{\vb}
\]
defines whether a \sideeffectdescriptorterm\ $\vs$ is considered \emph{\pureterm},
yielding the result in $\vb$.
Intuitively, a \emph{\pureterm} \sideeffectdescriptorterm\ helps to establish that
an expression can be evaluated during type-checking.

\ExampleDef{Pure, Read-only, and Symbolically Evaluable Side Effect Descriptors}
The table below shows all possible \sideeffectdescriptorsterm{} and whether they are \pureterm{}, \readonlyterm{}, and \symbolicallyevaluableterm{}.
Here, \Effect{} stands for either \LocalEffect{} or \GlobalEffect{}.

\begin{center}
\begin{tabular}{llll}
\textbf{\sideeffectdescriptorterm{}} & \textbf{\pureterm{}?} & \textbf{\readonlyterm{}?} & \textbf{\symbolicallyevaluableterm{}?}\\
\hline
$\Effect(\SEPure)$      & \True{}  & \True{}  & \True{}  \\
$\Effect(\SEReadonly)$  & \False{} & \True{}  & \True{}  \\
$\Effect(\SEImpure)$    & \False{} & \False{} & \False{} \\
$\Immutability(\True)$  & \True{}  & \True{}  & \True{}  \\
$\Immutability(\False)$ & \False{} & \True{}  & \False{} \\
\end{tabular}
\end{center}

The specification in \listingref{SideEffectIsPure} shows examples of expressions
and statements, and their corresponding \sideeffectdescriptorsterm{}.
Notice that the call to \verb|factorial(10)| is \readonlyterm, even though it is recursive.

\ASLListing{Side effect descriptors and whether they are pure and read-only and symbolically evaluable}{SideEffectIsPure}{\typingtests/TypingRule.SideEffectIsPure.asl}

\ProseParagraph
Define $\vb$ as $\True$ if and only if $\vs$ is either
    a \LocalEffectTerm{} with \purity{} $\SEPure$,
    a \GlobalEffectTerm{} with \purity{} $\SEPure$,
    or an \ImmutabilityTerm{} with immutability $\True$.

\FormallyParagraph
\begin{mathpar}
\inferrule{
    \vb \eqdef \vs \in \{\LocalEffect(\SEPure), \GlobalEffect(\SEPure), \Immutability(\True)\}
}{
    \sideeffectispure(\vs) \typearrow \vb
}
\end{mathpar}

\TypingRuleDef{SideEffectIsReadonly}
\hypertarget{def-sideeffectisreadonly}{}
\[
    \sideeffectisreadonly(\overname{\TSideEffect}{\vs}) \aslto \overname{\Bool}{\vb}
\]
defines whether a \sideeffectdescriptorsterm\ $\vs$ is considered \emph{\readonlyterm},
yielding the result in $\vb$.
Intuitively, a \emph{\readonlyterm} \sideeffectdescriptorterm\ helps to establish that
an expression evaluates without modifying values of storage elements.

See \ExampleRef{Pure, Read-only, and Symbolically Evaluable Side Effect Descriptors}.

\ProseParagraph
Define $\vb$ as $\True$ if and only if $\vs$ is either
    a \LocalEffectTerm{} with \purity{} greater than or equal to $\SEReadonly$,
    a \GlobalEffectTerm{} with \purity{} greater than or equal to $\SEReadonly$,
    or an \ImmutabilityTerm.

\FormallyParagraph
\begin{mathpar}
\inferrule{
    {
        \vb \eqdef \begin{cases}
            \vp \puritygeq \SEReadonly & \vs = \LocalEffect(\vp) \\
            \vp \puritygeq \SEReadonly & \vs = \GlobalEffect(\vp) \\
            \True                      & \vs = \Immutability(\Ignore)
        \end{cases}
    }
}{
    \sideeffectisreadonly(\vs) \typearrow \vb
}
\end{mathpar}

\TypingRuleDef{SideEffectIsSymbolicallyEvaluable}
\hypertarget{def-sideeffectissymbolicallyevaluable}{}
\[
    \sideeffectissymbolicallyevaluable(\overname{\TSideEffect}{\vs}) \aslto \overname{\Bool}{\vb}
\]
defines whether a \sideeffectdescriptorsterm\ $\vs$ is considered \emph{\symbolicallyevaluableterm},
yielding the result in $\vb$.
Intuitively, a \emph{symbolically evaluable} \sideeffectdescriptorterm\ helps establish that
an expression evaluates without modifying any storage element,
and always yielding the same result, that is, deterministically.

See \ExampleRef{Pure, Read-only, and Symbolically Evaluable Side Effect Descriptors}.

\ProseParagraph
\Proseeqdef{$\vb$}{$\True$ if and only if one of the following conditions holds:
  \begin{itemize}
    \item $\vs$ is a \LocalEffectTerm{} with \purity{} greater than or equal to $\SEReadonly$,
    \item $\vs$ is a \GlobalEffectTerm{} with \purity{} greater than or equal to $\SEReadonly$,
    \item $\vs$ is an \ImmutabilityTerm{} with immutability equal to $\True$.
  \end{itemize}
}

\FormallyParagraph
\begin{mathpar}
\inferrule{
    {
        \vb \eqdef \begin{cases}
            \vp \puritygeq \SEReadonly & \vs = \LocalEffect(\vp) \\
            \vp \puritygeq \SEReadonly & \vs = \GlobalEffect(\vp) \\
            \vbone                     & \vs = \Immutability(\vbone)
        \end{cases}
    }
}{
    \sideeffectissymbolicallyevaluable(\vs) \typearrow \vb
}
\end{mathpar}

%%%%%%%%%%%%%%%%%%%%%%%%%%%%%%%%%%%%%%%%%%%%%%%%%%%%%%%%%%%%%%%%%%%%%%%%%%%%%%%%
\section{Side Effect Sets\label{sec:SideEffectSets}}
%%%%%%%%%%%%%%%%%%%%%%%%%%%%%%%%%%%%%%%%%%%%%%%%%%%%%%%%%%%%%%%%%%%%%%%%%%%%%%%%

% Transliteration note:
% As with the previous transliteration of the fine-grained analysis, ASLRef's
% implementation does not track *sets* of side-effects.  Rather, it considers
% only the least pure `LocalEffect`/`GlobalEffect` descriptor, and the most
% mutable `Immutability` descriptor. These are efficiently represented as a
% record with three fields.

\TypingRuleDef{SideEffectsLDK}
\hypertarget{def-sesldk}{}
The function
\[
  \sesldk(\overname{\localdeclkeyword}{\ldk}) \aslto \overname{\pow{\TSideEffect}}{\vs}
\]
constructs a \sideeffectsetterm{} $\vs$ corresponding to a read of a storage element declared with a local declaration keyword $\ldk$.

\ExampleDef{The Side Effects of Local Storage Declarations}
The specification in \listingref{SideEffectsLDK} shows examples of local storage declarations
and their corresponding \sideeffectdescriptorsetsterm{} (in comments).
\ASLListing{The side effects of local storage declarations}{SideEffectsLDK}{\typingtests/TypingRule.SideEffectsLDK.asl}

\ProseParagraph
\AllApply
\begin{itemize}
  \item \Proseeqdef{$\vb$} as $\True$ if and only if $\ldk$ is $\LDKLet$;
  \item the result is a set containing a \LocalEffectTerm{} with \purity{} $\SEReadonly$, and an \ImmutabilityTerm{} with immutability $\vb$.
\end{itemize}

\FormallyParagraph
\begin{mathpar}
\inferrule{
  \vb \eqdef (\ldk = \LDKLet)
}{
  \sesldk(\ldk) \typearrow \{ \LocalEffect(\SEReadonly), \Immutability(\vb) \}
}
\end{mathpar}

\TypingRuleDef{SideEffectsGDK}
\hypertarget{def-sesgdk}{}
The function
\[
  \sesgdk(\overname{\globaldeclkeyword}{\gdk}) \aslto \overname{\pow{\TSideEffect}}{\vs}
\]
constructs a \sideeffectsetterm{} $\vs$ corresponding to a read of a storage element declared with a global declaration keyword $\gdk$.

\ExampleDef{The Side Effects of Global Storage Declarations}
The specification in \listingref{SideEffectsGDK} shows examples of global storage declarations
and their corresponding \sideeffectdescriptorsetsterm{} (in comments).
\ASLListing{The side effects of global storage declarations}{SideEffectsGDK}{\typingtests/TypingRule.SideEffectsGDK.asl}

\ProseParagraph
\AllApply
\begin{itemize}
  \item \Proseeqdef{$\vpurity$}{$\SEPure$ if $\gdk$ is $\GDKConstant$, and $\SEReadonly$ otherwise};
  \item \Proseeqdef{$\vb$} as $\False$ if and only if $\gdk$ is $\GDKVar$;
  \item the result is a set containing a \GlobalEffectTerm{} with \purity{} $\vpurity$ and \ImmutabilityTerm{} with immutability $\vb$.
\end{itemize}

\FormallyParagraph
\begin{mathpar}
\inferrule{
  \vpurity \eqdef \choice{\gdk = \GDKConstant}{\SEPure}{\SEReadonly} \\
  \vb \eqdef (\gdk \neq \GDKVar)
}{
  \sesgdk(\gdk) \typearrow \{ \GlobalEffect(\vpurity), \Immutability(\vb) \}
}
\end{mathpar}

\TypingRuleDef{SESIsSymbolicallyEvaluable}
\hypertarget{def-issymbolicallyevaluable}{}
\hypertarget{def-symbolicallyevaluable}{}
The function
\[
  \issymbolicallyevaluable(\overname{\TSideEffectSet}{\vses}) \aslto \overname{\Bool}{\bv}
\]
tests whether a set of \sideeffectdescriptorsterm\ $\vses$ are all \symbolicallyevaluableterm,
yielding the result in $\vb$.

See \ExampleRef{Checking Whether Expressions are Symbolically Evaluable}.

\ExampleDef{Why Symbolically Evaluability Matters}
The specification in \listingref{SESIsSymbolicallyEvaluable} is well-typed.
Specifically, it shows how the fact that an expression is \symbolicallyevaluableterm{}
is used to reason about the equivalence of bitwidth expressions.
\ASLListing{Why Symbolically Evaluability Matters}{SESIsSymbolicallyEvaluable}{\typingtests/TypingRule.SESIsSymbolicallyEvaluable.asl}

\ProseParagraph
Define $\vb$ as $\True$ if and only if every \sideeffectdescriptorterm\ $\vs$ in $\vses$
is \symbolicallyevaluableterm.

\FormallyParagraph
\begin{mathpar}
\inferrule{
  \vb \eqdef \bigwedge_{\vs\in\vses} \sideeffectissymbolicallyevaluable(\vs)
}{
  \issymbolicallyevaluable(\vses) \typearrow \vb
}
\end{mathpar}

\hypertarget{def-checksymbolicallyevaluable}{}
\TypingRuleDef{CheckSymbolicallyEvaluable}
The function
\[
  \checksymbolicallyevaluable(\overname{\TSideEffectSet}{\vses}) \aslto
  \{\True\} \cup \typeerror
\]
returns $\True$ if the set of \sideeffectdescriptorsterm\ $\vses$ is \symbolicallyevaluableterm.
\ProseOtherwiseTypeError

\ExampleDef{Checking Whether Expressions are Symbolically Evaluable}
In \listingref{CheckSymbolicallyEvaluable},
the expression \verb|plus_mul{SEVEN, 4}(SEVEN, 4)| is \symbolicallyevaluableterm.
This is checked in the following constructs:
\begin{itemize}
    \item the declaration of the \verb|data| bitfield of the \verb|Data| type;
    \item the pattern expression \verb|35 IN { plus_mul{SEVEN, 4}(SEVEN, 4) }|;
    \item the width of the bitvector for the local storage element \verb|x|;
    \item the type of the parameter \verb|N| of \verb|foo|; and
    \item the length of the array declared for \verb|arr|.
\end{itemize}
\ASLListing{Symbolically evaluable expressions}{CheckSymbolicallyEvaluable}{\typingtests/TypingRule.CheckSymbolicallyEvaluable.asl}

\ProseParagraph
\AllApply
\begin{itemize}
  \item applying $\issymbolicallyevaluable$ to $\vses$ yields $\vb$;
  \item the result is $\True$ if $\vb$ is $\True$, otherwise it is a \typingerrorterm{} indicating that the expression
  is not \symbolicallyevaluableterm.
\end{itemize}

\FormallyParagraph
\begin{mathpar}
\inferrule{
  \issymbolicallyevaluable(\vses) \typearrow \vb\\
  \checktrans{\vb}{\SideEffectViolation} \checktransarrow \True \OrTypeError
}{
  \checksymbolicallyevaluable(\vses) \typearrow \True
}
\end{mathpar}
\CodeSubsection{\CheckSymbolicallyEvaluableBegin}{\CheckSymbolicallyEvaluableEnd}{../Typing.ml}

\TypingRuleDef{SESIsReadonly}
\hypertarget{def-sesisreadonly}{}
The function
\[
    \sesisreadonly(\overname{\TSideEffectSet}{\vses}) \aslto \overname{\Bool}{\vb}
\]
tests whether all side effects in the set $\vses$ are \readonlyterm{}, yielding the result in $\vb$.

\ProseParagraph
Define $\vb$ as $\True$ if and only if $\sideeffectisreadonly$ holds for
every \sideeffectdescriptorterm\ $\vs$ in $\vses$.

\ExampleDef{Checking whether expressions are read-only}
The specification in \listingref{SESIsReadonly} checks whether the following expressions are \readonlyterm{}:
\begin{itemize}
    \item \verb|y > x| inside the \assertionstatementterm;
    \item \verb|ARBITRARY : integer{1..1000} > g| inside the \assertionstatementterm,
          since \\
          \verb|ARBITRARY| expressions are \readonlyterm{};
    \item \verb|x| and \verb|y| in the \forstatementterm.
\end{itemize}
\ASLListing{Read-only expressions}{SESIsReadonly}{\typingtests/TypingRule.SESIsReadonly.asl}

The specifications in
\listingref{SESIsReadonly-bad1} and
\listingref{SESIsReadonly-bad2},
are ill-typed due to expressions that are not \readonlyterm{}
where read-only expressions are expected (as explained in the comments).
\ASLListing{Non-read-only expression in assertion}{SESIsReadonly-bad1}{\typingtests/TypingRule.SESIsReadonly.bad1.asl}
\ASLListing{Non-read-only expression in for loop bounds}{SESIsReadonly-bad2}{\typingtests/TypingRule.SESIsReadonly.bad2.asl}

\FormallyParagraph
\begin{mathpar}
\inferrule{
    \bigwedge_{\vs\in\vses} \sideeffectisreadonly(\vs)
}{
    \sesisreadonly(\vses) \typearrow \True
}
\end{mathpar}

\TypingRuleDef{SESIsPure}
\hypertarget{def-sesispure}{}
The function
\[
    \sesispure(\overname{\TSideEffectSet}{\vses}) \aslto \overname{\Bool}{\vb}
\]
tests whether all side effects in the set $\vses$ are \pureterm{}, yielding the result in $\vb$.

\ProseParagraph
Define $\vb$ as $\True$ if and only if $\sideeffectispure$ holds for
every \sideeffectdescriptorterm\ $\vs$ in $\vses$.

\ExampleDef{Checking Whether Expressions are Pure}
The specification in \listingref{SESIsPure} is well-typed.
Specifically, all expressions match their expected purities (see comments).
\ASLListing{Pure expressions}{SESIsPure}{\typingtests/TypingRule.SESIsPure.asl}

The specification in \listingref{SESIsPure-bad} is ill-typed, since expressions
appearing in bitvector widths for bitvector types with bitfields must be pure.
\ASLListing{Purity violation}{SESIsPure-bad}{\typingtests/TypingRule.SESIsPure.bad.asl}

\FormallyParagraph
\begin{mathpar}
\inferrule{
    \bigwedge_{\vs\in\vses} \sideeffectispure(\vs)
}{
    \sesispure(\vses) \typearrow \True
}
\end{mathpar}

\TypingRuleDef{SESForSubprogram}
\hypertarget{def-sesforsubprogram}{}
The function
\[
  \sesforsubprogram(\overname{\Option{\qualifier}}{\vqualifier}) \aslto \overname{\pow{\TSideEffect}}{\vs}
\]
produces a \sideeffectsetterm{} given a subprogram qualifier $\vqualifier$.

\ExampleDef{Side effects for subprograms}
The specification in \listingref{SESForSubprogram} shows example subprogram declarations and the \sideeffectdescriptorsterm{} derived from their subprogram qualifiers (see comments).
Note that each subprogram shares the same body, the side effects are given by the qualifier keyword.
\ASLListing{Side effects for subprograms}{SESForSubprogram}{\typingtests/TypingRule.SESForSubprogram.asl}

\ProseParagraph
\OneApplies
\begin{itemize}
  \item \AllApplyCase{none\_or\_noreturn}
  \begin{itemize}
    \item $\vqualifier$ is either $\None$ or $\some{\Noreturn}$;
    \item the result is a set containing a \GlobalEffectTerm{} with \purity{} \\ $\SEImpure$, and an \ImmutabilityTerm{}
          with immutability \\ $\False$.
  \end{itemize}

  \item \AllApplyCase{some\_readonly}
  \begin{itemize}
    \item $\vqualifier$ is $\some{\Readonly}$;
    \item the result is a set containing a \GlobalEffectTerm{} with \purity{} \\ $\SEReadonly$, and an \ImmutabilityTerm{} with immutability $\False$.
  \end{itemize}

  \item \AllApplyCase{some\_pure}
  \begin{itemize}
    \item $\vqualifier$ is $\some{\Pure}$;
    \item the result is a set containing a \GlobalEffectTerm{} with \purity{} $\SEPure$, and an \ImmutabilityTerm{} with immutability $\True$.
  \end{itemize}
\end{itemize}

\FormallyParagraph
\begin{mathpar}
\inferrule[none\_or\_noreturn]{
  \vqualifier = \None \lor \vqualifier = \some{\Noreturn}\\
  \vs \eqdef \{ \GlobalEffect(\SEImpure), \Immutability(\False) \}
}{
  \sesforsubprogram(\vqualifier) \typearrow \vs
}
\end{mathpar}

\begin{mathpar}
\inferrule[some\_readonly]{
  \vs \eqdef \{ \GlobalEffect(\SEReadonly), \Immutability(\False) \}
}{
  \sesforsubprogram(\overname{\some{\Readonly}}{\vqualifier}) \typearrow \vs
}
\end{mathpar}

\begin{mathpar}
\inferrule[some\_pure]{
  \vs \eqdef \{ \GlobalEffect(\SEPure), \Immutability(\True) \}
}{
  \sesforsubprogram(\overname{\some{\Pure}}{\vqualifier}) \typearrow \vs
}
\end{mathpar}
